%!TEX root = FunkAnalysis.tex


\chapter{Räume}

Sei $X$ ein Vektorraum, $dim X < \infty $ und sei $x = (x_{1},\dotsc,x_{n})^{T} \in \MdR^{n}$
\begin{align*}
	& \|x\|_{2} := \left(\sum_{k = 1}^{n} \|x_{i}^{2}\| \right)^{\frac{1}{2}}\\
	& \|x\|_{\infty} := \smash{\displaystyle\max_{i = 1}^{n}}  \|x_{i}\|		
\end{align*}

Diese Normen sind äquivalent, denn:
$\| x \|_{\infty} \leq \| x \|_{2} \leq n^{\frac{1}{2}} \| x \|_{\infty}$ \newline

\begin{satz*}[Heine-Borel] \label{s:1-heineborell}  \todo{Nicht Heine-Borel!?}
$A \subset \MdR$ beschränkt. Dann hat jede Folge $(x_{n})_{n \in \MdN} \subset A$ eine konvergente Teilfolge.
\end{satz*}

\begin{beispiel}
$X = C[0, 1] = \{ f : [0, 1] \rightarrow \MdR: \text{ stetig auf } [0, 1] \}$
\begin{align*}
\begin{rcases*}
\| f \|_{2} := \left( \int_{0}^{1} \| f(t) \|^{2} dt \right)^{\frac{1}{2}} \\
\| f \|_{\infty} :=  \smash{\displaystyle\max_{t \in [0, 1]}}  \| f(t) \|
\end{rcases*}\text{ dabei gilt } \| f \|_{2} \leq \| f \|_{\infty} \leq \dotsc
\end{align*}
\newline
Aber:$\hspace{0.5cm} f_n(t) = §BILD§, \hspace{0.5cm} \| f_n \|_{\infty} = 1, \hspace{0.5cm} \| f_n \|_{2} \xrightarrow[n \rightarrow \infty]{} 0  $ \\
$ \| f_{n} - f_{m} \| = 1 \text{ für } n \neq m \Rightarrow $ \hyperref[s:1-heineborell]{ Satz von Heine-Borel} gilt im $\infty$-dimensionalen i.A. nicht!
\end{beispiel}

\chapter{Operatoren}

Sei $N = dim X, M = dim Y$ und seien $(e_n)$ bzw. $(f_n)$ Basen von $X$ bzw. $Y$.
Für $T : X \rightarrow Y$ gegeben durch: \\
\[ §Komm. Diagramm§ \] \\
Daraus folgt:
\begin{itemize}
	\item $T$ ist stetig
	\item X = Y \gdw T injektiv \gdw T surjektiv (Dimensionsformel) \\
	(Die Gleichung $Tx = y$ ist eindeutig lösbar $\gdw$ Gleichung hat für alle $y \in Y$ eine Lösung.)
	\item Falls A selbstadjungiert ist, d.h. $A = A^{*}$, gibt es eine Basis aus Eigenvektoren $(e_{n})$ von $A$, d.h. $ T( \sum \alpha_{n} e_{n} ) = \sum \lambda_{n} \alpha_{n} e_{n}$, wobei $\lambda_{n}$ Eigenwerte sind $A =
			\begin{pmatrix}
				\diagentry{\lambda_{1}}\\
				&\diagentry{\xddots}\\
				&&\diagentry{\lambda_{n}}\\
			\end{pmatrix} $ 
\end{itemize}

\begin{beispiel}[1]
$X = C^{1}[0, 1] = \{ f : [0, 1] \rightarrow \MdR: \text{ stetig auf } [0, 1] \}$ \\
\end{beispiel}