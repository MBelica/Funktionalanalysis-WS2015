%!TEX root = FunkAnalysis.tex


\chapter{Räume}

\section{Einführung}
%\subsection{Vorbemerkungen zu $\C$}

Sei X ein Vektorraum, $dim X < \infty $ und sei $x = (x_{1},\dotsc,x_{n})^{T} \in \MdR^{n}$
\begin{align*}
	& \|x\|_{2} := \left(\sum_{k = 1}^{n} \|x_{i}^{2}\| \right)^{\frac{1}{2}}\\
	& \|x\|_{\infty} := \smash{\displaystyle\max_{i = 1}^{n}}  \|x_{i}\|		
\end{align*}

Diese Normen sind äquivalent, denn:
$\| x \|_{\infty} \leq \| x \|_{2} \leq n^{\frac{1}{2}} \| x \|_{\infty}$ \newline

\begin{satz*}[Heine-Borel] \label{s:1-heineborell} 
$A \subset \MdR$ beschränkt. Dann hat jede Folge $(x_{n})_{n \in \MdN} \subset A$ eine konvergente Teilfolge.
\end{satz*}


%\begin{satz}[vgl. Analysis 1, Theorem 4.12] \label{satz1.3} Sei
%\[f(z) = \sum_{n=0}^\infty a_n(z-c)^n,\ z\in B(c,\rho),\]
%eine Potenzreihe mit Konvergenzradius $\rho > 0$. Dann ist $f\in H(B(c,\rho))$ und
%\[f^\prime(z) = \sum_{n=1}^\infty n a_n(z-c)^{n-1} =: g(z)\ (\forall z\in B(c,\rho)),\]
%\end{satz}
%\begin{beweis}
%	 Wie in Analysis 1 zeigt man: $g$ hat Konvergenzradius 
%	\begin{itemize}
%		\item $p_n(w) \ra 0$, $w\ra z$ (für jedes feste $n$)
%		\item $|p_n(w)|$
%	\end{itemize}
%\end{beweis}
%
%
%
%\begin{beispiel} \label{bsp1.1}
%\begin{enumerate}
%\item $f(z) = \bar{z}$, $\kmplx{z}$, ist nirgends komplex differenzierbar, obwohl $f(x,y) = \cmplx{x}{-y}$ reell $C^\infty$ ist. Denn $u(x,y) = x$, $v(x,y) = -y$; also 
%\[\frac{\partial u}{\partial x}(x,y) = 1 \neq -1 = \frac{\partial v}{\partial y}(x,y),\]
%was \eqref{CR}$_1$ widerspricht.
%\item $f(z) = |z|^2 = x^2 + y^2$, $\kmplx{z}$, ist nur in $z=0$ komplex differenzierbar, denn hier ist $u(x,y) = x^2+y^2$, $v(x,y) = 0$ und somit:
%\[\frac{\partial u}{\partial x}(x,y) = 2x \stackrel{!}{=} \frac{\partial v}{\partial y}(x,y) = 0 \gdw x=0,\]
%\[\frac{\partial u}{\partial y}(x,y) = 2y \stackrel{!}{=} -\frac{\partial v}{\partial x}(x,y) = 0 \gdw y=0.\]
%\item $\displaystyle f(z) = \frac{1}{z} = \frac{\bar{z}}{|z|^2} = \underbrace{\frac{x}{x^2+y^2}}_{=u} + \i\underbrace{\frac{-y}{x^2+y^2}}_{=v}$ ist holomorph für $z\neq 0$ (Bem.~\ref{bem1.2}).
%\end{enumerate}
%\end{beispiel}