%!TEX root = FunkAnalysis.tex

\chapter{Lineare Operatoren auf Banachräumen}
\section{Einführung}
\subsection{Räume}

Sei $X$ ein Vektorraum, $dim X < \infty $ und sei $x = (x_{1},\dotsc,x_{n})^{T} \in \MdR^{n}$
\begin{align*}
	& \|x\|_{2} := \left(\sum_{k = 1}^{n} \|x_{i}^{2}\| \right)^{\frac{1}{2}}\\
	& \|x\|_{\infty} := \smash{\displaystyle\max_{i = 1}^{n}}  \|x_{i}\|		
\end{align*}

Diese Normen sind äquivalent, denn:
$\| x \|_{\infty} \leq \| x \|_{2} \leq n^{\frac{1}{2}} \| x \|_{\infty}$ \newline

\begin{satz*}[Bolzano-Weierstraß] \label{s:1-bolzanoweierstrass}
$A \subset \MdR$ beschränkt. Dann hat jede Folge $(x_{n})_{n \in \MdN} \subset A$ eine konvergente Teilfolge.
\end{satz*}

\begin{beispiel}
$X = C[0, 1] = \{ f : [0, 1] \rightarrow \MdR: \text{ stetig auf } [0, 1] \}$
\begin{align*}
\begin{rcases*}
\| f \|_{2} := \left( \int_{0}^{1} \| f(t) \|^{2} dt \right)^{\frac{1}{2}} \\
\| f \|_{\infty} :=  \smash{\displaystyle\max_{t \in [0, 1]}}  \| f(t) \|
\end{rcases*}\text{ dabei gilt } \| f \|_{\infty} \leq \| f \|_{2} \leq \dotsc 
\end{align*}
\newline
Aber:$\hspace{0.25cm} f_n(t) = Bild, \hspace{0.5cm} \| f_n \|_{\infty} = 1, \hspace{0.5cm} \| f_n \|_{2} \xrightarrow[n \rightarrow \infty]{} 0  $ \\
$ \| f_{n} - f_{m} \| = 1 \text{ für } n \neq m \Rightarrow $ \hyperref[s:1-bolzanoweierstrass]{Satz von Bolzano-Weierstraß} gilt im $\infty$-dimensionalen i.A. nicht!
\end{beispiel}

\subsection{Operatoren}

Sei $N = dim X, M = dim Y$ und seien $(e_n)$ bzw. $(f_n)$ Basen von $X$ bzw. $Y$.
Für $T : X \rightarrow Y$ gegeben durch: \\
\[ §Komm. Diagramm§ \] \\
Daraus folgt:
\begin{itemize}
	\item $T$ ist stetig
	\item X = Y \gdw T injektiv \gdw T surjektiv (Dimensionsformel) \\
	(Die Gleichung $Tx = y$ ist eindeutig lösbar $\gdw$ Gleichung hat für alle $y \in Y$ eine Lösung.)
	\item Falls A selbstadjungiert ist, d.h. $A = A^{*}$, gibt es eine Basis aus Eigenvektoren $(e_{n})$ von $A$, d.h. $ T( \sum_{n=1}^{N} \alpha_{n} e_{n} ) = \sum{n=1}^{N} \lambda_{n} \alpha_{n} e_{n}$, wobei $\lambda_{n}$ Eigenwerte sind $A =
			\begin{pmatrix}
				\diagentry{\lambda_{1}}\\
				&\diagentry{\xddots}\\
				&&\diagentry{\lambda_{n}}\\
			\end{pmatrix} $
\end{itemize}

\begin{beispiel}[1]
$X = C^{1}[0, 1] = \{ f : [0, 1] \rightarrow \MdR: \text{ stetig auf } [0, 1] \}$ 
$T f = f', T : X \rightarrow Y \text{ stetig.} $ \\
(Aber: $T: C[0, 1] \rightarrow C[0, 1]$, hier ist $T$ nicht definiert.) \\
$T$ ist nicht stetig bzgl. $\| \cdot \|_{\infty}$-Norm, da:
\[ f_{n}(t) = \frac{1}{\sqrt{n}}e^{int}, \hspace{0.5cm}  \text{dann: } \|f_{n}\| \rightarrow 0 \text{ für } t \rightarrow \infty \]	
\[ T f_{n}(t) = i \sqrt{n} e^{int}, \hspace{0.5cm} \text{mit: } \| T f_{n} \|_{\infty} \rightarrow \infty, \text{ für } n \rightarrow \infty \]
\end{beispiel}

\begin{beispiel}[2]
$X = L_{2} = \{ (a_{n}): \left( \sum_{n \geq 1}^{\infty} \| a_{n} \| \right)^{\frac{1}{2}} < \infty \}$	\\
$T ( a_{1}, a_{2}, a_{3}, ...) = ( 0, a_{1}, a_{2}, a_{3}, ...)$ \\
$T$ ist injektiv, aber nicht surjektiv
\end{beispiel}

\subsection{Anwendungen}

\begin{enumerate}
	\item Fredholm'sche Integralglechungen
	\item Dirichletproblem
	\item Sturm-Liouville Problem
\end{enumerate}


\newpage
\section{Normierte Räume}

\begin{definition}
Sei $X$ ein Vektorraum über $\MdK \in\{ \MdR, \MdC \}$ \\
Eine Abbildung  $\| \cdot\| \cdot \| \| : X \rightarrow \MdR_{+}$ heißt eine Norm, wenn
\begin{enumerate}
	\item $ \| x \| \geq 0, \| x \| = 0 \gdw x = 0 $
	\item $\ | \lambda x \| = | \lambda \| x \| $
	\item $ \| x + y \| \leq \| x \| + \| y \| $
\end{enumerate}	
\end{definition}

\begin{bemerkung} Falls $ \| \cdot \| $ all die oben genannten Eigenschaften erfüllt außer $ \| x \| = 0 \Rightarrow x = 0 $, dann heißt $ \| \cdot \| $ Halbnorm
\end{bemerkung}

\begin{vereinbarung} 
Die Menge $ U_{X} = \{ x \in X:  \|x \| \leq 1 \}$ heißt \textbf{Einheitskugel}. \\
Eine Folge $(x_{n})$ des normierten Raums $X$ \textbf{konvergiert} gegen ein $ x \in X $, falls  $\| x_{n} - x \| \xrightarrow[n \rightarrow \infty]{} 0. $	
\end{vereinbarung}


\begin{bemerkung}
Für zwei Elemente $x, y \in (X, \| \cdot \|)$	... gilt die umgekehrte Dreiecksungleichung $( | \|x\| + \|y\| | \leq \| x + y \|)$
\end{bemerkung}

\begin{beispiel}
Sei $ X = \MdK^{n}, \hspace{0.5cm} X = ( X_{1}, \dotsc, X_{n}), \hspace{0.5cm} X_{i} = \MdK$ 
\begin{align*}
	\| x \|_{p} & = \left( \sum_{j = 1}^{n} |x_{j}^{p}| \right)^{\frac{1}{p}},  1 \leq p < \infty (p = 2: \text{ Euklidische Norm}) \\
	\| x \|_{\infty} & = \hspace{0.3cm} \sup_{j = 1}^{n} |x_{j}|	
\end{align*}

Beh: $\| \cdot \| $ ist Norm auf $\MdK^n$ für $1 \leq p \leq \infty$

$\| x + y \|_{\infty} = sup_{j = 1}^{k} |x_{j} + y_{j}| \leq \|x\|_{\infty} + \|y\|_{\infty} $
Für $p \in (1, \infty), p \neq 2:$ siehe Übungsaufgabe (Fall $p = 2$ läuft über Cauchy-Schwarz)

Beachte: $\|x\|_{\infty} \leq  \|x\|_{p} \leq n^{\frac{1}{p}} \|x\|_{\infty} \leq n \| x \|_{\infty}$
\end{beispiel}

\begin{definition}
	Zwei Normen $\| \cdot \|_{1}, \| \cdot \|_{2}$ heißen äquivalent auf $X$, falls es $0 < m, M < \infty$ gibt, so dass für alle $ x \in X$ gilt:
	\[ m \| x \|_{2} \leq \| x \|_{1} \leq M \| x \|_{2} \]
\end{definition}
 
\begin{satz}
	Auf einem endlich dimensionalen Vektorraum sind alle Normen äquivalent.
\end{satz}
\begin{beweis}
	Wähle eine algebraische Basis $(e_{1}, \dotsc, e_{n})$ von X, wobei $ n = dim X < \infty$. \\
	Definiere $|\|x\|| = \left(\sum_{i = 1}^{n} |x_{i}|^2\right)^{\frac{1}{2}}$, wobei $x = \sum_{i = 1}^{n} x_{i} e_{i}$ \\
	
	z. z. die gegeben Norm $|\| \cdot \||$ist äquivalent zu $\| \cdot \|$. \\ \\
	Beweis: \\
	In der einen Richtung betrachte: 
	\begin{align*}
		\| x \| = \| \sum_{i = 1}^{n} x_{i} e_{i} \| & \leq \sum_{i = 1}^{n} |x_{i}| \|  e_{i} \| \\ 
													& \leq \left( \sum_{i = 1}^{n} |x_{i}|^{2} \right)^\frac{1}{2}  \left( \sum_{i = 1}^{n} \| e_{i} \|^{2} \right)^\frac{1}{2} \\
													& =: \hspace{0.75cm} \nu \hspace{1.75cm} |\| x \||		
	\end{align*}
	
	Für die Umkehrung benutze die Funktion $J: \MdK^{n} \rightarrow X, \hspace{0.1cm} J(x_{1}, \dotsc, x_{n}) = \sum_{i = 1}^{n} x_{i} e_{i} $ \\
	\begin{align*}
 	 \text{Die Abbildung } y & \in \MdK^{n}  \rightarrow \| Jy \| \text{ ist stetig, denn} \\
 	 	 \| Jy \| - \| y \|_{\MdK^{n}} & = \left( \sum_{i = 1}^{n} |y_{i}|^{2} \right)^{\frac{1}{2}}, y = (y_{1}, \dotsc, y_{n}) \\
 	 	 \text{und } | \| Jy \| - \| Jz \| & \leq | \| Jy - Jz \| \leq | \| J(y - z) \| \\
 	 	 & \leq M |\|J (y - z) \|| \\
 	 	 & = M \| y - z \|_{\MdK^{n}} \\
 	 \end{align*}
 Daraus folgt die Stetigkeit von $ Jy \rightarrow \| Jy \| \in \MdR $ \\ \\
	Sei $S = \{y \in \MdK^{n}: \| y \|_2 = 1 \}$. Dann ist $S$ abgeschlossen und beschränkt.
	Die Abbildung $N := y \in S \rightarrow \| Jx \| > 0$ ist wie in (*) gezeigt stetig. Nach Analysis nimmt $N$ sein Minimum in einem Punkt $y_{0} \in S$ an. Setze
		\begin{align*}
			m = \inf\{\| x \| : |\| x \|| = 1\} & = \inf \{ \| Jy \|: y \in S \} \\
												& = \| J y_{0} \| > 0 \\ \\
			\text{Also } m \leq \| \frac{x}{ |\| x \|| } \| =  \frac{ \| x \| }{ |\| x \|| }  &\Rightarrow |\| x \|| \leq m \| x \|
		\end{align*}
\end{beweis}

\begin{prop}
Für zwei Normen $\| \cdot \|_{1}, \| \cdot \|_{2}$ auf $X$ sind äquivalent:
\begin{enumerate}
	\item $\| \cdot \|_{1}, \| \cdot \|_{2}$ sind äquivalent
	\item Für alle $(x_{n}) \subset X$, $x \in X$ gilt $\| x_{m} - x \|_{1} \rightarrow 0 \gdw \| x_{m} - x \|_{2} \rightarrow 0 $
	\item Für alle $(x_{n}) \subset X$ gilt $\| x_{m} |_{1} \rightarrow 0 \gdw \| x_{m} \|_{2} \rightarrow 0 $
	\item Es gibt Konstanten $0 < m$, $M < \infty$, so dass $m U_{(X, \| \cdot \|_{1})} \leq U_{(X, \| \cdot \|_{2})} \leq M U_{(X, \| \cdot \|_{1})}$
\end{enumerate}
\end{prop}
\begin{beweis}
	todo	
\end{beweis}

\begin{vereinbarung}
Sei $\MdF = \{ (x_{n}) \in \MdK^{N}: x_{i} = 0 \text{ bis auf endlich viele } n \in N  \} $ der Folgenraum \\
und $e_{j} = (0, \dotsc, 0, 1, 0, \dotsc, 0) $ der Einheitsvektor, wobei die $1$ an j-ter Stelle steht.
\end{vereinbarung}

\begin{beispiel}
	\begin{itemize}
		\item $l^{p} = \{ x = (x_{n}) \in \MdK^n: \|x\|_{p} = \left( \sum_{n = 1}^{\infty} | x_{n} |^{p}\right)^{\frac{1}{p}} < \infty \}$
		\item $l^{\infty} = \{ x = (x_{n}) \in \MdK^n: \|x\|_{\infty} = \sup_{n \in \MdN} |x_{n}| < \infty \}$
		\item $c_{0} = \{ x = (x_{n}) \in l^{\infty}: \lim_{n \rightarrow \infty} |x_{n}| = 0 \}$
	\end{itemize}
Gültigkeit der Dreiecksungleichung beweist man ähnlich wie bei $(\MdK^{n}, \| \cdot \|_{p})$.
\end{beispiel}

\begin{lemma}
Minkowskii-Ungleichung: $\left( \sum_{i=1}^{\infty} |x_{i} + y_{i}|^p\right)^{\frac{1}{p}} \leq\left( \sum_{i=1}^{\infty} |x_{i}|^p\right)^{\frac{1}{p}} \left( \sum_{i=1}^{\infty} |y_{i}|^p\right)^{\frac{1}{p}} $	\\
Hölder-Ungleichung: mit $\frac{1}{p} + \frac{1}{p'} = 1 \text{ gilt } \sum_{i=1}^{\infty} |x_{i}| |y_{i}| \leq \left( \sum_{i=1}^{\infty} |x_{i}|^{p} \right)^{\frac{1}{p}} \left( \sum_{i=1}^{\infty} |y_{i}|^{p'} \right)^{\frac{1}{p'}} $	\\	
\end{lemma}

\begin{bemerkung}
Im unendlich dimensionalen Fall sind die Normen $\| \cdot \|_{p}$ auf $\MdF$ nicht äquivalent.	
\end{bemerkung}
\begin{beweis}
todo
\end{beweis}


\begin{beispiel}
	\begin{enumerate}[label=\alph*\upshape)]
		\item Raum der stetigen Funktionen
		\item Raum der differentierbaren Funktionen \\
		$D^{\alpha}f(x) = \frac{ \delta^{ | \alpha | } }{ \delta x_{1}^{ \alpha_{1} } \dotsc \delta x_{n}^{ \alpha_{n} } } f(x), \text{ wobei } | \alpha | = \alpha_{1} + \dotsc + \alpha_{n} $ \\ 
	\end{enumerate}
\end{beispiel}

\begin{definition}
$C_{b}^{m}(\Omega) = \{ f: \Omega \rightarrow \MdR | D^{\alpha}f \text{ sind stetig in } \Omega, \text{ beschränkt auf } \Omega \text{ für alle } \alpha \in \MdN^{n}, |\alpha| \leq n \}$ \\
mit der Norm: $\| f \|_{C_{b}^{m}} = \sum_{|\alpha| \leq m} \| D^{\alpha}f \|_{\infty}$
\end{definition}

\begin{bemerkung}
Auf $C_{b}^{m} [0, 1]$ ist eine äquivalente Norm zu  $\| f \|_{C_{b}^{m}}$ gegeben durch
\begin{align*}
	\| f \|_{0} & = \sum_{i = 0}^{m - 1} |f^{(i)}(0)| + \| f^{(m)} \|_{\infty} \\
	\text{Denn } f^{(i)}(t) = f^{(i)}(0) + \int_{0}^{t} & f^{(i + 1)}(s) ds \text{ und damit } \| f^{(i)}\|_{0} \leq | f^{(i)}(0) | + \| f^{(i + 1)}\|_{\infty}	
\end{align*}
\end{bemerkung}
\begin{beispiel}
\begin{itemize}
	\item höldersche Funktionen
	\item $\Omega \subset \MdR^{n} offen$
\end{itemize}	
\end{beispiel}

\newpage
\section{Beschränkte und lineare Operatoren}

\begin{definition}
	Eine Teilmenge V eines normieren Raums $(X, \| \cdot \|)$ heißt beschränkt, falls 
	\[ c := \sup_{x \in V} \| x \| < \infty \]
	d.h. $V \subset c U_{(X, \| \cdot \| )}$
\end{definition}}

\begin{bemerkung}
Eine konvergente Folge $(x_{n})	\in X, x_{n} \rightarrow X$ ist beschränkt, denn $x_{m} \in \{ y: \| x - y \| \leq 1 \}$ für fast alle $m$.
\end{bemerkung}

\begin{satz}
	Seien $X$, $Y$ normierte Räume. Füreinen linearen Operator $S: X \rightarrow Y$ sind äquivalent:
	\begin{itemize}
		\item $T$ stetig, d.h. $x_{n} \rightarrow x$ impliziert $Tx_{n} \rightarrow Tx$
	\end{itemize}	
\end{satz}










