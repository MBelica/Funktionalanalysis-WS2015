%!TEX root = FunkAnalysis.tex

\chapter{Lineare Operatoren auf Banachräumen}
\section{Einführung}
\subsection{Räume}

Sei $X$ ein Vektorraum, $dim X < \infty $ und sei $x = (x_{1},\dotsc,x_{n})^{T} \in \MdR^{n}$
\begin{align*}
	& \|x\|_{2} := \left(\sum_{k = 1}^{n} \|x_{i}^{2}\| \right)^{\frac{1}{2}}\\
	& \|x\|_{\infty} := \smash{\displaystyle\max_{i = 1}^{n}}  \|x_{i}\|		
\end{align*}

Diese Normen sind äquivalent, denn:
$\| x \|_{\infty} \leq \| x \|_{2} \leq n^{\frac{1}{2}} \| x \|_{\infty}$ \newline

\begin{satz*}[Heine-Borel] \label{s:1-heineborell}  \todo{Nicht Heine-Borel!?}
$A \subset \MdR$ beschränkt. Dann hat jede Folge $(x_{n})_{n \in \MdN} \subset A$ eine konvergente Teilfolge.
\end{satz*}

\begin{beispiel}
$X = C[0, 1] = \{ f : [0, 1] \rightarrow \MdR: \text{ stetig auf } [0, 1] \}$
\begin{align*}
\begin{rcases*}
\| f \|_{2} := \left( \int_{0}^{1} \| f(t) \|^{2} dt \right)^{\frac{1}{2}} \\
\| f \|_{\infty} :=  \smash{\displaystyle\max_{t \in [0, 1]}}  \| f(t) \|
\end{rcases*}\text{ dabei gilt } \todo{anders herum?} \| f \|_{2} \leq \| f \|_{\infty} \leq \dotsc 
\end{align*}
\newline
Aber:$\hspace{0.5cm} f_n(t) = §BILD§, \hspace{0.5cm} \| f_n \|_{\infty} = 1, \hspace{0.5cm} \| f_n \|_{2} \xrightarrow[n \rightarrow \infty]{} 0  $ \\
$ \| f_{n} - f_{m} \| = 1 \text{ für } n \neq m \Rightarrow $ \hyperref[s:1-heineborell]{ Satz von Heine-Borel} gilt im $\infty$-dimensionalen i.A. nicht!
\end{beispiel}

\subsection{Operatoren}

Sei $N = dim X, M = dim Y$ und seien $(e_n)$ bzw. $(f_n)$ Basen von $X$ bzw. $Y$.
Für $T : X \rightarrow Y$ gegeben durch: \\
\[ §Komm. Diagramm§ \] \\
Daraus folgt:
\begin{itemize}
	\item $T$ ist stetig
	\item X = Y \gdw T injektiv \gdw T surjektiv (Dimensionsformel) \\
	(Die Gleichung $Tx = y$ ist eindeutig lösbar $\gdw$ Gleichung hat für alle $y \in Y$ eine Lösung.)
	\item Falls A selbstadjungiert ist, d.h. $A = A^{*}$, gibt es eine Basis aus Eigenvektoren $(e_{n})$ von $A$, d.h. $ T( \sum_{n=1}^{N} \alpha_{n} e_{n} ) = \sum{n=1}^{N} \lambda_{n} \alpha_{n} e_{n}$, wobei $\lambda_{n}$ Eigenwerte sind $A =
			\begin{pmatrix}
				\diagentry{\lambda_{1}}\\
				&\diagentry{\xddots}\\
				&&\diagentry{\lambda_{n}}\\
			\end{pmatrix} $ %todo summenindexes überprüfen
\end{itemize}

\begin{beispiel}[1]
$X = C^{1}[0, 1] = \{ f : [0, 1] \rightarrow \MdR: \text{ stetig auf } [0, 1] \}$ 
$T f = f', T : X \rightarrow Y \text{ stetig.} $ \\
(Aber: $T: C[0, 1] \rightarrow C[0, 1]$, hier ist $T$ nicht definiert.) \\
$T$ ist nicht stetig bzgl. $\|.\|_{\infty}$-Norm, siehe:
\[ f_{n}(t) = \frac{1}{\sqrt{n}}e^{int}, \hspace{0.5cm}  \text{dann: } \|f_{n}\| \rightarrow 0 \text{ für } t \rightarrow \infty \]	
\[ T f_{n}(t) = i \sqrt{n} e^{int}, \hspace{0.5cm} \text{mit: } \| T f_{n} \|_{\infty} \rightarrow \infty, \text{ für } n \rightarrow \infty \]
\end{beispiel}

\begin{beispiel}[2]
$X = L_{2} = \{ (a_{n}: \left( \sum_{n \geq 1}^{\infty} \| a_{n} \| \right)^{\frac{1}{2}} < \infty \}$	\\
$T ( a_{1}, a_{2}, a_{3}, ...) = ( 0, a_{1}, a_{2}, a_{3}, ...)$ \\
$T$ ist injektiv, aber nicht surjektiv
\end{beispiel}

\subsection{Anwendungen}

\begin{enumerate}
	\item Fredholm'sche Integralglechungen
	\item Dirichletproblem
	\item Sturm-Liouville Problem
\end{enumerate}


\newpage
\section{Normierte Räume}

\begin{definition}
Sei $X$ ein Vektorraum über $\MdK \in\{ \MdR, \MdC \}$ \\
Eine Abbildung  $\| . \| : X \rightarrow \MdR_{+}$ heißt eine Norm, wenn
\begin{enumerate}
	\item $ \| x \| \geq 0, \| x \| = 0 \gdw x = 0 $
	\item $\ | \lambda x \| = | \lambda \| x \| $
	\item $ \| x + y \| \leq \| x \| + \| y \| $
\end{enumerate}	
\end{definition}

\begin{bemerkung} Falls $ \| . \| $ all die oben genannten Eigenschaften erfüllt außer $ \| x \| = 0 \Rightarrow x = 0 $, dann heißt $ \| . \| $ Halbnorm %TODO alle |.| mit Mittelpunkt ersetze
\end{bemerkung}

Die Menge $ U_{X} = \{ x \in X:  \|x \| \leq 1 \}$ heißt \textbf{Einheitskugel}.

Eine Folge $(x_{n})$ des normierten Raums $X$ \textbf{konvergiert} gegen ein $ x \in X $, falls  $\| x_{n} - x \| \xrightarrow[n \rightarrow \infty]{} 0. $


\begin{bemerkung}
Für zwei Elemente $x, y \in (X, \|.\|)$	... gilt die umgekehrte Dreiecksungleichung $( | \|x\| + \|y\| | \leq \| x + y \|)$
\end{bemerkung}

\begin{beispiel}
Sei $ X = \MdK^{n}, \hspace{0.5cm} X = ( X_{1}, \dotsc, X_{n}), \hspace{0.5cm} X_{i} = \MdK$ \\
\begin{align*}
	\| x \|_{p} & = \left( \sum_{j = 1}^{n} | x_{j}^{p} | \right)^{\frac{1}{p}},  1 \leq p < \infty (p = 2: \text{ Euklidische Norm}) \] \\
	\| x \|_{\infty} & = sup_{j = 1}^{n} |x_{j}|	
\end{align*}

Beh: $\| . \| $ ist Norm auf $\MdK^n$ für $1 \leq p \leq \infty$

$\| x + y \|_{\infty} = sup_{j = 1}^{k} |x_{j} + y_{j}| \leq \|x\|_{\infty} + \|y\|_{\infty} $
Für $p \in (1, \infty), p \neq 2:$ siehe Übungsaufgabe (Fall $p = 2$ läuft über Cauchy-Schwarz)

Beachte: \|x\|_{\infty} \leq  \|x\|_{\infty} \leq n \|x\|_{\infty}
\end{beispiel}





