%!TEX root = Funktionalanalysis - Vorlesung.tex

\chapter{Metrische R{\"a}ume}

\begin{definition} \index{Metrik}
	Sei $M$ eine nichtleere Menge. Eine Abbildung $d: M \times M \rightarrow \MdR$ hei{\ss}t \textbf{Metrik} auf $M$, falls $\forall x, y, z \in M:$
		\begin{description}
			\item[$\hspace{0.5cm} (M1) \hspace{0.1cm} $] $d(x, y) \geq 0, \hspace{0.25cm} d(x, y) = 0 \gdw x = y $  (positive Definitheit)
			\item[$\hspace{0.5cm} (M2) \hspace{0.1cm} $] $d(x, y) = d(y, x)$  (Symmetrie)
			\item[$\hspace{0.5cm} (M3) \hspace{0.1cm} $] $d(x, z) \leq d(x, y) + d(y, z)$  (Dreiecksungleichung)
		\end{description}
	Das Tupel $(M, d)$ nennen wir dann einen metrischen Raum. 
\end{definition}

\begin{lemma}
	Eine Folge $(x_{n})_{n \geq 1} \subset M$ konvergiert gegen $x \in M$, falls
	\[ d(x_{n}, x) \rightarrow 0 \hspace{0.5cm} \text{für } n \rightarrow \infty \]
\end{lemma}
 
Notation: $x = \lim_{n \rightarrow \infty} x_{n}$ (in $M$)

\begin{bemerkung}
Der Grenzwert einer konvergenten Folge ist stets eindeutig, denn: \\
Sei $(x_{n})_{n \geq  1} \subset M$ mit $\lim_{n \rightarrow \infty} x_{n} = x \in M$ und $\lim_{n \rightarrow \infty} x_{n} = y \in M$, dann folgt: 
	\begin{align*}
		d(x, y) & \leq d(x, x_{n}) + d(x_{n}, y) \\
				& \rightarrow 0 \text{ für } n \rightarrow \infty
	\end{align*}
	d.h. $d(x, y) = 0 \Rightarrow x = y$
\end{bemerkung}

\begin{beispiel}
	\begin{enumerate}[label=\alph*\upshape)]
		\item Sei $X$ ein normierter Vektorraum und $M \subset X$ (nichtleere) Teilmenge. \\
		Dann definieren wir $d(x, y) := \| x - y\|, \hspace{0.25cm} x, y \in M$ eine Metrik auf $M$ \\
		Ein Unterschied hier: Eine Norm setzt eine lineare Struktur auf $X$ voraus, eine Metrik macht auch Sinn auf nicht-linearen Teilmengen.
		\index{diskrete Metrix} \label{bsp:1-diskreteMetrik}
		\item Sei $M$ eine nichtleere Menge, dann definieren wir die \textbf{diskrete Metrik} auf $M$ durch
			\[ d(x, y) := \begin{cases}1 &, x \neq y \\ 0 &, x = y\end{cases} \]
			Dann ist $(M, d)$ ein metrischer Raum und es gilt: 
				\[ x_{n} \rightarrow x \text{ in } M \gdw \exists N \in \MdN \text{ mit } x_{n} = x \hspace{0.25cm} \forall n \geq N \]
	\end{enumerate}	
\end{beispiel}

\begin{beispiel}
	\begin{enumerate}[label=\alph*\upshape)]	
		\item Sei $X$ ein Vektorraum und $p_{j}$ für $j \in \MdN$ Halbnormen auf $X$ mit der Eigenschaft, dass für jedes $x \in X \ {0}$ ein $K \in \MdN$ exisitert mit $p_{K} > 0$. Dann definiert
			\[ p := \sum_{j \geq 1} 2^{-j} \frac{p_{j}(x - y)}{1 + p_{j}(x -y)}, \hspace{0.5cm} x, y \in X \]
			eine Metrik auf $X$ mit
			\[ d(x_{n}, x) \rightarrow 0 \gdw p_{j}(x_{n}, x) \rightarrow 0 \hspace{0.25cm} (n \rightarrow  \infty) \hspace{0.25cm} \forall j \in \MdN \]
			Beweis siehe Übung
		\item Für $X = \MdK^{\MdN} = \{ (x_{n})_{n \geq 1}: x_{n} \in \MdK \}$ und $p_{j}(x) := |x_{j}|, j \in \MdN$ definiert also
			\[ d(x, y) = \sum_{j = 1}^{\infty} 2^{-j} \frac{|x_{j} - y_{j}|}{1 + |x_{j} - y_{j}|} \text{ gerade die komponentenweise Konvergenz auf } X \]
		\item In $l^{\infty}$ entspricht die Konvergenz bezüglich $\| \cdot \|_{l^{\infty}}$ gerade der gleichmä{\ss}igen Konvergenz der Folge $x_{n} := (x_{n_{i}})_(i \in \MdN)$ gegen $x := (x_{i})_(i \in \MdN)$
			\[ \| x_{n} - x \|_{l^{\infty}} = \sup | x_{n_{i}} - x_{i} | \rightarrow 0 \hspace{0.5cm} (n \rightarrow \infty) \]
			d.h. in $C[a, b]$ entspricht die Konvergenz bezüglich $\| \cdot \|_{\infty}$ ebenfalls die gleichmä{\ss}ige Konvergenz von Funktionen
			\begin{align*}
				f_{n} \rightarrow f \text{ in } [a, b] & \gdw \| f_{n} - f \|_{\infty} = \sup_{t \in [a, b]} |f_{n}(t) - f(t) | \rightarrow 0 \\
				& \gdw f_{n} \rightarrow f \text{ gleichmä{\ss}ig.}
			\end{align*}
	\end{enumerate}
\end{beispiel}

\begin{definition} \index{abgeschlossen} \index{offen}
	Sei $(M, d)$ ein metrischer Raum
	\begin{enumerate}[label=\alph*\upshape)]
		\item Eine Teilmenge $A \subset M$ hei{\ss}t \textbf{abgeschlossen} (in $M$), falls für alle in $M$ konvergenten Folgen $(x_{n})_{n \geq 1} \subset A$ der Grenzwert von $(x_{n})$ in $A$ liegt
		\item Eine Teilmenge $U \subset M$ hei{\ss}t \textbf{offen} (in $M$), falls zu jedem $x \in U$ ein $\epsilon > 0$ exisitert, sodass
			\[ \{ y \in M: d(x, y) < \epsilon \} \subset U \]
	\end{enumerate}
\end{definition}

\begin{bemerkung}
	\begin{enumerate}[label=\alph*\upshape)] 
		\item Wir benutzen die Bezeichnungen
			\begin{align*}
				K(x, r) & := \{ y \in M: d(x, y) < r \} \hspace{0.25cm} \text{\textbf{ offene Kugel}} \\
				\bar K(x, r) & := \{ y \in M: d(x, y) \leq r \} \hspace{0.25cm}  \text{\textbf{ abgeschlossene Kugel}}
			\end{align*}
			für $x \in M, r > 0$. Man sieht leicht, dass $K(x, r)$ offen und $\bar K(x, r)$ abgeschlossen ist.
		\begin{beweis}
			\begin{enumerate}
			\item Sei $y \in K(x, r)$ und wähle $\rho := r - d(x, y) > 0$ \\
				Wir zeigen: $K(y, \rho) \subset K(x, r)$ (Dann ist $K(x, r)$ offen). Sei dazu $z \in K(y, \rho)$. Dann folgt 
				\begin{align*}
					d(x, y) \leq d(x, z) + d(z,y) & \leq r - \rho + d(z, y) \\
												  & \leq r - \rho + \rho = r \\ \\
								\Rightarrow z \in K(x, r)
				\end{align*} 
				Da $z$ beliebig war, folgt die Behauptung.
			\item Sei $(y_{n})_{n \geq 1} \subset \bar K(x, r)$ eine beliebige Folge mit $ \lim_{n \rightarrow \infty} y_{n} = y \in M$. Wir müssen zeigen, dass $y \in \bar K(x, r)$ (Dann ist $\bar K(x, r)$ abgeschlossen). \\
				\begin{align*}
					d(x, y) & \leq d(x, y_{n}) + d(y_{n}, y) \\
							& \leq r + d(y_{n}, y) \rightarrow r \\ \\
						& \Rightarrow d(x, y) \leq r, \text{ d.h.} y \in \bar K(x, r).
				\end{align*} 
			\end{enumerate}	
		\end{beweis}
		\item $\emptyset, M$ sind sowohl offen, als auch abgeschlossen (in $M$)
		\item Bezüglich der diskreten Metrik $d$ aus Beispiel \eqref{bsp:1-diskreteMetrik} ist $\{x\} \subset M$ offen für jedes $x \in M$, da
			\[ K(x, r) = \{ x \} \subset \{ x \} \text{ für } r \in (0, 1] \]
	\end{enumerate}	
\end{bemerkung}

Wir fassen als nächstes die grundlegenden Eigenschaften offener und abgeschlossener Mengen zusammen.

\begin{prop}
	Sei $(M, d)$ ein metrischer Raum und $I$ eine beliebige Indexmenge
	\begin{enumerate}[label=\alph*\upshape)]
		\item $A \subset M$ ist abgeschlossen in $M$ genau dann, wenn $U = M \ A$ offen ist
		\item Für eine beliebige Familie von abgeschlossenen Mengen $(A_{i})_{i \in I}$ sind 
			\[ A := \bigcap_{i \in I} A_{i} \hspace{0.5cm} \text{ und } \hspace{0.5cm} A_{i_{1}} \cup \dotsc \cup A_{i_{N}} \hspace{0.25cm} (i_{1}, \dotsc, i_{N} \in I) \]
			abgeschlossen in $M$.
		\item Für eine beliebige Familie offenere Mengen $(U_{i})_{i \in I}$ sind
			\[ U := \bigcup_{i \in I} U_{i} \hspace{0.5cm} \text{ und } \hspace{0.5cm} U_{i_{1}} \cap \dotsc \cap U_{i_{N}} \hspace{0.25cm} (i_{1}, \dotsc, i_{N} \in I) \] 
			offen in $M$.
	\end{enumerate}
	\begin{beweis}
		\begin{enumerate}[label=\alph*\upshape)]
			\item Todo: missing in my notes if you have it just send me an email %todo
			\item folgt aus a) \& c), da
				\[ M \ \bigcap_{i \in I} A_{i} = \bigcup_{i \in I} M \ A_{i} \]
			\item Sei $x \in U$. Dann exisitert ein $U_{i_{0}}$ mit $x \in U_{i_{0}}$
			\[ \Rightarrow \exists r > 0: K(x, r) \subset U_{i_{0}} \subset U, \text{ d.h. } U \text{ ist offen.}  \]
			Sei $x \in U_{i_{1}} \cap \dotsc \cap U_{i_{N}}.$ Dann existieren $r_{1}, \dotsc, r_{N} > 0$ mit
			\[ K(x, r_{n} \subset U_{i_{n}} \hspace{0.5cm} n = 1, \dotsc, N \]
			Setze $r := \min \{ r_{1}, \dotsc, r_{n} \} > 0$. Dann ist $K(x, r) \subset U_{i_{n}} \hspace{0.25cm} \forall n \in \{ 1, \dotsc, N \}$
			\[ \Rightarrow K(x, r) \subset U_{i_{1}} \cap \dotsc \cap U_{i_{N}} \text{ ist offen.} \]
		\end{enumerate}	
	\end{beweis}
\end{prop}

\begin{definition}
	Sei $(M, d)$ ein metrischer Raum und $V \subset M$. Dann heißt 
	\begin{enumerate}[label=\alph*\upshape)] \index{Abschluss} \index{Innere} \index{Rand}
		\item $\bar V := \bigcap \{ A \subset M: A$ ist abgeschlossen mit $V \subset A \} $ der \textbf{Abschluss} von $V$.
		\item $\mathring V := \bigcup \{ U \subset M: U$ ist offen mit $U \subset V \}$ das \textbf{Innere} von $V$. 
		\item $ \partial V := \bar V \ \mathring V$ der \textbf{Rand} von $V$.
	\end{enumerate}
\end{definition}

Hierfür gelten die folgenden Eigenschaften:

\begin{prop}
	Sei $(M, d)$ ein metrischer Raum und $V \subset M$
	\begin{enumerate}[label=\alph*\upshape)]
		\item
			\begin{enumerate}
				\item $\bar V$ ist die kleinste abgeschlossene Menge, die $V$ enthält
				\item $V$ ist abgeschlossen $\gdw V = \bar V$
				\item $\bar V = \{ x \in M: \exists (x_{n}) \subset V$ mit $\lim_{n \rightarrow \infty} x_{n} = x \} =: \tilde V$
			\end{enumerate} 
		\item 
			\begin{enumerate}
				\item $\mathring V$ ist die grö{\ss}te offene Teilmenge von $V$.
				\item $V$ ist offen $\gdw V = \mathring V$
				\item $\mathring V = \{ x \in M: \exists \epsilon > 0$ mit $K(x, \epsilon) \subset V \} =: \hat V$
			\end{enumerate} 
		\item
			\begin{enumerate}
				\item $\partial V$ ist abgeschlossen
				\item $\partial V = \{ x \in M: \exists (x_{n}) \subset V, (y_{n}) \subset M \ V$ mit $ \lim_{n \rightarrow \infty} x_{n} = \lim_{n \rightarrow \infty} y_{n} = x \}$
			\end{enumerate} 
	\end{enumerate}	
	\begin{beweis}
		\begin{enumerate}[label=\alph*\upshape)]
			\item
				\begin{enumerate}
					\item Nach Definition gilt 
					\item folgt aus (i)
					\item %todo
				\end{enumerate} 
			\item 
				\begin{enumerate}
					\item zeigt man wie a) (i)
					\item folgt aus (i)
					\item %todo
				\end{enumerate} 
			\item
				\begin{enumerate}
					\item %todo
					\item %todo
				\end{enumerate} 
		\end{enumerate}	
	\end{beweis}
\end{prop}

\begin{definition}
	Sei $(M, d)$ ein metrischer Raum
	\begin{enumerate}[label=\alph*\upshape)] \index{dicht} \index{seperabel}
		\item Eine Menge $V \subset M$ hei{\ss}t \textbf{dicht} in M, falls $\bar V = M$, d.h. jeder Punkt in $M$ ist Grenzwert einer Folge aus $V$.
		\item $M$ hei{\ss}t \textbf{seperabel}, falls es eine abzählbare Teilmenge $V \subset M$ gibt, die dicht in $M$ liegt.
	\end{enumerate}
\end{definition}

\begin{bemerkung}
	\begin{enumerate}[label=\alph*\upshape)] \index{relative Offenheit} \index{relative Abgeschlossenheit} %todo verweise
		\item All die Begriffe und Bezeichnungen aus TODO: Verweise werden wir auch in normieren Räumen benutzen bzgl. der kanonischen Metrik $d(x, y) = \| x - y\|$
		\item Sei $(M, d)$ ein metrischer Rauam, $U \subset M$. Dann ist auch $(U, d)$ ein metrischer Raum. Für $V \subset U$ muss man dann aber unterscheiden bzgl. Abgeschlossenheit (bzw. Offenheit) von $V$ in $U$ oder in $M$. Man sagt dann, dass $V$ \textbf{relativ offen} bzw. \textbf{relativ abeschlossen} in $U$ ist.
	\end{enumerate}	
\end{bemerkung}

\begin{beispiel}
	\begin{enumerate}[label=\alph*\upshape)]
		\item Sei $X$ ein normierter Vektorraum, $x \in X, r > 0$. Dann gilt
			\begin{align*}
				\bar K(x, r) & = \overline{K(x, r)} \\
				\overset{\circ}{\bar K(x, r)} & = K(x, r) \\
				\partial \bar K(x, r) & = \partial K(x, r) (= \{ y \in X: \| x - y \| = r \} )		
			\end{align*}
			\begin{beweis} %todo verweise (2x)
				Da $ \bar K(x, r)$ abgeschlossen ist mit $K(x, r) \subset \bar K(x, r)$ folgt aus Proposition 4.8  a) (i) $\overline{K(x, r)} \subset \bar K(x, r)$. \\
				Sei umgekehrt $y \in \bar K(x, r)$ und $y_{n} = y - \frac{1}{n} (y - x), n \in \MdN$. Dann ist $y_{n} \in K(x, r)$ mit $\lim_{n \rightarrow \infty} y_{n} = y$, d.h. $y \in \overline{K(x, r)}$ nach Proposition 4.8 a) (iii).
			\end{beweis} %todo: verweis
		\item Da $K(x, r)$ offen ist mit $K(x, r) \subset \bar K(x, r)$ folgt mit Proposition 4.9 b) (ii): 
				\[ K(x, r) \subset \overset{\circ}{\bar K(x, r)} \]
			Sei umgekehrt 
		\item 3 % todo
		\item 4 % todo
	\end{enumerate}
\end{beispiel}

\begin{definition} \index{stetig}
	Seien $(M, d_{M}), (N, d_{N})$ metrische Räume. \\
	Eine Abbildung $f: M \rightarrow N$ hei{\ss}t \textbf{stetig in $x_{0} \in M$}, falls für alle $(x_{n}) \subset M$ gilt
	\[ x_{n} \rightarrow x_{0} \text{ in } M \Rightarrow f(x_{n}) \rightarrow f(x_{0}) \text{ in } N \]
	\[ (d(x_{n}, x_{0}) \rightarrow 0 \hspace{0.25cm} (n \rightarrow \infty) \hspace{0.5cm} \Rightarrow \hspace{0.5cm} d_{N}(f(x_{n}), f(x_{0})) \rightarrow \hspace{0.25cm} (n \rightarrow \infty)) \]
	Die Abbildung $f$ hei{\ss}t \textbf{stetig auf $M$}, falls $f$ in jedem Punkt von $M$ stetig ist.
\end{definition}

Hierfür gelten folgende Eigenschaften:

\begin{prop}
	Sei $(K,d_{K}), (M, d_{M})$ und $(N, d_{N})$ metrische Räume und $f: M \rightarrow N, g: K \rightarrow M$. Dann gilt:
	\begin{enumerate}[label=\alph*\upshape)]
		\item Ist $g$ stetig in $x_{0}$, $f$ stetig in $g(x_{0})$, dann ist auch 
			\[ f \circ g: K \rightarrow N \text{ stetig in } x_{0} \]
		\item $f$ ist stetig in $x_{0} \in M$ genau dann, wenn 
			\[ \forall \epsilon > 0 \hspace{0.15cm} \exists \delta > 0 \hspace{0.15cm} \forall x \in M \text{ mit } d_{M}(x, x_{0}) < \delta \text{ gilt } d_{N}(f(x), f(x_{0})) < \epsilon \]
		\item Die folgenden Aussagen sind äquivalent:
			\begin{enumerate}
				\item $f$ ist stetig auf $M$
				\item Ist $U \subset N$ offen, so ist auch $f^{-1}(U)$ offen in $M$
				\item Ist $A \subset N$ abgeschlossen, so ist auch $f^{-1}(A)$ abgeschlossen in $M$.
			\end{enumerate}
	\end{enumerate}	
	\begin{beweis} % todo		

	\end{beweis}
\end{prop}

\begin{beispiel}
	\begin{enumerate}[label=\alph*\upshape)]
		\item Sind $(M_{1}, d_{1}$ und $(M_{2}, d_{2})$ metrische Räume, so definieren wir
			\[ d(x, y) := d_{1}(x_{1}, y_{1}) + d_{2}(x_{2}, y_{2}) \]
			für $x = (x_{1}, x_{2}), y = (y_{1}, y_{2}) \in M_{1} \times M_{2}$ mit
			\[ d(x_{n}, x) \rightarrow 0 \gdw d_{1}(x_{n,1}, x_{1}) \rightarrow 0, d_{2}(x_{n,2}, x_{2}) \rightarrow 0 \]
			In diesem Sinne ist jede Metrik $d: M \times M \rightarrow \MdR$ stetig. Denn: \\
			Sei $(x_{n}, y_{n}) \rightarrow (x, y)$ in $M \times M$, d.h.
			\[ d(x_{n}, x) \rightarrow 0 \text{ und } d(y_{n}, y) \rightarrow 0 \hspace{0.5cm} (n \rightarrow \infty) \]
			\begin{align*}
				 d(x_{n}, y_{n}) - d(x, y) & \leq d(x_{n}, x) + d(x, y_{n}) - d(x, y) \\
										  & \leq d(x_{n}, x) + d(x, y) + d(y, y_{n}) - d(x, y) 		
			\end{align*}
			\[ d(x, y)- d(x_{n}, y_{n}) \leq \dotsc \leq d(x, x_{n}) + d(y_{n}, y) \]
			\[ \Rightarrow | d(x, y) - d(x_{n}, y_{n}) | \leq d(x, x_{n}) + d(y_{n}, y) \rightarrow 0 \hspace{0.5cm} (n \rightarrow \infty) \]
		\item Sei $X$ ein normierter Vektorraum und
			\begin{align*}
				A: \MdK \times X \rightarrow X, & \hspace{0.5cm} A( \alpha, x) = \alpha x \\
				S: X \times X \rightarrow X, & \hspace{0.5cm} S(x, y) = x + y
			\end{align*}
			Dann sind $A$ und $S$ stetig.
		\item Sei $X = C[0, 1], t_{0} \in [0, 1], \psi : X \rightarrow \MdK, \psi(f) = f(t_{0})$ \\	% todo verweis 
		Nach Beispiel 3.15 ist $\psi$ stetig. D.h. ist $A \subset \MdK$ offen (abgeschlossen), so ist $\psi(A)$ offen (abgeschlossen) nach Prop 4.13. %verweis y
	\end{enumerate}	
\end{beispiel}

































