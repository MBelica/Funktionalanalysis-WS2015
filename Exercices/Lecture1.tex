%!TEX root = Funktionalanalysis - Übung.tex

\chapter{Übung}

\section{Wiederholung}

\subsection{$\mathcal{L}^{p}$-Räume}
\begin{enumerate}
	\item \textbf{$\sigma$-Algebren} \\
	Sei $X$ eine nichtleeren Menge. Dann nennen wir $\mathcal{A} \subset \mathcal{P}(X)$ eine $\sigma$-Algebra, falls
		\begin{itemize}
			\item $X \in \mathcal{A}$
			\item $\forall A \in \mathcal{A}: \hspace{0.25cm} A^{c} = X\verb|\|A \in \mathcal{A}$
			\item $\forall (A_{n})_{n \geq 1} \subset \mathcal{A}: \hspace{0.25cm} \bigcup_{n \geq 1} A_{n} \in \mathcal{A}$
		\end{itemize}
	\begin{bemerkung}
		Aus der Definition der $\sigma$-Algebren folgt dierekt für $(A_{n})_{n \geq 1} \subset \mathcal{A}$:
		\begin{itemize}
			\item $\emptyset \in \mathcal{A}$
			\item ...
			\item $\bigcap_{n \geq 1} A_{n} \in \mathcal{A}$
		\end{itemize}
	\end{bemerkung}

\end{enumerate}