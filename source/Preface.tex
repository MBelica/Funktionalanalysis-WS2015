%!TEX root = Funktionalanalysis - Vorlesung.tex

%\chapter*{Vorwort}
%Dieses Skript wurde im Wintersemester 2015/2016
%von Martin Belica geschrieben. Es beinhaltet die Mitschriften aus
%der Vorlesung von Prof.~Dr.~Weis sowie die Mitschriften einiger
%Übungen.

\thispagestyle{empty}

\section*{Einleitung}

Die Funktionalanalysis liefert den begrifflichen Rahmen sowie allgemeine Methoden, die in weiten Teilen der modernen Analysis verwendet werden. Zum Beispiel ist es möglich Integral- und Differentialgleichungen als lineare Gleichungen in einem geeigneten unendlichdimensionalen Vektorraum (wie z.B. einem Raum stetiger oder integrierbarer Funktionen) aufzufassen. Will man nun auf diese unendlichdimensionalen Gleichungen Ideen der linearen Algebra anwenden, so treten Konvergenz- und Kompaktheitsprobleme auf, die wir in dieser Vorlesung behandeln wollen. Zu den Themen gehören:

  \begin{itemize}
     \item Beschränkte und abgeschlossene Operatoren auf normierten Räumen
     \item Stetigkeit und Kompaktheit auf metrischen Räumen
     \item Geometrie und Operatorentheorie in Hilberträumen
     \item Der Satz von Hahn-Banach und Dualität von Banachräumen    
  \end{itemize}
  
Die allgemeinen Aussagen werden durch konkrete Beispiele von Räumen und Operatoren der Analysis illustriert.

\section*{Erforderliche Vorkenntnisse}
Analysis I-III, Lineare Algebra I-II