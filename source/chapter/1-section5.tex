%!TEX root = Funktionalanalysis - Vorlesung.tex

\section{Vollst{\"a}ndigkeit}

\begin{definition}
	Sei $(M, d)$ ein metrischer Raum.
	\begin{enumerate}[label=\alph*\upshape)]
		\item $x_{n} \in M$ hei{\ss}t \begriff{Cauchy-Folge}, falls es zu jedem $\epsilon > 0$ ein $n_{0} \in \MdN$ gibt, sodass $\forall m, n \geq n_{0}$ gilt:
			\[ d(x_{n}, x_{m}) \leq \epsilon \]
		\item $(M, d)$ hei{\ss}t \begriff{vollständig}, falls jede Cauchy-Folge $(x_{n}) \subset M$ einen Grenzwert \uline{in M} hat:
			\[ \lim_{n \rightarrow \infty} x_{n} = x \quad x \in M \]
		\item Ein normierter Raum $(X, \| \cdot \|)$, der vollständig ist bezüglich $d(x, y) = \| x - y \|$ heißt \begriff{Banachraum}.
	\end{enumerate}
\end{definition}

\begin{bemerkung}
	\begin{enumerate}[label=\alph*\upshape)]
		\item Jede konvergente Folge in $(M, d)$ ist eine Cauchy-Folge:
			\[ \text{Sei } \lim_{n \rightarrow \infty} x_{n} = x: \quad d(x_{n}, x_{m}) \leq d(x_{n}, x) + d(x, x_{m}) \xrightarrow[n, m \rightarrow \infty]{} 0 \]
		\item \uline{Aber:} nicht jede Cauchy-Folge eines normierten Raums $X$ konvergiert in $X$
			\begin{beispiel*}
				$X = C[0, 2], \quad \| f \|_{1} = \int_{0}^{2} | f(t) | dt, \quad
				f_{n}(x) = \begin{cases}x^{n} & \text{ für } x \in [0, 1] \\ 1 & \text{ für } x \in [1, 2]\end{cases}$	
				\[ f_{n}(x) \rightarrow f(x) = \begin{cases} 0 & \text{ für } x \in [0, 1) \\ 1 & \text{ für } x \in [1, 2] \end{cases} \quad \text{ für feste } x \in [0, 2] \]
				Nach dem Satz von Lebesgue folgt $\| f - f_{n} \|_{1} \rightarrow 0$ für $n \rightarrow \infty$, aber $f \notin C[0, 2]$ \\
				Demnach ist $f_{n}$ zwar eine Cauchy-Folge, aber $f_{n}$ konvergiert nicht gegen $f$ in $X$ bezüglich der $\| \cdot \|_{1}$-Norm.
			\end{beispiel*}
	\end{enumerate}	
\end{bemerkung}

\begin{prop}
	Sei $X$ ein metrischer Raum, $Y$ ei Banachraum.
	\[ C(x, Y) = \{ f : X \rightarrow Y: f \text{ stetig} \}, \quad \| f \|_{\infty} = \sup_{x \in X} \| f(x) \|_{Y} \]
	Dann ist $C(X, Y)$ ein (linearer) Banachraum.
\end{prop}

\begin{beispiel*}
	$\Omega \subseteq \MdR^{n}, \quad C(\Omega, \MdR)$
	\begin{beweis}
		Sei $(f_{n})$ eine Cauchy-Folge in $C(X, Y)$. \\
		Für alle $x \in X$:
		\[ \| f_{n}(x) - f_{m}(x) \|_{Y} \leq \| f_{n} - f_{m} \|_{\infty} \xrightarrow[n, m \rightarrow \infty]{} 0 \]
		Für alle $x \in X$: $(f(x))_{n \in \MdN}$ ist eine Cauchy-Folge in $Y$.
		da $Y$ vollständig ist, existiert $f(x) := \lim_{n \rightarrow \infty} f_{n}(x)$ in Y. \\ \\
		z.z. $f \in C(X, Y), \quad \| f - f_{n} \|_{\infty} \rightarrow 0$. \\
		Zu $\epsilon > 0$ gibt es ein $n_{0}$, sodass für alle $x \in X$:
		\[ \| f_{n}(x) - f_{m}(x) \|_{Y} \leq \| f_{n} - f_{m} \|_{\infty} \leq \epsilon \]
		Für jedes $x \in X$ fest, folgt für $m \rightarrow \infty$:
		 \[ \| f_{n}(x) - f(x) \| \leq \epsilon \quad \text{ für } n \geq n_{0} \]
		 \[ \Rightarrow \| f_{n} - f \|_{\infty} \leq \epsilon \text{ für } n \geq n_{0} \quad  \text{ nehme das Supremum über } x \in X \]
		 $f \in C(X, Y)$, da der gleichmä{\ss}ige Limes stetiger Funktionen stetig ist.
	\end{beweis}
\end{beispiel*}

\begin{beispiel}
	Sei $\Omega \subseteq \MdR^{n}$ offen und beschränkt. $C^{m}(\bar \Omega)$ ist vollständig bezüglich der Supremums-Norm.	
	\begin{beweis}
		Für $C^{1}(\bar \Omega)$ gilt $\| f \|_{C^{1}} = \| f \|_{\infty} + \sum_{i = 1}^{n}  \| \frac{\partial}{\partial x_{i}} f \|_{\infty}$. \\
		 Sei $(f_{j}) \subset F$ in $C^{1}(\bar \Omega)$.
		 \[ \Rightarrow (f_{j})_{j \in \MdN}, \quad (\frac{\partial}{\partial x_{j}} f_{j})_{j \in \MdN}, \quad i = 1, \dotsc, n \text{ Cauchy-Folgen in } C(\bar \Omega) \]
		 Da $C(\bar \Omega)$ vollständig ist, existieren für $i \in \{ 1, \dotsc, n \} $
		 	\begin{align*}
		 		f & = \lim_{j \rightarrow \infty} f_{j} \\
		 		g_{i} & = \lim_{j \rightarrow \infty} \frac{\partial}{\partial x_{i}} f_{j}
		 	\end{align*}
		 in $C(\bar \Omega)$. Setze $g = (g_{1}, \dotsc, g_{n})$ \\ \\
		 z. z. $f \in C^{1}(\bar \Omega)$ und $g = \nabla f$ \\
		 Beweis: zu $u \in \Omega$ und $v$ nahe bei $u$ wähle
		 \[ u_{t} = (1 - t) u + t v \]
		 \begin{align*}
		 	| f_{k}(v) - f_{k} (u) - \nabla f_{k}(u) (v - u) | & = | \int_{0}^{1} [ \nabla f_{k} (u_{t}) - \nabla f_{k} (u) ] (v - u) dt \\
		 		& \leq  \int_{0}^{1} | \nabla f_{k} (u_{t}) - \nabla f_{k} (u) | dt (v - u) \\
		 		& \leq \left[ \int_{0}^{1} | \nabla f_{k}(u_{t}) - g(u_{t}) | dt + \int_{0}^{1} | g(u_{t}) - g(u) | dt + \int_{0}^{1} | g(u) - \nabla f_{k}(u) | dt \right] \\
		 		&  \hspace{5cm} \cdot |v - u| \\
		 		& \leq \left[ z \| \nabla f_{k} - g \|_{\infty} + \sup_{0 \leq t \leq 1} | g(u_{t}) - g(u) | \right] |v - u|
 		 \end{align*}
 		 \begin{align*}
 		 	\text{für } k \rightarrow \infty: | f(v) - f(u) - g(u)(v - u) | & \leq \sup_{t \in [0, 1]} | g(u_{t}) - g(u) | |v - u| \\
 		 		& \rightarrow 0 \text{ für } v \rightarrow u \text{ (da g gleichmä{\ss}ig stetig)}
 		 \end{align*} 
	\end{beweis}
\end{beispiel}

\begin{bemerkung}
	\begin{enumerate}[label=\alph*\upshape)]
		\item $\| \cdot \|_{1}, \| \cdot \|_{2}$ seien äquivalente Normen auf $X$. Ist $X$ bezüglich $\| \cdot \|_{1}$, so auch bezüglich $\| \cdot \|_{2}$.
			\begin{beweis}
				Äquivalente Normen haben gleiche Cauchy-Folgen.
			\end{beweis}
			Bsp.: $C^{1}[0, 1]$, $\vertiii{f} = |f(0)| + \sup_{t \in [0, 1]} | f'(t) |$. Früher: $\vertiii{\cdot} \sim \| \cdot \|_{\infty} \Rightarrow \left( C[0, 1], \vertiii{\cdot} \right)$ ist vollständig.
		\item Abgeschlossene Teilmengen von 	Banachräumen sind vollständige metrische Räume bezüglich $d(x, y) = \| x - y\|$
			\begin{beweis}
				$(x_{n}) \subset M$, Cauchy-Folge in $X \Rightarrow \lim_{n \rightarrow \infty} x_{n} = x \in X \xRightarrow[M abg.]{} x \in M$ existiert.	
			\end{beweis}
			Bsp.: $X = C([0, 1], \MdC), M = \{ f \in X: |f ( t )| = 1 \}$ ist ein vollständiger metrischer Raum.
	\end{enumerate}
\end{bemerkung}



\begin{satz}
	Sei $X$ ein normiert Raum, $Y$ ein Banachraum.
	Dann ist $B(X, Y)$ mit der Operatornorm vollständig. \\
	Insbesondere: $X' = B(X, \MdK)$ ist immer  vollständig.
\end{satz}
\begin{beweis}
	Sei $(T_{n}) \subset B(X, Y)$ eine Cauchy-Folge bezüglich der Operatornorm. Sei $x \in X$
	\[ \| T_{n} x - T_{m} \|_{Y} \leq \| T_{n} - T_{m} \| \|x\|_{X} \]
	Also $(T_{n}x)_{n \in \MdN}$ eine Cauchy-Folge in $Y$ für alle $x \in X$. Definiere $T x : = \lim_{n \rightarrow \infty} T_{n} x$ \\
	z.z. $T \in B(X, Y),$ $s\| T_{n} - T \| \rightarrow 0$ \\
	\[ T_{n} (x + y)  = T_{n} x + T_{n} y \xrightarrow[n \rightarrow \infty]{} T(x + y) = Tx + Ty \]
	\begin{align*}
		\| Tx - T_{n}x \| & \overset{Norm \text{ } stetig}{=} \lim_{m \rightarrow \infty} \| T_{m} x - T_{n} x \| \\
		 & \leq \lim_{m \rightarrow \infty} \| T_{n} - T_{m} \| \| x \| \\
		 & \leq \epsilon \| x \| \text{ für n gro{\ss} genug.}
	\end{align*}
	Wobei $\| T - T_{n} \| \leq \epsilon$ für $n$ gro{\ss} genug.
	Also für $\| x \| \leq 1$:
	\[ \| T x \| \leq \| T_{n} x \| + \epsilon \leq \| T_{n} \| + \epsilon  \]
	\[ \Rightarrow \| T \| \leq \| T_{n} \| + \epsilon, \quad \text{ also } T \in B(X, Y) \]
\end{beweis}

\begin{bemerkung}[Exponentialfunktion]
	$A \in B(X)$, $X$ Banachraum
	\begin{itemize}
		\item Frage: $e^{tA}$
		\item Idee: $e^{tA} = \sum_{n = 0}^{\infty} \frac{1}{n!} t^{n} A^{n}$
		\item Setze $S_{m} = \sum_{n = 0}^{m}	 \frac{1}{n!} t^{n} A^{n}$
	\end{itemize}
	z. z. $S_{m}$ ist eine Cauchy-Folge in $B(X)$ \\
	Seien $k, m \in \MdN, k > 0$
	\begin{align*}
		\| S_{k} - S_{m} \| & \leq \sum_{n = m +1}^{k} \| \frac{1}{n!} t^{n} A^{n} \| \\
			& \leq \sum_{n = m +1}^{k}  \frac{1}{n!} | t^{n} | \| A^{n} \| \xrightarrow[k, m \rightarrow \infty]{} 0
	\end{align*}
	Da $B(X)$ vollständig ist, ist $e^{tA} = \lim_{m \rightarrow \infty} S_{m}$ in $B(X)$.
\end{bemerkung}

\begin{prop}[Neumann'sche Reihe] \label{prop:5.8-NeumannscheReihe}
	Sei $A \in B(X)$, $X$ ein Banachraum mit $\| A \| < 1$. \\
	Dann ist $Id - A$ invertierbar und 
	\[ \left( Id - A \right)^{-1} = \sum_{n = 0}^{\infty} A^{n} \]
	\begin{beweis}
		$S_{m} = \sum_{n = 0}^{m} A^{n}$ ist eine Cauchy-Folge in $B(X)$, denn für
		\begin{align*}
		k > m: \quad \| S_{k} - S_{m} \| & \leq \sum_{n = m + 1}^{k} \| A^{n} \| \\
			& \leq \sum_{n = m + 1}^{k} \| A \|^{n} \rightarrow 0 \text{ für } m, n \rightarrow \infty \text{, da} \| A \| < 1
		\end{align*}
		$R := \lim_{m \rightarrow \infty} S_{m}$ existiert in $B(X)$, da $B(X)$ vollständig.
		\[ S_{m} (Id - A) = (Id - A) S_{m} = Id - A^{m} \]
		mit $\| A^{m} \| \leq \| A \|^{m} \rightarrow 0 \text{ für } m \rightarrow \infty$
		\[ \Rightarrow R (Id - A) = (Id - A) R = Id, \quad R = (Id - A)^{-1} \]
	\end{beweis}
\end{prop}

\begin{kor}
	$X$ sei ein Banachraum und $J: X \rightarrow X$ ein (surjektiver) Isomorphismus. \\
	Für $A \in B(X)$ und $\| A \| < \| J^{-1} \|^{-1}$ ist auch $J - A$ ein Isomorphismus \\ \\
	Insbesondere: $G = \{ T \in B(X): T \text{ stetig und invertierbar} \}$ ist eine offene Menge in $B(X)$.
	\begin{beweis}
		Da $J - A = J (Id - J^{-1}A) $ folgt:
		\[ \| J^{-1} A \| \leq \| J^{-1} \| \underbrace{\| A \|}_{< \| J^{-1} \|^{-1}} < 1 \]
		Nach \hyperref[prop:5.8-NeumannscheReihe]{5.8} ist $(Id - J^{-1} A)$ invertierbar mit
			\begin{align*}
				(J - A)^{-1} & = (Id - J^{-1} A)^{-1} J^{-1} \\
					& = \sum_{n = 0}^{\infty} (J^{-1} A)^{n} J^{-1}
			\end{align*}
	\end{beweis}
\end{kor}

\begin{prop} \label{prop:5.10}
	Sei $X$ ein normierter Raum, $Y$ ein Banachraum und $D \subset X$ ein dichter Teilraum. \\
	Jeder linearere Operator $T: X \rightarrow Y$ mit
		\[ \| T x \|_{Y} \leq M \| x \|_{X}, \quad \text{ für alle } x \in D \]
	lässt sich zu einem eindeutig bestimmten Operator $ \tilde T \in B(X, Y)$ mit $\| \tilde T \| \leq c$ fortsetzen.	
\end{prop}

\begin{kor} \label{kor:5.11}
	Sei $X$ ein normierter Banachraum, $D \subset X$ dicht in $X$ und sei eine Folge $T_{n} \in B(X, Y)$, wobei $(T_{n} x)$ eine Cauchy-Folge für jedes $x \in D$ sei. \\
		Dann gibt es genau einen Operator $T \in B(X, Y)$ mit
		\[ \lim_{n \rightarrow \infty} T_{n} x = T x \]
	\begin{beweis}
		Setze für $x \in D$: $Tx = \lim_{n \rightarrow \infty} T_{n}(x)$, da $Y$ vollständig. \\
		Dieses $T: X \rightarrow Y$ ist linear. \\
		Nach \hyperref[prop:5.10]{5.10} gibt es genau ein $\tilde T \in B(X, Y)$ mit $\tilde T x = T x$ für $x \in D$, $\| \tilde T \| \leq M$, denn für $x \in D$:
		\[ \| T x \| \leq \limsup \| T_{n} x \| \leq M \| x \| \]
		z. z. $\tilde T x = \lim_{n \rightarrow \infty} T_{n} x$ für alle $x \in X$. \\
		Zu $\epsilon > 0$ wähle $y \in D$ mit $\| x - y \| \leq \frac{\epsilon}{2 M}$. Dann:
		\begin{align*}
			\| \tilde T x - T_{n} x \| & \leq \| \tilde T x - T y \| + \| T y - T_{n} y \| +  \| T_{n} y - T_{n} x \| \\
				& \leq \| \tilde T \| \| x - y \| + \| T y - T_{n} y \| + \| T_{n} \| \| x - y\| 
		\end{align*}
		\[ \limsup_{n \rightarrow \infty}  \| \tilde T x - T_{n} x \| \leq M \frac{\epsilon}{2 M} + \limsup_{n \rightarrow \infty} \| T y - T_{n} y \| + M \frac{\epsilon}{2 M} \leq \epsilon + 0 \]
	\end{beweis}
\end{kor}

\begin{beispiel}
	$e_{n} (t) = e^{2 \pi n i t} = \cos(2 \pi n t) + i \sin(2 \pi n t), \quad D = \{ (\alpha_{n}) \in L^{2}(\MdZ):$ Nur endlich viele $\alpha_{n} \neq 0 \}$
	\[ T : \begin{cases} D \rightarrow L^{2}[0, 1] \\ (\alpha_{n}) \rightarrow \sum_{n \in \MdZ} \alpha_{n} e_{n} \end{cases} \]
	Wie kann man man unendlich Reihen $\sum_{n \in \MdZ} \alpha_{n} e_{n}$ definieren? \\
	Ohne Beweis aus der Fourieranalysis:
	\begin{itemize}
		\item Es gibt $(\alpha_{n}) \in \MdC^{2}$, so dass $\sum_{n \in \MdZ} \alpha_{n} e_{n}(t)$ nicht für alle $t \in [0, 1]$ konvergent.
		\item Für alle $(\alpha_{n}) \in \MdC^{2}$ konvergiert $\sum_{n \in \MdZ} \alpha_{n} e_{n}(t)$ punktweise für fast alle t $t \in [0, 1]$
		\item $\int_{0}^{1} e_{n} \overline e_{m} dt = \delta_{n, m}$
	\end{itemize}
	Für $(\alpha_{n}) \in D$ wie in der Linearen Algebra folgt:
	\begin{align*}
		\| \sum_{n} \alpha_{n} e_{n}(t) \|_{L^{2}[0, 1]}^{2} & = \int (\sum_{n} \alpha_{n} e_{n}(t)) \overline{(\sum_{m} \alpha_{m} e_{m}(t))} dt \\
		& = \sum_{n, m} \alpha_{n} \alpha_{m} \int e_{n}(t) \overline{e_{m}(t)} dt \\
		& = \sum_{n} |\alpha_{n}|^2
	\end{align*}
	D.h. $\| T (\alpha_{n}) \| \leq 1 \left( \sum_{n} |\alpha_{n}|^2 \right)^{\frac{1}{2}} = \| \alpha_{n} \|_{L^{2}}$, $(M = 1)$. Nach \hyperref[prop:5.10]{5.10} gibt es dann ein $\tilde T: \ell^{2}[0, 1] \rightarrow L^{2}(0, 1)$ mit $\| \tilde T \| \leq 1$ \\ \\
	\uline{Zusatz:} $T_{m}((\alpha_{n})) = \sum_{n = m}^{\infty} \alpha_{n} e_{n}$. $T_{m}: \ell^{2} \rightarrow L^{2}[0, 1],$ $\| Tm \| \leq 1$ \\
	Für $(\alpha_{n}) \in \ell^{2}$ gilt: $T (\alpha_{n}) = \lim_{m \rightarrow \infty} T_{m}(\alpha_{n})$. \\
	Nach \hyperref[kor:5.11]{Kor. 11}: $T_{m}(\alpha_{n}) \rightarrow T (\alpha_{n})$ in $L^{2}[0, 1]$ für alle $(\alpha_{n}) \in \ell^{2}$. Damit konvergiert die Partialsumme der Fourierreihen in $L^{2}[0, 1]$.
\end{beispiel}

\begin{lemma} \label{lemma:5.13}
	Für einen normierten Raum $(X, \| \cdot \|)$ sind äquivalent:
	\begin{enumerate}[label=\alph*\upshape)]
		\item $X$ ist vollständig
		\item Jede absolut konvergente Reihe $\sum_{n \geq 1} x_{n}$ mit
			\[ x_{n} \in X, \sum_{n \geq 1} \| x_{n} \| < \infty \]
			hat einen Limes in $X$.
	\end{enumerate}	
\end{lemma}
\begin{beweis}
	$a) \Rightarrow b)$	$y_{n} = \sum_{m = 1}^{n} x_{m}$ ist eine Cauchy-Folge in $X$, denn für $l > n$:
		\[ \| y_{l} - y_{n} \| = \| \sum_{m = n + 1}^{l} x_{m} \| \leq \sum_{m = n + 1}^{l} \| x_{m} \| \xrightarrow[n, l \rightarrow \infty]{} 0 \]
	Also ist $(y_{n})$ eine Cauchy-Folge und konvergiert da $X$ vollständig ist. \\ \\
	$b) \Rightarrow a)$	Sei $(x_{n})$ eine Cauchy-Folge in $X$, d.h. für $\epsilon = 2^{-k} (k \in \MdN)$ gibt es ein $n_{k}$ so, dass:
		\[ \| x_{n} - x_{m} \| \leq 2^{-k} \text{ für } n, m \geq n_{k} \]
		Wähle zu $k \in \MdN$ ein $x_{n_{k}}$ so, dass
		\[ \| x_{n_{k+1}} - x_{n_{k}} \| \leq 2^{-k}.\]
		Setze $y_{0} = x_{k_{1}}$ und für $k \in \MdN: y_{k} = y_{n_{k+1}} - y_{n_{k}}$. \\
		Dann gilt: $\sum_{k \geq 1} \| y_{k} \| \leq \sum_{k \geq 1} 2^{-k} < \infty$. Nach $b)$ konvergiert die Reihe $\sum_{k \geq 1} y_{k}$ in $Y$. \\
		$\sum_{j = 0}^{k} x_{j} \overset{Teleskopsumme}{=} x_{n_{k}} \Rightarrow $ die Folge $(x_{n_{k}})_{k \in \MdN}$ konvergiert in $X$. \\
		$(x_{n})$ ist eine Cauchy-Folge $\Rightarrow \lim_{n \rightarrow \infty} x_{n}$ existiert in $X$.
\end{beweis}

\begin{kor}
	Sei $X$ ein Banachraum und $M \subset X$ ein abgeschlossener, linearer Teilraum. \\
	Dann $\hat X = \QR{X}{M}$ ist vollständig.
	\begin{beweis}[mit Lemma 5.13]	
		Sei $x_{k} \in X$ so, dass für $\hat x_{k} \in \hat X$ gilt:
		\[ \sum_{k \geq 1} \| \hat x_{k} \|_{\hat X} < \infty \]
		Nach der Definition des Quotientenraums kann man annehmen, dass
		\[ \| x_{k} \|_{X} \leq \| \hat x_{k} \|_{\hat X} + \frac{1}{2^{k}}, \quad k \in \MdK \quad \text{(vgl. \hyperref[def:1-2.15]{Quotientennorm})} \]
		Dann $\sum_{k \geq 1} \| x_{k} \|_{X} \leq \sum_{k \geq 1} \| \hat x_{k} \|_{\hat X} + \sum_{k \geq 1} 2^{-k} < \infty$ \\
		Da $X$ vollständig ist, gibt es nach \hyperref[lemma:1-5.13]{5. 13} ein $x \in X$ mit:
			\[ x = \lim_{n \rightarrow \infty} \sum_{k = 1}^{n} x_{k} \text{ in } X. \]
		Anwendung der Quotientenabbildung liefert $\hat x = \lim_{n \rightarrow \infty} \sum_{k = 1}^{n} \hat x_{k}$ konvergiert in $\hat X$.
	\end{beweis}
\end{kor}

\subsection{Anhang zu Kapitel 5: Vervollständigung}
$\MdQ \rightarrow \MdR, \quad C[0, 1], \quad \int_{0}^{1} |f(t)| dt \rightarrow L^{1}[0, 1]$ \\
Sei $X$ ein normierter Vektorraum, $M \subset X$ beliebig, $d(x, y) := \| x - y \|$, wobei $x, y \in M$ und damit $(M, d)$ ein metrischer Raum.

\begin{definition} \label{def:5.15-Lipschitz}
	$f: M \rightarrow \MdR$ hei{\ss}t \begriff{Lipschitz}, falls:
	\[ \sup_{x, y \notin M, x \neq y} \frac{|f(x) - f(y)|}{d(x, y)} = \underbrace{\| f \|_{L}}_{Lipschitz-Konstante} < \infty \]
\end{definition}

\begin{bemerkung}
	Sei $(M, d)$ ein metrischer Raum, $x_{0} \in M$ fest. \\
	\[ X = \{ f: M \rightarrow \MdR: \quad f \text{ Lipschitz}, f(x_{0}) = 0 \} \text{ ist bezüglich } \| \cdot \|_{L} \text{ ein normierter Raum.} \]	
	Dann ist $X' = B(X, \MdR)$ vollständig.
	\begin{beweis}
		Seien $f, g \in X$, dann gilt:  
		\begin{align*}
			\| f + g \|_{L} & = \sup_{x, y \in X} \frac{|f(x) + g(x) - f(y) - g(y)|}{d(x, y)} \\
							& \leq \sup_{x, y \in X} \frac{|f(x) - f(y)|}{d(x, y)} + \sup_{x, y \in X} \frac{|g(x) - g(y)|}{d(x, y)} \\
							& = \| f \|_{L} + \| g \|_{l} 			
		\end{align*} 
	\end{beweis}
\end{bemerkung}

\begin{satz}
	Sei $(M,d)$ ein metrische Raum, $x_{0} \in M$ fest, $X$ definiert wie in \hyperref[def:5.15-Lipschitz]{5.15}: \\ \\
	Zu $x \in M$ definiere $F_{x} \in X'$ durch $F_{x}(f) = f(x)$ für $f: M \rightarrow \MdR$ in $X$. \\
	Dann ist $x \in M \rightarrow F_{x} \in X'$ eine Abbildung, die eine isometrische Einbettung von $M$ nach $X'$ gibt, d.h. 
	\[ d(x, y) = \| F_{x} - F_{y} \|_{X'} \]
\end{satz}
\begin{beweis}
	Für die eine Richtung betrachten wir:
	\begin{align*}
		\| F_{x} - F_{y} \|_{X'}  & = \sup_{\| f \|_{L} \leq 1} |(F_{x} - F_{y})(f)| \\
								  & = \sup_{\| f \|_{L} = 1} |(F_{x} - F_{y})| \\
								  & = \left( \sup_{\| f \|_{L} = 1} \frac{|f(x) - f(y)|}{d(x, y)} \right) d(x, y) \\
								  & \leq 1 \cdot d(x, y)
	\end{align*}
	Zur Umkehrung wähle für $x \in M$ fest: $f(z) = d(x, z)$. Für diese Funktion gilt nach umgekehrter Dreiecksungleichung: \\
	\[ |f(z_{1} - f(z_{2})| = | d(x, z_{1}) - d(x, z_{2}) | \leq d(z_{1}, z_{2}) \Rightarrow \| f \|_{L} \leq 1 \]
	\begin{align*}
		\| F_{x} - F_{y} \| = \sup_{\|y \| \leq 1} |(F_{x} - F_{y})(y)| & \geq (F_{x} - F_{y})(f) \\
																		& = f(x) - f(y) = d(x, y)
	\end{align*}
	\[ \Rightarrow d(x, y) = \| F_{x} - F_{y} \| \]
	$F_{x_{0}} = 0$ in $X', \| F_{x} \|_{X'} = \| F_{x} - F_{x_{0}} \| = d(x, x_{0}) < \infty$
\end{beweis}

\begin{beispiel}
	\begin{enumerate}[label=\alph*\upshape)]	
		\item $M =$ Polynome auf $[0, 1]$
		\item missing example b
		\item missing example c	
	\end{enumerate}
\end{beispiel} % todo missing sth at 5.18: add examples

\begin{satz}
	Zu jedem metrischen Raum $(M, d)$ gibt es eine Vervollständigung, die bis auf Isometrie eindeutig bestimmt ist.
\end{satz}
\begin{beweis}
	missing proof %todo missing sth at 5.19: proof
\end{beweis}

\begin{bemerkung}
Alternativer Beweis der Existenz der Vervollständigung (nach Cantor) für einen normierten Raum $(X, \| \cdot \|)$:
	missing a part + proof % todo missing sth at 5.20: proof
\end{bemerkung}


\newpage






























