%!TEX root = Funktionalanalysis - Vorlesung.tex

\chapter{Vollst{\"a}ndigkeit}

\begin{definition}
	Sei $(M, d)$ ein metrischer Raum.
	\begin{enumerate}[label=\alph*\upshape)] \index{Cauchy-Folge} \index{vollständig} \index{Banachraum}
		\item $x_{n} \in M$ hei{\ss}t \textbf{Cauchy-Folge}, falls es zu jedem $\epsilon > 0$ ein $n_{0} \in \MdN$ gibt, sodass $\forall m, n \geq n_{0}$ gilt:
			\[ d(x_{n}, x_{m}) \leq \epsilon \]
		\item $(M, d)$ hei{\ss}t \textbf{vollständig}, falls jede Cauchyfolge $(x_{n}) \subset M$ einen Grenzwert \uline{in M} hat:
			\[ \lim_{n \rightarrow \infty} x_{n} = x \quad x \in M \]
		\item Ein normierter Raum $(X, \| \cdot \|)$, der vollständig ist bezüglich $d(x, y) = \| x - y \|$ heißt \textbf{Banachraum}.
	\end{enumerate}
\end{definition}

\begin{bemerkung}
	\begin{enumerate}[label=\alph*\upshape)]
		\item Jede konvergenzte Folge in $(M, d)$ ist eine Cauchy-Folge:
			\[ \text{Sei } \lim_{n \rightarrow \infty} x_{n} = x: \quad d(x_{n}, x_{m}) \leq d(x_{n}, x) + d(x, x_{m}) \xrightarrow[n, m \rightarrow \infty]{} 0 \]
		\item \uline{Aber:} nicht jede Cauchy-Folge eines normierten Raums $X$ konvergiert in $X$
			\begin{beispiel*}
				$X = C[0, 2], \quad \| f \|_{1} = \int_{0}^{2} | f(t) | dt, \quad
				f_{n}(x) = \begin{cases}x^{n} & \text{ für } x \in [0, 1] \\ 1 & \text{ für } x \in [1, 2]\end{cases}$	
				\[ f_{n}(x) \rightarrow f(x) = \begin{cases} 0 & \text{ für } x \in [0, 1) \\ 1 & \text{ für } x \in [1, 2] \end{cases} \quad \text{ für feste } x \in [0, 2] \]
				Nach dem Satz von Lebesgue folgt $\| f - f_{n} \|_{1} \rightarrow 0$ für $n \rightarrow \infty$, aber $f \notin C[0, 2]$ \\
				Demnach ist $f_{n}$ zwar eine Cauchy-Folge, aber $f_{n}$ konvergiert nicht gegen $f$ in $X$ bezüglich der $\| \cdot \|_{1}$-Norm.
			\end{beispiel*}
	\end{enumerate}	
\end{bemerkung}

\begin{prop}
	Sei $X$ ein metrischer Raum, $Y$ ei Banachraum.
	\[ C(x, Y) = \{ f : X \rightarrow Y: f \text{ stetig} \}, \quad \| f \|_{\infty} = \sup_{x \in X} \| f(x) \|_{Y} \]
	Dann ist $C(X, Y)$ ein (linearer) Banachraum.
\end{prop}

\begin{beispiel*}
	$\Omega \subseteq \MdR^{n}, \quad C(\Omega, \MdR)$
	\begin{beweis}
		Sei $(f_{n})$ eine Cauchy-Folge in $C(X, Y)$. \\
		Für alle $x \in X$:
		\[ \| f_{n}(x) - f_{m}(x) \|_{Y} \leq \| f_{n} - f_{m} \|_{\infty} \xrightarrow[n, m \rightarrow \infty]{} 0 \]
		Für alle $x \in X$: $(f(x))_{n \in \MdN}$ ist eine Cauchy-Folge in $Y$.
		da $Y$ vollständig ist, existiert $f(x) := \lim_{n \rightarrow \infty} f_{n}(x)$ in Y. \\ \\
		z.z. $f \in C(X, Y), \quad \| f - f_{n} \|_{\infty} \rightarrow 0$. \\
		Zu $\epsilon > 0$ gibt es ein $n_{0}$, sodass für alle $x \in X$:
		\[ \| f_{n}(x) - f_{m}(x) \|_{Y} \leq \| f_{n} - f_{m} \|_{\infty} \leq \epsilon \]
		Für jedes $x \in X$ fest folgt für $m \rightarrow \infty$:
		 \[ \| f_{n}(x) - f(x) \| \leq \epsilon \quad \text{ für } n \geq n_{0} \]
		 \[ \Rightarrow \| f_{n} - f \|_{\infty} \leq \epsilon \text{ für } n \geq n_{0} \quad  \text{ nehme das Supremum über } x \in X \]
		 $f \in C(X, Y)$, da der gleichmä{\ss}ige Limes stetiger Funktionen stetig ist.
	\end{beweis}
\end{beispiel*}

\begin{beispiel}
	Sei $\Omega \subseteq \MdR^{n}$ offen und beschränkt. $C^{m}(\bar \Omega)$ ist vollständig bezüglich der Supremums-Norm.	
	\begin{beweis}
		Für $C^{1}(\bar \Omega)$ gilt $\| f \|_{C^{1}} = \| f \|_{\infty} + \sum_{i = 1}^{n}  \| \frac{\partial}{\partial x_{i}} f \|_{\infty}$. \\
		 Sei $(f_{j}) \subset F$ in $C^{1}(\bar \Omega)$.
		 %todo: missing
	\end{beweis}
\end{beispiel}

\begin{prop}
	Hier fehlt etwas % todo: compelte missing parts	
\end{prop}


\begin{satz}
	Sei $X$ ein normiert Raum, $Y$ ein Banachraum.
	Dann ist $B(X, Y)$ mit der Operatornorm vollständig. \\
	Insbesondere: $X' = B(X, \MdK)$ ist immer  vollständig.
\end{satz}
\begin{beweis}
	Sei $(T_{n}) \subset B(X, Y)$ eine Cauchy-Folge bezüglich der Operatornorm. Sei $x \in X$
	\[ \| T_{n} x - T_{m} \|_{Y} \leq \| T_{n} - T_{m} \| \|x\|_{X} \]
	Also $(T_{n}x)_{n \in \MdN}$ eine Cauchy-Folge in $Y$ für alle $x \in X$. Definiere $T x : = \lim_{n \rightarrow \infty} T_{n} x$ \\
	z.z. $T \in B(X, Y),$ $s\| T_{n} - T \| \rightarrow 0$ \\
	\[ T_{n} (x + y)  = T_{n} x + T_{n} y \xrightarrow[n \rightarrow \infty]{} T(x + y) = Tx + Ty \]
	\begin{align*}
		\| Tx - T_{n}x \| & \overset{Norm \text{ } stetig}{=} \lim_{m \rightarrow \infty} \| T_{m} x - T_{n} x \| \\
		 & \leq \lim_{m \rightarrow \infty} \| T_{n} - T_{m} \| \| x \| \\
		 & \leq \epsilon \| x \| \text{ für n gro{\ss} genug.}
	\end{align*}
	Wobei $\| T - T_{n} \| \leq \epsilon$ für $n$ gro{\ss} genug.
	Also für $\| x \| \leq 1$:
	\[ \| T x \| \leq \| T_{n} x \| + \epsilon \leq \| T_{n} \| + \epsilon  \]
	\[ \Rightarrow \| T \| \leq \| T_{n} \| + \epsilon, \quad \text{ also } T \in B(X, Y) \]
\end{beweis}

\begin{bemerkung}[Exponentialfunktion]
	$A \in B(X)$, $X$ Banachraum
	\begin{itemize}
		\item Frage: $e^{tA}$
		\item Idee: $e^{tA} = \sum_{n = 0}^{\infty} \frac{1}{n!} t^{n} A^{n}$
		\item Setze $S_{m} = \sum_{n = 0}^{m}	 \frac{1}{n!} t^{n} A^{n}$
	\end{itemize}
	z. z. $S_{m}$ ist eine Cauchy-Folge in $B(X)$ \\
	Seien $k, m \in \MdN, k > 0$
	\begin{align*}
		\| S_{k} - S_{m} \| & \leq \sum_{n = m +1}^{k} \| \frac{1}{n!} t^{n} A^{n} \| \\
			& \leq \sum_{n = m +1}^{k}  \frac{1}{n!} | t^{n} | \| A^{n} \| \xrightarrow[k, m \rightarrow \infty]{} 0
	\end{align*}
	Da $B(X)$ vollständig ist, ist $e^{tA} = \lim_{m \rightarrow \infty} S_{m}$ in $B(X)$.
\end{bemerkung}

\begin{prop}[Neumann'sche Reihe] \label{prop:1-5.8-NMR}
	Sei $A \in B(X)$, $X$ ein Banachraum mit $\| A \| < 1$. \\
	Dann ist $Id - A$ invertierbar und 
	\[ \left( Id - A \right)^{-1} = \sum_{n = 0}^{\infty} A^{n} \]
	\begin{beweis}
		$S_{m} = \sum_{n = 0}^{m} A^{n}$ ist eine Cauchy-Folge in $B(X)$, denn für
		\begin{align*}
		k > m: \quad \| S_{k} - S_{m} \| & \leq \sum_{n = m + 1}^{k} \| A^{n} \| \\
			& \leq \sum_{n = m + 1}^{k} \| A \|^{n} \rightarrow 0 \text{ für } m, n \rightarrow \infty \text{, da} \| A \| < 1
		\end{align*}
		$R := \lim_{m \rightarrow \infty} S_{m}$ existiert in $B(X)$, da $B(X)$ vollständig.
		\[ S_{m} (Id - A) = (Id - A) S_{m} = Id - A^{m} \]
		mit $\| A^{m} \| \leq \| A \|^{m} \rightarrow 0 \text{ für } m \rightarrow \infty$
		\[ \Rightarrow R (Id - A) = (Id - A) R = Id, \quad R = (Id - A)^{-1} \]
	\end{beweis}
\end{prop}

\begin{kor}
	$X$ sei ein Banachraum und $J: X \rightarrow X$ ein (surjektiver) Isomorphismus. \\
	Für $A \in B(X)$ und $\| A \| < \| J^{-1} \|^{-1}$ ist auch $J - A$ ein Isomorphismus \\ \\
	Insbesondere: $G = \{ T \in B(X): T \text{ stetig und invertierbar} \}$ ist eine offene Menge in $B(X)$.
	\begin{beweis}
		Da $J - A = J (Id - J^{-}A) $ folgt:
		\[ \| J^{-1} A \| \leq \| J^{-1} \| \underbrace{\| A \|}_{< \| J^{-1} \|^{-1}} < 1 \]
		Nach \hyperref[prop:1-5.8-NMR]{5.8} ist $(Id - J^{-1} A)$ invertierbar mit
			\begin{align*}
				(J - A)^{-1} & = (Id - J^{-1} A)^{-1} J^{-1} \\
					& = \sum_{n = 0}^{\infty} (J^{-1} A)^{n} J^{-1}
			\end{align*}
	\end{beweis}
\end{kor}

\begin{prop} \label{prop:1-5.10}
	Sei $X$ ein normierter Raum, $Y$ ein Banachraum und $D \subset X$ ein dichter Teilraum. \\
	Jeder linearere Operator $T: X \rightarrow Y$ mit
		\[ \| T x \|_{Y} \leq M \| x \|_{X}, \quad \text{ für alle } x \in D \]
	lässt sich zu einem eindeutig bestimmten Operator $ \tilde T \in B(X, Y)$ mit $\| \tilde T \| \leq c$ fortsetzen.	
\end{prop}

\begin{prop} % todo: really a prop?
	Sei $X$ ein normierter Banachraum, $D \subset X$ dicht in $X$ und sei eine Folge $T_{n} \in B(X, Y)$, wobei $(T_{n} x)$ eine Cauchy-Folge für jedes $x \in D$ sei. \\
		Dann gibt es genau einen Operator $T \in B(X, Y)$ mit
		\[ \lim_{n \rightarrow \infty} T_{n} x = T x \]
	\begin{beweis}
		Setze für $x \in D$: $Tx = \lim_{n \rightarrow \infty} T_{n}(x)$, da $Y$ vollständig. \\
		Dieses $T: X \rightarrow Y$ ist linear. \\
		Nach \hyperref[prop:1-5.10]{5.10} gibt es genau ein $\tilde T \in B(X, Y)$ mit $\tilde T x = T x$ für $x \in D$, $\| \tilde T \| \leq M$, denn für $x \in D$:
		\[ \| T x \| \leq \limsup \| T_{n} x \| \leq M \| x \| \]
		z. z. $\tilde T x = \lim_{n \rightarrow \infty} T_{n} x$ für alle $x \in X$. \\
		Zu $\epsilon > 0$ wähle $y \in D$ mit $\| x - y \| \leq \frac{\epsilon}{2 M}$. Dann:
		\begin{align*}
			\| \tilde T x - T_{n} x \| & \leq \| \tilde T x - T y \| + \| T y - T_{n} y \| +  \| T_{n} y - T_{n} x \| \\
				& \leq \| \tilde T \| \| x - y \| + \| T y - T_{n} y \| + \| T_{n} \| \| x - y\| 
		\end{align*}
		\[ \limsup_{n \rightarrow \infty}  \| \tilde T x - T_{n} x \| \leq M \frac{\epsilon}{2 M} + \limsup_{n \rightarrow \infty} \| T y - T_{n} y \| + M \frac{\epsilon}{2 M} \leq \epsilon + 0 \]
	\end{beweis}
\end{prop}

\begin{beispiel}
	$e_{n} (t) = e^{2 \pi n i t} = \cos(2 \pi n t) + i \sin(2 \pi n t), \quad D = \{ (\alpha_{n}) \in L^{2}(\MdZ):$ Nur endlich viele $\alpha_{n} \neq 0 \}$
	\[ T : \begin{cases} D \rightarrow L^{2}[0, 1] \\ (\alpha_{n}) \rightarrow \sum_{n \in \MdZ} \alpha_{n} e_{n} \end{cases} \]
	Wie kann man man unendlich Reihen $\sum_{n \in \MdZ} \alpha_{n} e_{n}$ definieren? \\
	Ohne Beweis aus der Fourieranalysis:
	\begin{itemize}
		\item Es gibt $(\alpha_{n}) \in \MdC^{2}$, so dass $\sum_{n \in \MdZ} \alpha_{n} e_{n}(t)$ nicht für alle $t \in [0, 1]$ konvergent
		\item Für alle $(\alpha_{n}) \in \MdC^{2}$ konvergiert $\sum_{n \in \MdZ} \alpha_{n} e_{n}(t)$ punktweise für fast alle t $t \in [0, 1]$
	\end{itemize}
\end{beispiel}

\newpage






























