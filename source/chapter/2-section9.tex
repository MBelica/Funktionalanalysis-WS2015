%!TEX root = Funktionalanalysis - Vorlesung.tex

\chapter*{Elemente der Operatortheorie} \addcontentsline{toc}{chapter}{Elemente der Operatortheorie} \setcounter{section}{8}



\section{Der Satz von Baire und der Satz von Banach-Steinhaus}



\begin{satz}[Satz von Baire] \label{satz:9.1-baire} \index{Baire}
	Sei $(M, d)$ ein vollständiger metrischer Raum und seien $U_{n}, n \in \MdN$ offen und dicht in $M$.
	\[ \text{Dann ist } \bigcap_{n \in \MdN} U_{n} \text{ dicht in } M. \]
\end{satz}

\begin{beweis}
    Definiere der Kürze halber $D \coloneqq \bigcap_{n \in \MdN} U_{n}$. Damit werden wir zeigen, dass zu jeder Kugel $K(x_{0}, \epsilon)$, mit $x_{0} \in M, \epsilon > 0$ ein $x \in K(x_{0}, \epsilon) \cap D$ existiert.
    
	Beweis: 
	\begin{itemize}
		\item $U_{1} \cap K(x_{0}, \epsilon)$ ist offen und nichtleer, da $U_{1}$ dicht ist. Also existiert ein $x_{1} \in U_{1}$ und $\epsilon_{1} > 0$ mit $\epsilon_{1} < \frac{1}{2} \epsilon$ so, dass $K( x_{1}, 2 \epsilon_{1}) \subset U_{1} \cap K(x_{0}, \epsilon)$
			\[ \Rightarrow \overline{K(x_{1}, \epsilon_{1})} \subseteq U_{1} \cap K(x_{0}, \epsilon) \]
		\item Nun ist $U_{2} \cap K(x_{1}, \epsilon_{1})$  offen und nichtleer, da $U_{1}$ dicht ist. Also existiert ein $x_{2} \in U_{2}$ und ein $\epsilon_{2} < \frac{1}{2} \epsilon_{1}$ mit
			\[ \overline{K(x_{2}, \epsilon_{2})} \subset K(x_{2}, 2 \epsilon_{2}) \subset U_{2} \cap K(x_{1}, \epsilon_{1}) \subset U_{2} \cap U_{1} \cap K(x_{0}, \epsilon) \]
		\item Induktiv findet man Folgen $\epsilon_{n} > 0, x_{n}$ mit
			\begin{enumerate}[label=(\roman*\upshape)]
				\item $\epsilon_{n} < \frac{1}{2} \epsilon_{n - 1}$ und damit $\epsilon_{n} \leq \frac{1}{2^{n}} \epsilon$
				\label{satz:9.1-proof-ii}
				\item $\overline{K(x_{n}, \epsilon_{n})} \subset U_{n} \cap K(x_{n - 1}, \epsilon_{n - 1}) \subset \dotsc \subset U_{n} \cap \dotsc \cap U_{1} \cap K(x_{0}, \epsilon)$
			\end{enumerate}
	\end{itemize}
	Insbesondere für $n > N$:
	\[ d(x_{n}, x_{N}) \leq \sum_{j = N + 1}^{n} d(x_{j}, x_{j - 1}) \leq \sum_{j = N + 1}^{n} \epsilon_{n} \leq \left( \sum_{j = N + 1}^{\infty} \frac{1}{2^{j}} \right) \epsilon \]
	d.h. $(x_{n})$ ist eine Cauchy-Folge und da $M$ vollständig ist, existiert $\lim_{n \rightarrow \infty} x_{n} =: x$ in $M$. \\
	Mit \hyperref[satz:9.1-proof-ii]{ii} gilt: $x_{n} \in \overline{K(x_{N}, \epsilon_{N})}$ für $n \geq N$
	\[ \Rightarrow x = \lim_{n} x_{n} \in \overline{K(x_{N}, \epsilon_{N})} \subset U_{N} \cap \dotsc \cap U_{1} \cap K(x_{0}, \epsilon) \text{ für alle } N \]
	\[ \Rightarrow x \in \underbrace{\bigcap_{n = 1}^{\infty} U_{n}}_{= D} \cap K(x_{0}, \epsilon) \]
\end{beweis}


\begin{definition} \label{def:9.2}
	\begin{enumerate}[label=\alph*\upshape)]
		\label{def:9.2a}
		\item Eine Teilmenge $L$ eines metrischen Raums $M$ hei{\ss}t \begriff{nirgends dicht}, falls $\overline{L}$ keine inneren Punkte enthält.
		\label{def:9.2b}
		\item Eine Teilmenge $L$, die sich als Vereinigung von einer Folge von nirgends dichten Mengen $L_{n}$ darstellen lässt, d.h. $L = \bigcup_{n \in \MdN} L_{n}$ hei{\ss}t von \begriff{1. Kategorie}.
		\label{def:9.2c}
		\item $L$ hei{\ss}t von \begriff{2. Kategorie}, falls $L$ nicht von erster Kategorie ist.
	\end{enumerate}
\end{definition}


\begin{bemerkung*}
	\begin{itemize}
		\item Ist $L$ nirgends dicht, dann ist $M \setminus \overline{L}$ dicht in $M$
		\item Ist $L$ nirgends dicht oder von Kategorie 1, dann ist $L$ $"$dünn$"$, $"$voller Löcher$"$.
	\end{itemize}	
\end{bemerkung*}


\begin{kor}[Kategoriensatz von Baire] \label{kor:9.3-KategoriensatzVonBaire}
	\begin{enumerate}[label=\alph*\upshape)]
		\item In einem vollständigen metrischen Raum $(M, d)$ liegt das Komplement einer Menge $L$ von 1. Kategorie stets dicht. Insbesondere:
		\item Ein vollständig metrischer Raum ist von 2. Kategorie
		\item Sei $(M, d)$ vollständig und $M_{n}, n \in \MdN$ eine Folge abgeschlossener Mengen mit $M = \bigcup_{n \in \MdN} M_{n}$. Dann enthält mindestens ein $M_{n}$ eine Kugel
	\end{enumerate}	
\end{kor}

\begin{beweis}
	$L = \bigcup_{n \in \MdN} L_{n}$, $L_{n}$ nirgends dicht. \\
	Damit gilt $L^{c} = \left( \bigcup_{n \in \MdN} L_{n} \right)^{c} = \bigcap_{n \in \MdN} L_{n}^{c} \supset \bigcap_{n \in \MdN} \overline{L_{n}}^{c}$, mit $\overline{L_{n}}^{c}$ offen. \\
	Da $L_{n}$ nirgends dicht ist, ist daher $\overline{L_{n}}^{c}$ dicht in $M$ \\
	Nach \hyperref[satz:9.1-baire]{9.1} ist auch $\bigcap_{n \in \MdN} \overline{L_{n}}^{c}$ dicht in $M$ und da $\bigcap_{n \in \MdN} \overline{L_{n}}^{c} \subset L^{c}$, ist auch $L^{c}$  dicht in $M$. \\ \\
	$a) \Rightarrow b) \Rightarrow c)$ nach Definition.
\end{beweis}


\begin{satz}
	$E = \{ x \in C[0, 1]:$ $x$ ist in keinem Punkt von $[0, 1]$ differenzierbar$\}$ ist dicht in $(C[0, 1], \|\cdot\|_{\infty})$. \\
	Insbesondere:
 		\begin{itemize}
			\item $E \neq \emptyset$
			\item $C^{1}[0, 1]$ ist von 1. Kategorie in $C[0, 1]$, also $C^{1}[0, 1] \subset C[0, 1]$ dicht
		\end{itemize}
\end{satz}

\begin{beweis}
	Betrachte 
	\[ E_{n} = \{ x \in C[0, 1]: \sup_{0 < |h| \leq \frac{1}{n}} \left| \frac{x(t + h) - x(t)}{h} \right| > n, \text{ für alle } t \in [0, 1] \}, \]
	dann ist $\bigcap E_{n} \subset E$. Wir wolle nun zeigen, dass 
	\begin{enumerate}[label=\roman*\upshape)]
		\item $E_{n}$ sind offen in $C[0, 1]$ für alle $n$
		\item $E_{n}$ sind dicht in $C[0, 1]$ für alle $n$
	\end{enumerate}
	Damit können den \hyperref[satz:9.1-baire]{Satz von Baire} anwenden und erhalten dass $\bigcap_{n \in \MdN} E_{n}$ und damit auch die Menge $E$ dicht in $C[0, 1]$ ist. \\
	\begin{enumerate}[label=\roman*\upshape)]
		\item Sei $n \in \MdN$ und $x \in E_{n}$ fest. \\
			Zu jedem $t \in [0, 1]$ wähle $f_{t}$ definiert durch
			\[ \sup_{0 < |h| \leq \frac{1}{n}} \left| \frac{x(t + h) - x(t)}{h} \right| = n + 2 \delta_{t} \]
			$\Rightarrow \left| \frac{x(s + h_{t}) - x(s)}{h_{t}} \right| > n + \delta_{t}$ für $s \in U_{t}$, $0 < |h_{t}| \leq \frac{1}{n} $ \\
			Da $x$ stetig ist, gibt es zu $t$ eine kleine Umgebung $U_{t}$ mit
			\[ \left| \frac{x(s + h_{t}) - x(s)}{h_{t}} \right| > n + \delta_{t} \text{ für } s \in U_{t}, \quad 0 < |h_{t}| \leq \frac{1}{n} \]
			Da $[0, 1]$ kompakt und $[0, 1] = \bigcup_{t \in [0, 1]} U_{t}$ gibt es endlich viele $U_{t_{1}}, \dotsc, U_{t_{n}}$ mit $[0, 1] \subset U_{t_{1}} \cup \dotsc \cup U_{t_{n}}$ \\
			Setze $\delta \coloneqq \min \{ \delta_{t_{1}}, \dotsc, \delta_{t_{1}} \} > 0, \quad h \coloneqq \min \{ h_{t_{1}}, \dotsc, h_{t_{n}} \}$. \\
			Wähle nun ein $\epsilon$ mit $0 < \epsilon < \frac{1}{2} h \delta$. \\
			Zu zeigen bleibt $K(x, \epsilon) \subset E_{n}$: \\
			Sei $y \in C[0, 1]$ mit $\| x - y \|_{\infty} < \epsilon$. Zu $t \in [0, 1]$ wähle $i \in \{ 1, \dotsc, n \}$ mit $t \in U_{i}$. \\
			Dann:
			\begin{align*}
				\left| \frac{y(t + h_{t_{i}}) - y(t)}{h_{t_{i}}} \right| & \overset{\triangle-\text{Ungl.}}{\geq} \left| \frac{x(t + h_{t_{i}}) - x(t)}{h_{t_{i}}} \right| - 2 \frac{\| x - y \|_{\infty}}{|h_{t_{i}}|} \\
				& \overset{t \in U_{t_{i}}}{>} n + \delta - 2 \frac{\epsilon}{n} \\
				& > n \quad \text{nach Wahl von } \epsilon
			\end{align*}
			$\Rightarrow x \in E_{n}, K(x, \epsilon) \subseteq E_{n} \Rightarrow A_{n}$ offen, $n \in \MdN$.
		\item Wir wollen noch zeigen, dass $U_{n}$ dicht in $C[0, 1]$ ist. \\
			Sei $V \subset C[0, 1]$ offen, $V \neq \emptyset$. Nach dem Approximationssatz von Weierstra{\ss} gibt es ein Polynom $p$ mit $p \in V, \epsilon > 0: \| x - p \|_{\infty} \leq \epsilon \Rightarrow x \in V$. \\
			Sei weiter $y_{m}$ eine Sägezahnfunktion mit $y_{m}: [0, 1] \rightarrow [0, \epsilon]$ und der Steigung $\pm m$.
			Dann ist $x \coloneqq p + {m} \in K(p, \epsilon)$. \\
			Wähle zu $n$ die Zahl $m \in \MdN$ so, dass $m > n + \| p \|_{\infty}$. \\
			Für $t \in [0, 1], 0 < | n | \leq \frac{1}{n}$ gilt:
			\[ \left| \frac{x_{m}(t + h) - x(t)}{h} \right| \overset{\triangle-\text{Ungl.}}{\geq} \left| \frac{y_{m}(t + h) - y(t)}{h} \right| - \underbrace{\left| \frac{p(t + h) _ p(t)}{h} \right|}_{\leq{\| p' \|_{\infty}} \text{nach MWS}}  \]
			\[ \Rightarrow \sup_{0 < |h| \leq \frac{1}{n}} \left| \frac{x_{m}(t + h) - x_{m}(t)}{h} \right| \geq m - \| p' \|_{\infty} \overset{\text{nach Wahl}}{\underset{\text{von }m}{\geq}} n \] \\
			Damit ist $x_{m} \in E_{n}$ und sogar $x_{m} \in E_{n} \cap V$ \\ \\
			$\Rightarrow E_{n} \cap V \neq \emptyset \quad \Rightarrow E_{n} \text{ dicht.}$
	\end{enumerate}	
\end{beweis}


\begin{satz}[Banach-Steinhaus] \label{satz:9.5-Banach-Steinhaus} \index{Banach-Steinhaus}
	Sei $X$ ein Banachraum, $Y$ ein normierter Raum, $I$ eine Indexmenge und $(T_{i})_{i} \in B(X, Y)$.
	Falls: 
	\[ \sup_{i \in I} \| T_{i} x \| = C(X) < \infty \quad \forall x \in X \]
	Dann:
	\[ \sup_{i \in I} \| T_{i} \| = \sup_{i \in I} \sup_{\| x \| \leq 1} \| T_{i} x \| < \infty. \]
\end{satz}

\begin{beweis}
	Betrachte die Menge $E_{n} \coloneqq \{ x \in X: \sup \| T_{i} x \| \leq n\}$. Dann ist $E_{n}$ abgeschlossen, denn $E_{n} = \bigcap_{i \in \MdN} \{ x \in I: \| T_{i} x \| \leq n \}$. Außerdem ist $E_{n}$ symmetrisch, d.h. $x \in E_{n} \gdw -x \in E_{n}$ und konvex, denn für $x_{1}, x_{2} \in E_{n}$ gilt
		\[ \sup \| T_{i} \left( \frac{1}{2} x_{1} + \frac{1}{2} x_{2} \right) \| \leq \frac{1}{2} \sup \| T_{i} x_{1} \| + \frac{1}{2} \sup \| T_{i} x_{2} \| \leq n \quad \text{also } \frac{1}{2} x_{1} + \frac{1}{2} x_{2} \in E_{n} \]
		Es gilt $X = \sum_{n = 1}^{\infty} E_{n}$, nach \hyperref[satz:9.1-baire]{Baire} existiert dann ein $N \in \MdN, x \in E_{N}, \epsilon > 0$ mit $K(x, \epsilon) \subset E_{N} \Rightarrow K(-x, \epsilon) \subseteq E_{n}$ \\
		Sei $z \in X: \| z \| < \epsilon$. $z = \frac{1}{2} (x + z) + \frac{1}{2} (z - x) \in \frac{1}{2} E_{N} + \frac{1}{2} E_{N} \subseteq E_{N}$. 
		\[ \Rightarrow \sup_{i \in \MdN} \sup_{\| z \| \leq 1} \| T_{i} z \| = \sup_{i \in \MdN} \sup_{\| z \| < \epsilon} \| T_{i} \frac{z}{\epsilon} \| \leq \sup_{i \in \MdN} \sup_{z \in E_{N}} \frac{1}{\epsilon} \| T_{i} z \| \leq \frac{N}{\epsilon} \]	
\end{beweis}


\begin{bemerkung}  \label{bem:9.6}
	\begin{enumerate}[label=\alph*\upshape)]
		\item Aus $C(X)$ kann man $\sup_{i \in I} \| T_{i} \|$ nicht herleiten (\hyperref[satz:9.3-KategoriensatzVonBaire]{Baire} ist nicht konstruktiv).
		\item Die Vollständigkeit von $X$ ist notwendig.
	\end{enumerate}	
\end{bemerkung}


\begin{beispiel*}
Sei $F = \{ (x_{n})_{n}: x_{n} = 0$ für alle, bis auf endlich viele $n \}$ mit $\| (x_{n})_{n} \|_{p} = \sup_{n \in \MdN} | x_{n} |$. \\
Betrachte $T_{k} (x_{n})_{n} = k x_{k}$, dann gilt $\| T_{k} \| = k$ und $\sup_{k \in \MdN} \| T_{k} \| = \infty$ \\
Aber $\sup_{k \in \MdN} \| T_{k} (x_{n})_{n} \| = \sup_{k \in \MdN} \| k x_{k} \| < \infty$, da $\sup$ nur über endlich viele Werte genommen wird.
\end{beispiel*}


\begin{kor} \label{kor:9.7}
	Sei $X$ ein Banachraum, $Y$ normiert und $(T_{n})_{n} \subset B(X, Y)$ derart, dass
	\[ \lim_{n \rightarrow \infty} T_{n} x =: y \text{ existiert für alle } x \in X \]
	Dann ist $T_{x} \coloneqq y_{x}$ linear und $T \in B(X, Y)$
	\begin{beweis}
		$T$ ist linear, da $\lim$ und $T_{n}$ jeweils linear sind. \\
		$(T_{n} x)_{n}$ ist beschränkt $\xRightarrow[\text{Steinhaus}]{\text{Banach-}} \sup_{n \in \MdN} \| T_{n} \| < \infty$ \\
		\[ \| T x \| = \lim_{n \rightarrow \infty} \| T_{n} x \| \leq \sup_{n \in \MdN} \|T_{n} \| \cdot \|x \| \]
		\[ \Rightarrow T \in B(X, Y) \text{ mit } \| T \| \leq \sup_{n \in \MdN} \| T_{n} \|. \]
	\end{beweis}
\end{kor} 
Frage: $T_{n} \xrightarrow[n \rightarrow \infty]{} T$ in $B(x, Y)$? Nein! \\
\begin{beispiel*}
	$X = Y = \ell^{p}, \quad T_{k} (x_{n})_{n} = (x_{1}, \dotsc, x_{k}, 0, \dotsc)$ \\
	\[ \| T_{k} (x_{n})_{n} - I (x_{n})_{n} \| = \left( \sum_{n = k + q}^{\infty} | x_{n} |^{p} \right)^{\frac{1}{p}} \xrightarrow[n \rightarrow \infty]{} 0 \quad \text{(Identität ist punktweiser Gw.)} \]
	Aber: $\| T_{k} - I \| \geq \|(T_{k} - I)\underbrace{(0, \dotsc, 0, 1, 0, \dotsc)	}_{1 \text{ an } k+1-\text{ter Stelle}} \|_{\ell^{p}} = 1 \xrightarrow[k \rightarrow \infty]{} 0$
\end{beispiel*}



\newpage