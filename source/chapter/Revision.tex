%!TEX root = Funktionalanalysis - Vorlesung.tex

\chapter*{Anhang} \addcontentsline{toc}{chapter}{Anhang} 


\section*{Wiederholungen} \addcontentsline{toc}{section}{Wiederholungen} 



Zu jedem Kapitel eine kleine Wiederholung der einzelnen Sätze und dazu Beispiele. Sobald ich die Zeit finde, werde ich die auch sortieren an die entsprechend Kapitel dran hängen. Bitte denkt dran, dass das hier keine vollständige Liste der wichtigsten Sätze ist, lediglich ein paar der Sätze die für die Aufgabenblätter benötigt wurden:	 % todo: revision

\subsection*{Wiederholung zu Aufgabenblatt 10}

\textbf{Zu Aufgabe 29: Spektrum kompakter Operatoren:} \\
Sei $X$ ein endlich dimensionaler Banachraum, $K \in B(X)$ kompakt
	\[ \Rightarrow 0 \in \sigma(K) \]
Außerdem ist $\sigma(K)$ endlich oder besteht aus einer Nullfolge \textbf{und} jedes $\lambda \in \sigma(K) \setminus \{ 0 \}$ ist ein Eigenwert mit endlich dimensional. 
~ \newline

\textbf{Zu Aufgabe 30: Hilberträume:} \\
Sei $X$ ein Vektorraum über $\MdK$.
\begin{enumerate}
	\item Wir nennen eine Abbildung $\< \cdot , \cdot \> \colon X^{2} \rightarrow \MdK$
		Skalarprodukt, falls
 		\begin{description}
 			\item[$\hspace{0.5cm} (S1) \hspace{0.1cm} $] $\< \alpha x + \beta y , z \> = \alpha \<  x , z \> + \beta \< y , z \>$
 			\item[$\hspace{0.5cm} (S2) \hspace{0.1cm} $] $\< x , y \> = \overline{\< y , z \>}$
 			\item[$\hspace{0.5cm} (S3) \hspace{0.1cm} $] $\< x , x \> \geq 0$, $\< x , x \> = 0 \gdw x = 0$
 		\end{description}
 		Bedingungen $(S1), (S2)$ werden als Sesquilinearität und die Bedingung $(S3)$ als positive Definitheit bezeichnet. Das Tupel $(X , \< \cdot , \cdot \>)$ nennen wir dann einen Prä-Hilbertraum.
 	\item Ein Prä-Hilbertraum $(X , \< \cdot , \cdot \>)$ ist stets ein normierter Vektorraum bezüglich der Norm $\| x \| = \sqrt{\< x , x \>}$.
 	\item Ist ein Prä-Hilbertraum bezüglich der vom Skalarprodukt induzierten Norm vollständig, so nennen wir $(X , \< \cdot , \cdot \>)$ bzw. $(X , \| \cdot \|)$ einen Hilbertraum.
\end{enumerate}

Bemerkung: Sei $(X, \| \cdot \|)$ ein normierter Vektorraum, dann ist $\| \cdot \|$ genau dann von einem Skalarprodukt induziert, wenn die Parallelogrammgleichung gilt, d.h.
	\[ \| x + y \|^{2} + \| x - y \|^{2} = 2 \| x \|^{2} + 2 \| y \|^{2} \]
~ \newline

\textbf{Für das nächste Aufgabenblatt:} \\
Sei $X$ ein Hilbertraum

\begin{enumerate}
	\item $x, y \in X$ hei{\ss}en orthogonal $(x \bot y)$, falls $\< x , y \> = 0$
	\item todo % todo at revision
\end{enumerate}


\subsection*{Wiederholung zu Aufgabenblatt 11}

\textbf{Für das nächste Aufgabenblatt:} \\
Adjungierte Operatoren in Hilberträumen: \\

Beispiel (Links- und Rechtsshift) \\
Sei $X = \ell^{2}$ und $x \in X$
\begin{align*}
	R \colon \ell^{2} \rightarrow \ell^{2}, & R x = ( 0 , x_{1} , x_{2}, \dotsc) \\
	L \colon \ell^{2} \rightarrow \ell^{2}, & L x = ( x_{2}, x_{3}, x_{4}, \dotsc)
\end{align*}

Dann gilt todo % todo at revision


\newpage