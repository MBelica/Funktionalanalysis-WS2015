%!TEX root = Funktionalanalysis - Vorlesung.tex


\chapter*{Dualität in Banachräumen} \addcontentsline{toc}{chapter}{Dualität in Banachräumen} \setcounter{section}{19}



\section{Der Fortsetzungssatz von Hahn-Banach}



\begin{motivation}
	Sei $X$ ein Hilbertraum und {\Longunderstack{\hbox{$M \subseteq X$} \hbox{$T \in B(X)$} \hbox{schwach konv.}}}: {\Longunderstack{\hbox{$M^{\bot}$} \hbox{$T^{*} \in B(X)$} \hbox{Kompaktheit}}}
	\[ \begin{xy} \xymatrix{  
			X \ar[rd]^L \ar[d]|{ \rotatebox[origin=c]{270}{$\supseteq$}  }          \\
      		M \ar[r]^l    				  &   \MdK			
		} \end{xy} \] % todo at 20.0.5: check if I've understood drawing correctly 
	Wir suchen zu $l \in M'$ ein $L \in X'$ mit $L|_{M} = l$ und $\| L \| = \| l \|$. \\ \\
 	Zum Beispiel mit $X$ einem beliebigen Hilbertraum und $M \subset X$ betrachte
 		\[ \begin{xy} \xymatrix{
			X \ar[d]_{ P_{U} }                      \\
      		M \ar[r]^l    				   &   \MdK			
		} \end{xy} \]
		Dann besitzt $L \coloneqq l \circ P_{U}$ die gewünschten Eigenschaften.
\end{motivation}
 
 
\begin{definition}
	Sei $X$ ein Vektorraum über $\MdK$. Eine Abbildung $p \colon X \rightarrow \MdR$ hei{\ss}t \begriff{sublinear}, falls
 	\begin{enumerate}[label=\alph*\upshape)]
 		\item $p( x + y) \leq p(x) + p(y) \quad \forall x, y \in X$
 		\item $p(\lambda x ) = \lambda p(y) \quad \forall y \in X, \lambda \geq 0$
 	\end{enumerate}
\end{definition}


\begin{satz}[Hahn-Banach für Vektorräume] \index{Hahn-Banach für Vektorräume} \label{satz:20.2-HahnBanach}
		Sei $X$ ein Vektorraum, $p \colon X \rightarrow \MdR$ sublinear. \\
		Zu jeder linearen Abbildung $l \colon U \rightarrow \MdK$, $U \subset X$ Untervektorraum mit
			\[ \Re l(x) \leq p(x) \quad \forall x \in U \]
		gibt es eine lineare Abbildung $L \colon X \rightarrow \MdR$ mit 
			\[ L|_{U} = l, \quad \Re  L(x) \leq p(x) \quad \forall x \in X \]
		Es wird jedoch keine Eindeutigkeit behauptet.
\end{satz}

\begin{beweis}
	1. Schritt: sei $\MdK = \MdR$, $\dim \QR{X}{U} = 1$. Wähle $x_{0} \in \QR{X}{U}$, $x = \ospan(x) \oplus U$, d.h. zu $x \in X$ gibt es ein $u \in U, \lambda \in \MdR$ mit $x = u + \lambda x_{0}$ \\
	Für jedes $r \in \MdR$ erhalten wir eine lineare Abbildung:
	\[ L_{r} \colon X \rightarrow \MdR, \quad L_{r} (x) = l(u) + \lambda r, \quad L|_{U} = l \]
	Zeige dass $r$ so gewählt werden kann, dass für alle $x \in X$ gilt $L_{r}(x) \leq p(x)$ und 
	\[ l(u) + \lambda r \leq p(u + \lambda_{0} x_{0}) \quad \forall u \in U, \lambda \in \MdR \quad (1) \label{eq:20.2.5-1} \]
	Für $\lambda > 0$ bedeutet \hyperref[eq:20.2.5-1]{$(1)$}:
	\begin{align*}
		r & \leq p(\frac{u}{\lambda} + x_{0}) - l(\frac{u}{\lambda}) \quad \forall u \in U \\
		\gdw r & \leq \inf \{ p( v + x_{0}) - l(v) : v \in U \} \quad (2) \label{eq:20.2.5-2}
	\end{align*}
	Für $\lambda < 0$ bedeutet \hyperref[eq:20.2.5-1]{$(1)$}:
	\begin{align*}
		- r & \leq p(\frac{u}{- \lambda} - x_{0}) - l(\frac{u}{- \lambda}) \quad \forall u \in U \\
		\gdw r & \geq \sup \{ - p( w + x_{0}) + l(w) : w \in U \} \quad (3) \label{eq:20.2.5-3}
	\end{align*}
	Um \hyperref[eq:20.2.5-2]{$(2)$} und \hyperref[eq:20.2.5-3]{$(3)$} und damit \hyperref[eq:20.2.5-1]{$(1)$} zu erfüllen, benötigt man
	\begin{align*}
		l(w) - p(w - x_{0}) & \leq p(v + x_{0}) - l(v) \quad \forall v, w \in U \\
		\gdw l(w) + l(v) & \leq p(v + x_{0}) + p(w - x_{0}) \quad u, v \in U
	\end{align*}
	Damit $l(v) + l(w) = l(v + w) \leq p(v + w) \leq p(v) + p(w)$ \\
	Rückwärts betrachten liefert die erste Behauptung. \\ \\
	2. Schritt: sei $\MdK = \MdR$ und $U$ allgemein mit $U \subset U_{1} \subset U_{2} \subset \dotsc$ und $\dim \QR{U_{j}}{U_{j - 1}} = 1$ \\
	Da wir aber keine Topologie haben, können wir nicht annehmen, dass uns diese vollständige Induktion über z.B. $\bigcup_{j \in \MdN} U_{j}$ eine in X dichte Menge liefert.
	
	todo % todo rest des beweises
\end{beweis}

\textit{Vorläufig überspringe ich ab hier alle Beweise, werde sie aber nachtragen.} % todo add proof

\begin{satz}[Hahn-Banach für normierte Räume]
	Sei $X$ normierter Raum. Sei $U \subset X$ ein linearer Teilraum. \\
	Zu jedem stetigen Funktional $u' \colon U \rightarrow \MdK$, gibt es ein stetiges lineares Funktional $x' \colon X \rightarrow \MdK$ mit 
		\[ \| x' \|_{X'} = \| u' \|_{U'} \quad \text{und} \quad x'|_{U} = u'. \]	
\end{satz}


\begin{kor}
	Sei $X$ ein normierter Vektorraum. Zu jedem $x \in X, x \neq 0$ gibt es ein $x \in X$ mit 
		\[ \| x \|_{X'} \quad \text{und} \quad x'(x) = \| x \| \]
\end{kor}


\begin{folgerung}
	\begin{enumerate}[label=\alph*\upshape)]
		\item $x \in X, x'(x) = 0$ für alle $x' \in X^{*} \Rightarrow x = 0$
		\item Zu $x_{1}, x_{2} \in X$ gibt es $x' \in X'$ mit $x'(x_{1}) \neq x'(x_{2})$ (folgt aus $a)$ mit $x = x_{1} - x_{2}$)
		\item $\| x \| = \sup \{ x'(x) : \| x' \| = 1, x' \in X' \}$
	\end{enumerate}	
\end{folgerung}


\begin{kor}
	Sei $X$ ein normierter Raum. $U \subset X$ ein abgeschlossener, linearer Teilraum, $x \in X \setminus U$. \\
	Dann gibt es ein $x' \in X'$ mit $x'|_{U} = 0, x'(x) =\neq 0$.	
\end{kor}


\begin{definition}
	$X$ sei ein normierter Raum, $U \subset X, V \subset X'$ Teilmenge. Setze
	\begin{align*}
		U^{\bot} & \coloneqq \{ x' \in X' : x'(x) = 0 \text{ für alle } x \in X \} \\
		V_{\bot} & \coloneqq \{ x \in X : x'(x) = 0 \text{ für alle } x \in V \} 
	\end{align*}
	Beachte: $X'' \supset V^{\bot} = \{  x'' \in \left( X' \right)' = X'' : x''(x) = 0 \text{ für alle } x' \in V \} \neq V_{\bot} \subset X$
\end{definition}


\begin{bemerkung}
	\begin{enumerate}[label=\alph*\upshape)]
		\item $U^{\bot}$ ist ein linearer abgeschlossener Teilraum $X'$, $V_{\bot}$ ist ein linearer abgeschlossen Teilraum von $X$.
		\item $U_{1} \subset U_{2} \Rightarrow U_{2}^{\bot} \subset U_{1}^{\bot}$, $V_{1} \subset V_{2} \Rightarrow (V_{2})_{\bot} \subset (V_{1})_{\bot}$
		\item $(U^{\bot})_{\bot} = \overline{\ospan} U$
	\end{enumerate}	
\end{bemerkung}


\begin{prop}
	Sei $X$ ein Banachraum, $U \subset X$ abgeschlossen. Dann gilt
		\[ \left( \QR{X}{U} \right)' \equalhat U^{\bot}, \quad U' \equalhat \QR{X'}{U^{\bot}} \]	
\end{prop}

\begin{beweis}
	siehe Übung.	
\end{beweis}



\newpage