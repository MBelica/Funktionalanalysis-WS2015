%!TEX root = Funktionalanalysis - Vorlesung.tex


\chapter*{Dualität in Banachräumen} \addcontentsline{toc}{chapter}{Dualität in Banachräumen} \setcounter{section}{19}



\section{Der Fortsetzungssatz von Hahn-Banach}



\begin{motivation}
	Sei $X$ ein Hilbertraum und {\Longunderstack{\hbox{$M \subseteq X$} \hbox{$T \in B(X)$} \hbox{schwach konv.}}}: {\Longunderstack{\hbox{$M^{\bot}$} \hbox{$T^{*} \in B(X)$} \hbox{Kompaktheit}}}
	\[ \begin{xy} \xymatrix{  
			X \ar[rd]^L \ar[d]|{ \rotatebox[origin=c]{270}{$\supseteq$}  }          \\
      		M \ar[r]^l    				  &   \MdK			
		} \end{xy} \]
	Wir suchen zu $l \in M'$ ein $L \in X'$ mit $L|_{M} = l$ und $\| L \| = \| l \|$. \\ \\
 	Zum Beispiel mit $X$ einem beliebigen Hilbertraum und $M \subset X$ betrachte
 		\[ \begin{xy} \xymatrix{
			X \ar[d]_{ P_{U} }                      \\
      		M \ar[r]^l    				   &   \MdK			
		} \end{xy} \]
		Dann besitzt $L \coloneqq l \circ P_{U}$ die gewünschten Eigenschaften.
\end{motivation}
 
 
\begin{definition}
	Sei $X$ ein Vektorraum über $\MdK$. Eine Abbildung $p \colon X \rightarrow \MdR$ hei{\ss}t \begriff{sublinear}, falls
 	\begin{enumerate}[label=\alph*\upshape)]
 		\item $p( x + y) \leq p(x) + p(y) \quad \forall x, y \in X$
 		\item $p(\lambda x ) = \lambda p(y) \quad \forall y \in X, \lambda \geq 0$
 	\end{enumerate}
\end{definition}


\begin{satz}[Hahn-Banach für Vektorräume] \label{satz:20.2-Hahn-Banach} \index{Hahn-Banach für Vektorräume} \label{satz:20.2-HahnBanach}
		Sei $X$ ein Vektorraum, $p \colon X \rightarrow \MdR$ sublinear. \\
		Zu jeder linearen Abbildung $l \colon U \rightarrow \MdK$, $U \subset X$ Untervektorraum mit
			\[ \Re l(x) \leq p(x) \quad \forall x \in U \]
		gibt es eine lineare Abbildung $L \colon X \rightarrow \MdR$ mit 
			\[ L|_{U} = l, \quad \Re  L(x) \leq p(x) \quad \forall x \in X \]
		Es wird jedoch keine Eindeutigkeit behauptet.
\end{satz}

\begin{beweis}
	Wir gehen in 3 Schritten vor: 1. $\MdK = \MdR, \dim \QR{X}{U} = 1$, 2. $ \MdK = \MdR$, allgemeines $U$, 3. $\MdK = \MdC$ \\ \\
	1. Schritt: sei $\MdK = \MdR$, $\dim \QR{X}{U} = 1$. Wähle $x_{0} \in \QR{X}{U} \Rightarrow X = \ospan(x_{0}) \oplus U$, d.h. zu $x \in X$ gibt es ein $u \in U, \lambda \in \MdR$ mit $x = u + \lambda x_{0}$ \\
	Für jedes $r \in \MdR$ erhalten wir eine lineare Abbildung:
	\[ L_{r} \colon X \rightarrow \MdR, \quad L_{r} (x) = l(u) + \lambda r, \quad L_{r}|_{U} = l \]
	Zeige dass $r$ so gewählt werden kann, dass für alle $x \in X$ gilt
	\[  L_{r}(x) \leq p(x) \quad \gdw \quad l(u) + \lambda r \leq p(u + \lambda_{0} x_{0}) \quad \forall u \in U, \lambda \in \MdR \quad (1) \label{eq:20.2.5-1} \]
	Für $\lambda > 0$ bedeutet \hyperref[eq:20.2.5-1]{$(1)$}:
	\begin{align*}
		r & \leq p(\frac{u}{\lambda} + x_{0}) - l(\frac{u}{\lambda}) \quad \forall u \in U \\
		\gdw r & \leq \inf \{ p( v + x_{0}) - l(v) : v \in U \} \quad (2) \label{eq:20.2.5-2}
	\end{align*}
	Für $\lambda < 0$ bedeutet \hyperref[eq:20.2.5-1]{$(1)$}:
	\begin{align*}
		- r & \leq p(\frac{u}{- \lambda} - x_{0}) - l(\frac{u}{- \lambda}) \quad \forall u \in U \\
		\gdw r & \geq \sup \{ l(w) - p( w + x_{0}) : w \in U \} \quad (3) \label{eq:20.2.5-3}
	\end{align*}
	Um \hyperref[eq:20.2.5-2]{$(2)$} und \hyperref[eq:20.2.5-3]{$(3)$} und damit \hyperref[eq:20.2.5-1]{$(1)$} zu erfüllen, benötigt man
	\begin{align*}
		l(w) - p(w - x_{0}) & \leq p(v + x_{0}) - l(v) \quad \forall v, w \in U \\
		\gdw l(w) + l(v) & \leq p(v + x_{0}) + p(w - x_{0}) \quad \forall u, v \in U
	\end{align*}
	Diese Ungleichung gilt wegen $l(v) + l(w) = l(v + w) \leq p(v + w) \leq p(v) + p(w)$ $\forall v, w \in U$ und liefert die erste Behauptung. \\ \\
	2. Schritt: sei $\MdK = \MdR$ und $U$ allgemein. 
		\[ \mathcal{A} \coloneqq \{ (V, L_{V}) : U \subseteq V \subseteq X \text{ UVR }, L_{V} \colon V \rightarrow \MdR \text{ linear, mit } L_{V}|_{U} = l, ~ \Re L_{V}(x) \leq p(x) \forall x \in V \} \]
	Definiere eine Ordnung auf $\mathcal{A}:$ $(V_{1}, L_{1}) \leq (V_{2}, L_{2}) \gdw V_{1} \subseteq V_{2}, L_{2}|_{V_{1}} = L_{1}$. \\
	Zu zeigen ist: jede total geordnete Teilmenge von $\mathcal{A}$ hat eine obere Schranke. \\
	Sei $\mathcal{A}_{0}  \subseteq \mathcal{A}$ total geordnet, setze $V_{0} = \bigcup_{V \in \mathcal{A}} V$, dann ist $V_{0}$ ein Untervektorraum von $X$. Definiere $L_{0}(x) = L_{V_{j}}(x)$ falls $x \in V_{j} \in \mathcal{A}_{0}$. Dann ist $L_{0}$ wohldefiniert, linear und erfüllt die Abschätzung $\Rightarrow (V_{0}, L_{0}) \in \mathcal{A}$ und $(\hat{V}, \hat{L}) \leq (V_{0}, L_{0})$ für alle $(\hat{V}, \hat{L}) \in \mathcal{A}_{0} \Rightarrow (V_{0}, L_{0})$ ist eine obere Schranke für $\mathcal{A}_{0} \in \mathcal{A}$. \\ \\
	3. Sei $\MdK = \MdC$ und sei $X_{\MdR}$ der zu $X$ gehörige reelle Vektorraum. Definiere $l_{\MdR}(x) = \Re l(x) = \frac{1}{2} \left( l(x) + \overline{l(x)} \right)$. Nach $(ii)$ gibt es ein reelles lineares Funktional $L_{\MdR} \colon X_{\MdR} \rightarrow \MdR$ mit $L_{\MdR}|_{U} = l_{\MdR}$ und $L_{\MdR}(x) \leq p(x) ~ \forall x \in X_{\MdR}$. Setze $L(x) = L_{\MdR}(x) - i L_{\MdR}(ix)$, dann ist $L$ additiv und es gilt $L(ix) = L_{\MdR}(ix) - i L_{\MdR}(-x) = i \left( - i L_{\MdR}(ix) + L_{\MdR}(x) \right) = i L(x)$. Somit ist L linear, weiter ist $\Re L(x) = L_{\MdR}(x) \leq p(x) ~ \forall x \in X$ und $L|_{U} = l$.
\end{beweis}


\begin{satz}[Hahn-Banach für normierte Räume] \label{satz:20.3-Hahn-Banach} 
	Sei $X$ normierter Vektorraum. Sei $U \subset X$ ein linearer Teilraum. \\
	Zu jedem stetigen Funktional $u' \colon U \rightarrow \MdK$, gibt es ein stetiges lineares Funktional $x' \colon X \rightarrow \MdK$ mit 
		\[ \| x' \|_{X'} = \| u' \|_{U'} \quad \text{und} \quad x'|_{U} = u'. \]	
\end{satz}

\begin{beweis}
	Setze $l \coloneqq u'$ und $p(x) \coloneqq \| u'\| \cdot \| x \|$. Dann gilt 
		\[ \Re l(x) \leq | u'(x) | \leq \| u' \|_{U} \cdot \| x \| = p(x), ~ \forall x \in U \]
	Nach \hyperref[satz:20.2-Hahn-Banach]{20.2.} gibt es eine Fortsetzung $x'$ von $l$ auf $X$ mit $\Re x'(x) \leq \| u' \| \cdot \| x \|$ und $x'|_{U'} = u'$.	Zu $x' \in X'$ und $x \in X$ wähle $\alpha \in \MdC$ mit $|\alpha| = 1$ und 
	\[ | x'(x) | = \alpha \Re x'(x) \leq \| u' \| \| \alpha x \| = \| u' \| \| x \|\]
	Somit ist $\| x' \| \leq \| u' \|$.
\end{beweis}


\begin{kor} \label{kor:20.4}
	Sei $X$ ein normierter Vektorraum. Zu jedem $x \in X, x \neq 0$ gibt es ein $x' \in X'$ mit 
		\[ \| x' \|_{X'} = 1 \quad \text{und} \quad x'(x) = \| x \|_{X} \]
\end{kor}

\begin{beweis}
	Sei $U = \ospan \{ x \}, u'( \lambda x) \coloneqq \lambda \| x \|$. Dann ist $u'$ linear, $\| u \| \leq 1$, $u'(\frac{x}{\| x \|}) = 1$, also $\| u' \| = 1$. Sei $x'$ die Fortsetzung von $u'$ auf X wie in \hyperref[satz:20.3-Hahn-Banach]{20.3}, dann folgt
	\[ \| x' \| = \| u' \| = 1 \text{ und } x'(x) = u'(x) = \| x \| \]
\end{beweis}


\begin{folgerung}
	\begin{enumerate}[label=\alph*\upshape)]
		\item $x \in X, x'(x) = 0$ für alle $x' \in X' \Rightarrow x = 0$
		\item Zu $x_{1}, x_{2} \in X$ gibt es $x' \in X'$ mit $x'(x_{1}) \neq x'(x_{2})$ (folgt aus $a)$ mit $x = x_{1} - x_{2}$)
		\item $\| x \| = \sup \{ x'(x) : \| x' \| = 1, ~ x' \in X' \}$
	\end{enumerate}	
\end{folgerung}

\begin{beweis}
	$|x'(x) | \leq \| x' \| \cdot \| x \| \leq \| x \|$ für $\| x' \| \leq 1$.
	Für die Umkehrung wähle $x'$ wie in \hyperref[kor:20.4]{20.4}.
\end{beweis}


\begin{kor} \label{kor:20.6}
	Sei $X$ ein normierter Raum. $U \subset X$ ein abgeschlossener Untervektorraum, $x \in X \setminus U$. \\
	Dann gibt es ein $x' \in X'$ mit $x'|_{U} \equiv 0$ und $x'(x) \neq 0$.	
\end{kor}

\begin{beweis}
	Sei $q \colon X \rightarrow \QR{X}{U}$ Quotientenabbildung $\Rightarrow q(x) \in \QR{X}{U}, q(x) \neq 0 \xRightarrow[]{\ref{kor:20.4}}$ Wähle $l \in \left(\QR{X}{U}\right)'$ mit $l(q(x)) \neq 0$. Setze $x' = l \circ q \colon X \rightarrow \MdK \Rightarrow x' \in X'$ und $x'(x) = l(q(x)) \neq 0$.
\end{beweis}


\begin{definition}
	Sei $X$ ein normierter Raum, $U \subset X, V \subset X'$ Teilmenge. Setze
	\begin{align*}
		U^{\bot} & \coloneqq \{ x' \in X' : x'(x) = 0 \text{ für alle } x \in U \} \\
		V_{\bot} & \coloneqq \{ x \in X : x'(x) = 0 \text{ für alle } x' \in V \} 
	\end{align*}
	Beachte: $X'' \supset V^{\bot} = \{  x'' \in X'' : x''(x') = 0 \text{ für alle } x' \in V \} \neq V_{\bot} \subset X$
\end{definition}


\begin{bemerkung}
	\begin{enumerate}[label=\alph*\upshape)]
		\item $U^{\bot}$ ist ein linearer abgeschlossener Teilraum $X'$, $V_{\bot}$ ist ein linearer abgeschlossen Teilraum von $X$.
		\item $U_{1} \subset U_{2} \Rightarrow U_{2}^{\bot} \subset U_{1}^{\bot}$, $V_{1} \subset V_{2} \Rightarrow (V_{2})_{\bot} \subset (V_{1})_{\bot}$
		\item $(U^{\bot})_{\bot} = \overline{\ospan } \left( U \right)$
	\end{enumerate}	
\end{bemerkung}

\begin{beweis}
	$c)$ $"$$\supseteq$$"$ Nach Definition. $"$$\subseteq$$"$ Sei $x \notin \overline{\ospan}(U) \xRightarrow[]{\ref{kor:20.6}} \exists ~ x' \in X'$ mit $x'|_{\overline{\ospan}(U)} \equiv 0$ und $x'(x) \neq 0$
	\[ \Rightarrow x' \in U^{\bot} \text{, aber wegen } x'(x) \neq 0 \text{ gilt } x' \notin ( U^{\bot} )_{\bot}\]
\end{beweis}


\begin{prop}
	Sei $X$ ein Banachraum, $U \subset X$ abgeschlossen. Dann gilt
		\[ \left( \QR{X}{U} \right)' \equalhat U^{\bot}, \quad U' \equalhat \QR{X'}{U^{\bot}} \]	
\end{prop}

\begin{beweis}
	siehe Übung.	
\end{beweis}



\newpage