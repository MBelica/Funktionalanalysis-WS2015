%!TEX root = Funktionalanalysis - Vorlesung.tex

\section{Satz von der offenen Abbildung}



\begin{definition} \label{def:10.1-offAbbildung}
	Eine Abbildung zwischen metrischen Räumen heißt \begriff{offen}, wenn offene Mengen auf offene Mengen abgebildet werden.
\end{definition}


\begin{lemma} \label{lemma:10.2}
	Seien $X, Y$ normiert und $T: X \rightarrow Y$ ein linearer Operator. \\
	Dann sind äquivalent:
	\begin{enumerate}[label=\alph*\upshape)]
		\item $T$ ist offen
		\item $\exists \epsilon > 0: K_{y}(0, \epsilon) \subset T(K_{x}(0, 1))$
	\end{enumerate}
\end{lemma}

\begin{beweis}
	$"\Rightarrow":$ $T(K_{X}(0, 1))$ ist offen und $T(0) = 0$. \\
	$"\Leftarrow":$ Sei $U \subset X$ offen, $x \in U, \epsilon > 0: K_{X}(x, \epsilon) \subset \overline{U}$.
	\begin{align*}
		& \Rightarrow T( K_{X}(x, \epsilon)) \subset T(y) \\
		& \Rightarrow T_{X} + \epsilon T(K_{X}(0, 1)) = Tx + T(K_{X}(0, \epsilon) \subset T(U)
	\end{align*}
	Nach $b)$ $\exists \delta > 0: K_{Y}(0, \delta) \subset T(K_{X}(0, 1))$
	\begin{align*}
		& \Rightarrow K_{Y}(Tx, \epsilon \delta) = T x + \epsilon K_{Y}(0, \delta) \subset T(U) \\
		& \Rightarrow T(U) \text{ ist offen}
	\end{align*}	
\end{beweis}


\begin{satz}[von der offenen Abbildung] \index{Satz von der offenen Abbildung}  \label{satz:10.3-offeneAbbildung}
	Seien $X, Y$ Banachräume und $T \in B(X, Y)$. \\
	Dann gilt:
	\[ T \text{ surjektiv} \gdw T \text{ offen} \]
\end{satz}

\begin{beweis}
	$"\Rightarrow":$ Nach \hyperref[lemma:10.2]{Lemma 10.2} $\exists \epsilon > 0: K_{Y}(0, \epsilon) \subset T(K_{X}(0, 1)) $
	\begin{align*}
		& \Rightarrow K_{Y}(0, R) \subset T(K_{X}(0, \frac{R}{\epsilon})) \quad \forall R > 0 \\
		& \Rightarrow Y \subset T(X)
	\end{align*}
	$"\Leftarrow":$ siehe Werner (S. 154f). % todo: missing sth at 10.3: add proof
\end{beweis}


\begin{kor} \label{kor:10.4}
	Seien $X, Y$ Banachräume und $T \in B(X, Y)$ bijektv. Dann ist $T^{-1} \in B(Y, X)$	
\end{kor}

\begin{beweis}
	Nach \hyperref[satz:10.3-offeneAbbildung]{10.3} ist $T$ offen, d.h. ist $U \subset X$ offen, so ist auch $T(U)$ offen in $Y$. \\
	\[ \Rightarrow T(U) = \left( T^{-1} \right)^{-1}(U) \text{ offen} \]
	\[ \Rightarrow \text{ Urbilder offener Mengen sind offen unter } T^{-1} \Rightarrow T^{-1} \text{ ist stetig} \] 
	
	Die Inverser linearer Operatoren ist bekanntlich linear.	
\end{beweis}


\begin{kor}
	Sei $X$	ein Vektorraum, der mit $\| \cdot \|, \vertiii{\cdot}$ ausgestattet ein Banachraum wird. \\
	Gibt es ein $c > 0: \| x \| \leq c \cdot \vertiii{x}$ $\forall x \in X$, dann sind die Normen äquivalent (d.h. $\exists \hat{c} $ mit $\hat{c} \cdot \vertiii{x} \leq \| x \|$ $\forall x \in X$). 
\end{kor}

\begin{beweis}
	Nach Voraussetzung ist $I : (X, \vertiii{\cdot}) \rightarrow (X, \| \cdot \|)$ beschränkt. \\
	$\xRightarrow[\ref{kor:10.4}]{} I : (X, \| \cdot \|) \rightarrow (X, \vertiii{\cdot})$ beschränkt.
\end{beweis}



\newpage