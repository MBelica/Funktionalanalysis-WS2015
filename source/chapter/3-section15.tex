%!TEX root = Funktionalanalysis - Vorlesung.tex


\chapter*{Operatoren auf Hilberträumen} \addcontentsline{toc}{chapter}{Operatoren auf Hilberträumen} \setcounter{section}{14}



\section{Hilberträume}



\begin{definition} \label{def:15.1-Skalarprodukt}
	Sei $X$ ein Vektorraum über $\MdK$. Eine Abbildung
	\[ \< \cdot, \cdot \> \colon X \times X \rightarrow \MdK \]
	hei{\ss}t \begriff{Skalarprodukt}, falls für $x, y \in X, \lambda \in \MdK$ gilt:
	\begin{description}
	 	\label{def:15.1i}
	 	\item[$\hspace{0.5cm} (S1) \hspace{0.1cm} $] $\< x_{1} + x_{2}, y \> = \< x_{1}, y \> + \< x_{2}, y \>$ \\
			 $~\hspace{0.6cm} \< x, y_{1} + y_{2} \> = \< x, y_{1} \> + \< x, y_{2} \>$
 		\label{def:15.1ii}
	 	\item[$\hspace{0.5cm} (S2) \hspace{0.1cm} $] $\< \lambda x, y \> = \lambda \< x, y \>,$ $\< x, \lambda y \> = \overline{\lambda} \< x, y \>$
 		\label{def:15.1iii}
	 	\item[$\hspace{0.5cm} (S3) \hspace{0.1cm} $] $\< x, y \> = \overline{\< y, x \> }$
 		\label{def:15.1iv}
	 	\item[$\hspace{0.5cm} (S4) \hspace{0.1cm} $] $\< x, y \> \geq 0, \quad \< x, x \> = 0 \gdw x = 0$
	\end{description}
\end{definition}


\begin{prop} \label{prop:15.2}
	Sei $X$ ein Vektor mit Skalarprodukt $\< \cdot, \cdot \>$
	\begin{enumerate}[label=\alph*\upshape)] \label{prop:15.2a}
		\item Für $x, y \in X$ gilt die \begriff{Cauchy-Schwarz-Ungleichung}
			\[ | \< x, y \> |^{2} \leq \< x, x \> \cdot \< y, y \> \] \label{prop:15.2b}
		\item $\| x \| = \< x, x \>^{\frac{1}{2}}$ definiert eine Norm auf $X$
			Insbesondere: $\< x, y \> \leq \| x \| \cdot \| y \|$
	\end{enumerate}	
\end{prop}

\begin{bemerkung*}
	\[ \< x + y, x + y \> = \| x\|^{2} + 2 \Re \< x, y \> + \| y \|^{2} \quad (*) \label{eq:15.2-*} \]
\end{bemerkung*}

\begin{beweis}
	\begin{enumerate}[label=\alph*\upshape)]
		\item Für $\lambda \in \MdK$ gilt:
			\[ 0 \leq \< x + \lambda y, x + \lambda y \> = \< x, x \> + \overline{\lambda} \< x, y \> + \lambda \< x, y \> + |\lambda|^{2} \< y, y \> \]
			Für $\lambda \coloneqq - \frac{\< x, y \>}{\< y, y \>}$ folgt:
			\[ 0 \leq \< x, y \> - \frac{|\< x,y \>|^{2}}{\< y, y \>} - \frac{|\< x, y \>|^{2}}{\< y, y \>} + \frac{|\< x, y \>|^{2}}{\< y, y \>} \]
			\[ \gdw 0 \leq \< x, x \> \cdot \< y, y \> - |\< x, y \>|^{2} \Rightarrow \text{Behauptung } a) \]
		\item $\| x \| \geq 0, \quad \| x \| = 0 \gdw x = 0$ folgt aus \hyperref[def:15.1iv]{$(S4)$} \\
			\begin{align*}
				\| x + y \|^{2} = \< x + y, x + y \> & \overset{\hyperref[def:15.1-Skalarprodukt]{(S1), (S2)}}{=} \< x, x \> + \< x, y \> + \overline{\< x, y \>} + \< y, y \> \\
						& = \| x \|^{2} + 2 \Re \< x, y \> + \| y \|^{2}
			\end{align*}
			Damit ist auch \hyperref[eq:15.2-*]{$(*)$} gezeigt.
			\begin{align*}
				\| x + y \|^{2} & \leq \| x \|^{2} + \| x \| \cdot \| y \| + \| y \|^{2} \\
						& \overset{\hyperref[prop:15.2a]{a)}}{=} \left( \| x \| + \| y \| \right)^{2}
			\end{align*}
			$\Rightarrow$ Die Dreiecksungleichung gilt für $\| \cdot \|$.
	\end{enumerate}	
\end{beweis}


\begin{bemerkung} \label{bem:15.3}
	Man kann aus der in \hyperref[prop:15.2b]{$b)$} definierten Norm das Skalarprodukt zurückgewinnen durch:
	\begin{align*}
		\text{Falls } \MdK = \MdR: &  \< x, y \> = \frac{1}{4} \left( \| x + y \|^{2} - \| x - y \|^{2} \right) \\
		\text{Falls } \MdK = \MdC: &  \< x, y \> = \frac{1}{4} \left( \| x + y \|^{2} - \| x - y \|^{2} + i \| x + i y\|^{2} - i \| x - iy\|^{2} \right)
	\end{align*}
\end{bemerkung}


\begin{definition}
	Ein metrischer Raum $(X, \| \cdot \|)$ hei{\ss}t \begriff{Prä-Hilbertraum}, falls es ein Skalarprodukt $\< \cdot, \cdot \>$ auf $X \times X$ gibt mit
		\[ \| x \| = \< x, x \>^{\frac{1}{2}} \]
	Falls $(X, \| \cdot \|)$ au{\ss}erdem noch vollständig ist, dann hei{\ss}t $X$ ein \begriff{Hilbertraum}.	
\end{definition}


\begin{beispiel}
	\begin{enumerate}[label=\alph*\upshape)]
		\item $\MdC$ mit $\< x, y \> = \sum_{i = 1}^{n} x_{i} \overline{y_{i}}$ für $x = (x_{i}), y = (y_{i}) \in \MdC^{n}$ \\
			\[ \| x \| = \left( \sum | x_{i}|^{2} \right)^{\frac{1}{2}} \]
		\item $X = \ell^{2}, x = (x_{i})_{i \in \MdN} \in \ell^{2}, y = (y_{i})_{i \in \MdN} \in \ell^{2}$ \\
			\[ \< x, y \> = \sum_{\i \in \MdN} x_{i} \overline{y_{i}},  \quad \|x \| = \left( \sum_{i = 1}^{\infty} |x_{i}|^{2} \right)^{\frac{1}{2}} \]
			$X = \ell^{p}$ ist kein Hilbertraum für $n \neq 2$ (Übung).
		\item $X = C(\Omega), \Omega \subset \MdR^{n}$ beschränkt und abgeschlossen \\
			\[ \< x, y \> = \int_{\Omega} x(u) \overline{y(u)} du \]
			Dann ist $(X, \< \cdot, \cdot \>)$ ist ein Prä-Hilbertraum, aber nicht vollständig. \\
			$(L^{2}(\Omega), \< \cdot, \cdot \>)$ ist ein Hilbertraum, da vollständig.
	\end{enumerate}	
\end{beispiel}


\begin{bemerkung*}
	$L^{p}(\Omega)$ ist kein Hilbertraum für $n \neq 2$.
\end{bemerkung*}


\begin{satz} \label{satz:15.6}
	Ein normierter Raum $(X, \| \cdot \|)$ ist genau dann ein Prä-Hilbertraum, falls die sogenannte \begriff{Prallelogramm-Gleichung} gilt, d.h.
	\[ \forall x, y \in X: \quad \|x + y \|^{2} + \| x - y \|^{2} = 2 \| x \|^{2} + 2 \| y \|^{2} \quad (P) \label{eq:15.6-rallelogrammGleichung} \]
\end{satz}

\begin{beweis}
	Nehmen wir an $(X, \| \cdot \|)$ sei ein Prä-Hilbertraum $\Rightarrow$ \hyperref[eq:15.6-rallelogrammGleichung]{$(P)$} gilt (einfaches nachrechnen mit \hyperref[eq:15.2-*]{$(*)$}). \\
	Angenommen es gilt \hyperref[eq:15.6-rallelogrammGleichung]{$(P)$} und sei o.B.d.A $\MdK = \MdR$ (der Fall $\MdC$ absolut analog) 
	\[  \< x, y \> \coloneqq \frac{1}{4} \left( \| x + y \|^{2} - \| x - y \|^{2} \right) \]
	Überprüfe die Eigenschaften des Skalarproduktes:
	\begin{enumerate}[label=\roman*\upshape)]
		\item Zu zeigen: $\< x_{1} + x_{2}, y \> = \< x_{1}, y \> + \< x_{2}, y \>$
			\begin{align*}
				\< x_{1} + x_{2}, y \> & = \frac{1}{4} \left( \| x_{1} + x_{2} + y \|^{2} - \| x_{1} + x_{2} - y \|^{2} \right) \quad (1) \\
				\< x_{1}, y \> + \< x_{2}, y \> & = \frac{1}{4} \left( \| x_{1} + y \|^{2} + \| x_{2} + y \|^{2} - \| x_{1} - y \|^{2} - \| x_{2} - y \|^{2} \right) \quad (2)
			\end{align*}
			Wir müssen also zeigen, dass $(1) = (2)$. \\
			Nach \hyperref[eq:15.6-rallelogrammGleichung]{$(P)$} folgt:
			\begin{align*}
				\| x_{1} + x_{2} + y \|^{2} & = 2 \| x_{1} + y \|^{2} + 2 \| x_{2} \|^{2} - \| x_{1} - x_{2} + y \|^{2} \\
				\| x_{1} + x_{2} + y \|^{2} & = 2 \| x_{2} + y \|^{2} + 2 \| x_{1} \|^{2} - \| - x_{1} + x_{2} + y \|^{2} \\
			\end{align*}
			Addieren dieser beiden Gleichungen liefert
			\[ \| x_{1} + x_{2} + y \|^{2} =  \| x_{1} + y \|^{2} + \| x_{2} \|^{2} + \| x_{2} + y \|^{2} +  \| x_{1} \|^{2} - \frac{1}{2} \left( \| x_{1} - x_{2} + y \|^{2} + \| - x_{1} + x_{2} + y \|^{2} \right) \]
			Ersetze $y$ durch $(-y)$:
			\[ \| x_{1} + x_{2} - y \|^{2} =  \| x_{1} - y \|^{2} + \| x_{2} \|^{2} + \| x_{2} - y \|^{2} +  \| x_{1} \|^{2} - \frac{1}{2} \left( \| x_{1} - x_{2} - y \|^{2} + \| - x_{1} + x_{2} - y \|^{2} \right) \]
			Subtrahieren der letzten beiden Zeilen liefert damit:
			\[ \| x_{1} + x_{2} + y \|^{2} - \| x_{1} + x_{2} - y \|^{2} = \| x_{1} + y \|^{2} + \| x_{2} + y \|^{2} - \| x_{1} - y \|^{2} - \| x_{2} - y \| ^{2} \]
			Dividieren durch $4$ liefert gerade die Behauptung.
		\item klar, da $\MdK = \MdR$, mit $\< x, x \> = \| x \|$.
		\item Aussage $\lambda \< x, y \> = \< \lambda x, y \>$ gilt falls $\lambda \in \MdN$ (nach \hyperref[def:15.1i]{$(S1)$}) \\
			Nach Definition okay für $\lambda = 0, \lambda = -1$ und damit auch für $\lambda \in \MdZ$
			Für $\lambda \in \MdQ$, setze $\lambda = \frac{m}{n}$ mit $m, n \in \MdZ, n \neq 0$ 
			\[ n \< \lambda x, y \> = \< n \lambda x, y \> = \< m x, y \> = m \< x, y \> \]
			\[ \< \lambda x, y \> = \frac{m}{n} \< x, y \> = \lambda \< x, y \>, \quad \lambda \in \MdQ \]
			$\lambda \colon \MdR \rightarrow \< \lambda x, y \>, \lambda \colon \MdR \rightarrow \lambda \< x, y \>$ sind stetige Funktionen auf $\MdR$, die auf dichten Teilmenge von $\MdQ$ übereinstimmen.
	\end{enumerate}
\end{beweis}


\begin{satz}[Beste Approximation] \label{satz:15.7-besteApproximation} \index{Beste Approximation}
	Sei $X$ ein Hilbertraum und $K$ eine konvexe und abgeschlossene Teilmenge von $X$.
	\begin{enumerate}[label=\alph*\upshape)]
		\item Zu jedem $x \in X$ gibt es genau ein $y_{0} \in K$ so, dass
			\[ \| x - y_{0} \| = \inf \{ \| x - y \|: y \in K \} \]
		\item Dieses $y_{0} \in K$ ist charakterisiert durch die Ungleichung 
			\[ \Re \< x - y_{0}, y - y_{0} \> \leq 0 \quad (w) \label{eq:15.7.5-SkalarproductbestApproximation} \]
	\end{enumerate}
\end{satz}

\begin{beweis}
	\begin{enumerate}[label=\alph*\upshape)]
		\item Da Aussage invariant ist gegenüber von Translationen ist, sei o.B.d.A. $x = 0$ und $0 \notin K$. \\ \\
			Existenz: Setze $d \coloneqq \inf \{ \| y \| : y \in K \}$ \\
			Wähle eine Folge $y_{n} \in K$ mit $\lim_{n} \| y_{n} \| = d$; wir wollen zeigen, dass dann $(y_{n})$ eine Cauchy-Folge in $X$ ist. \\
			Da $K$ konvex ist, gilt $\frac{y_{n} + y_{m}}{2} \in K: \| \frac{y_{n} + y_{m}}{2} \| \geq d$. Mit \hyperref[eq:15.6-rallelogrammGleichung]{$(P)$} folgt:
			\[ d^{2} \leq \| \frac{y_{n} + y_{m}}{2} \|^{2} + \| \frac{y_{n} - y_{m}}{2} \|^{2} \overset{\hyperref[eq:15.6-rallelogrammGleichung]{$(P)$}}{=} \frac{1}{2} \| y_{n} \|^{2} + \frac{1}{2} \| y_{m} \|^{2} \xrightarrow[n, m \rightarrow \infty]{} d^{2} \]
			Also $\| y_{n} - y_{m} \| \xrightarrow[n, m \rightarrow \infty]{} 0$, d.h. $(y_{n})$ ist eine Cauchy-Folge, $\lim_{n \rightarrow \infty} y_{n} = y_{0}$ existiert in $X$ und $y_{0} \in K$, da $K$ abgeschlossen ist. \\
			Demnach $y_{0} \in K, \| y_{0} \| = \lim_{n} \| y_{n} \| = d = \inf \{ \| y \| : y \in K \}$. \\ \\
			Eindeutigkeit: Seien $y_{1}, y_{2} \in K$, $\|y_{1} \| = \|y_{2} \| = d, y_{1} \neq y_{2}$ \\
			Mit \hyperref[eq:15.6-rallelogrammGleichung]{$(P)$} gilt:
			\[ \left| \frac{y_{0} + y_{1}}{2} \right|^{2} < \left| \frac{y_{0} + y_{1}}{2} \right|^{2} + \left| \frac{y_{0} - y_{1}}{2} \right|^{2} \overset{\hyperref[eq:15.6-rallelogrammGleichung]{$(P)$}}{=} \frac{1}{2} \| y_{0} \|^{2} + \frac{1}{2} \| y_{1} \|^{2} = d^{2} \]
			Also $\frac{y_{0} + y_{1}}{2} \in K$ und $ \| \frac{y_{0} + y_{1}}{2} \| < d$, was ein Widerspruch zur Definition von $D$ ist
			\begin{vereinbarung}
				Das Element $y_{0} \in K$ im Satz hei{\ss}t das \begriff{Element bester Approximation} von $x$ zu $K$.
			\end{vereinbarung}
		\item Annahme: $y_{0} \in K$ erfüllt die Ungleichung. 
			\begin{align*}
				\| x - y \|^{2} & = \| x - y_{0} + y_{0} - y \|^{2} \\
								& = \| x - y_{0} \|^{1} + \| y_{0} - y \|^{2} + \underbrace{2 \Re \< x - y_{0}, y_{0} - y \>}_{ \geq 0} \\
								& \geq \| x - y_{0} \|^{2}
			\end{align*}
			Also ist $y_{0}$ das Element bester Approximation. \\ \\
			Sei umgekehrt $y_{0}$ das Element bester Approximation und $y \in K$. \\
			Für $t \in (0, 1)$ setze $ K \ni y_{t} = ( 1 - t ) y_{0} + t y$.
			\begin{align*}
				\| x - y_{0} \|^{2} & \leq \| x - y_{0} \|^{2} \\
									& = \< x - y_{0} + t ( y_{0} - y ), x - y_{0} + t ( y_{0}  - y ) \> \\
									& = \| x - y_{0} \|^{2} + 2 \Re \<  x - y_{0}, t ( y_{0} - y ) \> + t^{2} \| y_{0} - y \|^{2}
			\end{align*}
			$\Rightarrow \Re \< x - y_{0}, y - y_{0} \> \leq \frac{t}{2} \| y_{0} - y \|^{2}$ für $t \in (0, 1)$. Für $t \rightarrow 0$ folgt die Ungleichung.
	\end{enumerate}
\end{beweis}


\begin{bemerkung}[Winkel zwischen Vektoren]
		Seien $u, v \in H \setminus \{ 0 \}$ und definiere entsprechend $u_{1} = \frac{u}{\| u \|}, v_{1} = \frac{v}{\| v \|}$.
		\[ \Rightarrow 1 \geq \frac{|\< u, v \>|}{\|u \| \cdot \| v \|} = | \< u_{1}, v_{1} \> | = \cos(\alpha), \text{ wobei } \alpha \in [0, \pi) \text{ eindeutig gewählt.} \]
\end{bemerkung}



\newpage