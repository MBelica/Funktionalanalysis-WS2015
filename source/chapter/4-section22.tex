%!TEX root = Funktionalanalysis - Vorlesung.tex


\section{Schwache Konvergenz \& Reflexivität}


\begin{notation}
	Sei $X$ ein normierter Raum und $X'$ der Dualraum. Für $x \in X, x' \in X'$ setze
	\[ x'(x) \eqqcolon (x, x'), \]
	\[ \text{Dann ist} X \times X' \ni (x, x') \rightarrow ( x , x') \text{ eine Bilinearform auf } X \times X'. \]	
\end{notation}


\begin{bemerkung*}
	Ist $X$ ein Hilbertraum mit Skalarprodukt $\< \cdot , \cdot \>$, so existiert nach \hyperref[lemma:6.3-Riesz]{Riesz} ein 
	\[ J : X \rightarrow X', J(x)(y) = \< y, x \> \text{ (antilineare Isometrie)} \]
	Für $x' = J(x)$ gilt $(y , x') = (y , J(x)) = \< y , x \> = \< y , J^{-1} y' \>$. \\
	\textbf{Ziele:}
	\begin{itemize}
		\item schwache Konvergenz
		\item $X \equalhat X''$ (Reflexivität)
		\item schwache Kompaktheit
	\end{itemize}	
\end{bemerkung*}


\begin{definition}
	Sei $X$ ein normierter Raum und $X'$ der zugehörige Dualraum
	\begin{enumerate}[label=\alph*\upshape)]
		\item Eine Folge $(x_{n})_{n} \subset X$ \begriff{konvergiert schwach} gegen $x \in X$, falls 
			\[ (x_{n} , x') \xrightarrow[n \rightarrow \infty]{} (x , x'), \quad \forall x' \in X' \]
		\item Eine Folge $(x_{n}')_{n} \subset X'$ \begriff{konvergiert schwach*} gegen $y' \in X'$, falls 
			\[ (x , x_{n}') \xrightarrow[n \rightarrow \infty]{} (x, x'), \quad \forall x \in X \]
	\end{enumerate}
\end{definition}

\begin{notation*}
	\begin{enumerate}[label=\alph*\upshape)]
		\item $x_{n} \xrightarrow[]{w} x$ oder $x_{n} \rightarrow x$ in $\sigma(X , X')$ oder $x = w$-$\lim_{n} x_{n}$
		\item $x_{n}' \xrightarrow[]{w^{*}} x'$ oder $x_{n}' \rightarrow x'$ in $\sigma(X' , X)$ oder $x' = w^{*}$-$\lim_{n} x_{n}'$
	\end{enumerate}
\end{notation*}


\begin{eig}
	\begin{enumerate}[label=\alph*\upshape)]
		\item $w$-$\lim$ und $w^{*}$-$\lim$ sind (falls existent) eindeutig bestimmt
		\item Normkonvergenz in $X$ (oder $X'$) impliziert schwache (bzw. w*-schwache) Konvergenz; nicht umgekehrt.
		\item Falls $x_{k} \xrightarrow[]{w} x \Rightarrow \| x \| \leq \liminf_{n} \| x_{n} \|$ und falls $x_{k} \xrightarrow[]{w^{*}} xÄ \Rightarrow \| x' \| \leq \liminf_{n} \| x_{n} \|$ 
		\item Falls $X$ ein Banachraum ist und $(x_{n}) \subset X$ schwach konvergiert, so ist $(x_{n})$ beschränkt; das selbe gilt für $(x_{n}') \subset X'$
	\end{enumerate}	
\end{eig}


\begin{lemma}
	Sei $(x_{n}')$ beschränkt in $X'$ und $D \subset X$ mit $\overline{\ospan D} = X$. \\
	Falls $x_{n}'(x)$ eine Cauchy-Folge ist für alle $x \in D$, dann konvergiert $x_{n}'$ schwach* gegen ein $x' \in X'$
\end{lemma}


\begin{prop} \label{prop:21.5}
	Setze $J \colon X \rightarrow X'', J(x)(x') = x'(x)$, für alle $x' \in X$. \\
	Dann ist $J$ eine isometrische Einbettung von $X$ nach $X''$.	
\end{prop}


\begin{definition}
	Ein Banachraum hei{\ss}t \begriff{reflexiv}, falls $J \colon X \rightarrow X''$ (in \hyperref[prop:21.5]{21.5}) surjektiv ist ($X \equalhat X''$).
\end{definition}

\textit{Rest fehlt. Kommt diese Woche noch.}


\newpage