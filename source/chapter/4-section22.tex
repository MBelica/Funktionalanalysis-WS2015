%!TEX root = Funktionalanalysis - Vorlesung.tex


\section{Schwache Konvergenz \& Reflexivität}


\begin{notation}
	Sei $X$ ein normierter Raum und $X'$ der Dualraum. Für $x \in X, x' \in X'$ setze
	\[ (x, x') \coloneqq x'(x) , \]
	dann ist $X \times X' \ni (x, x') \rightarrow ( x , x') \in \MdK$ eine Bilinearform auf $X \times X'$.
\end{notation}


\begin{bemerkung*}
	Ist $X$ ein Hilbertraum mit Skalarprodukt $\< \cdot , \cdot \>$, so existiert nach \hyperref[lemma:6.3-Riesz]{Riesz} eine antilineare Isometrie
	\[ J : X \rightarrow X', J(x)(y) = \< y, x \> \]
	Für $x' = J(x)$ gilt $(y , x') = (y , J(x)) = \< y , x \> = \< y , J^{-1} x' \>$. \\ \\
	\textbf{Ziele:}
	\begin{itemize}
		\item schwache Konvergenz
		\item $X \equalhat X''$ (Reflexivität)
		\item schwache Kompaktheit
	\end{itemize}	
\end{bemerkung*}


\begin{definition}
	Sei $X$ ein normierter Raum und $X'$ der zugehörige Dualraum
	\begin{enumerate}[label=\alph*\upshape)]
		\item Eine Folge $(x_{n})_{n \geq 1} \subset X$ \begriff{konvergiert schwach} gegen $x \in X$, falls 
			\[ (x_{n} , x') \xrightarrow[n \rightarrow \infty]{} (x , x'), \quad \forall x' \in X' \]
		\item Eine Folge $(x_{n}')_{n \geq 1} \subset X'$ \begriff{konvergiert schwach*} gegen $y' \in X'$, falls 
			\[ (x , x_{n}') \xrightarrow[n \rightarrow \infty]{} (x, x'), \quad \forall x \in X \]
	\end{enumerate}
\end{definition}

\begin{notation*}
	\begin{enumerate}[label=\alph*\upshape)]
		\item $x_{n} \xrightarrow[]{w} x$ oder $x_{n} \rightarrow x$ in $\sigma(X , X')$ oder $x = w$-$\lim_{n} x_{n}$
		\item $x_{n}' \xrightarrow[]{w^{*}} x'$ oder $x_{n}' \rightarrow x'$ in $\sigma(X' , X)$ oder $x' = w^{*}$-$\lim_{n} x_{n}'$
	\end{enumerate}
\end{notation*}


\begin{eig}
	\begin{enumerate}[label=\alph*\upshape)]
		\item $w$-$\lim$ und $w^{*}$-$\lim$ sind (falls existent) eindeutig bestimmt
		\item Normkonvergenz in $X$ (oder $X'$) impliziert schwache (bzw. schwache*) Konvergenz; nicht umgekehrt.
		\item Falls $x_{k} \xrightarrow[]{w} x \Rightarrow \| x \| \leq \liminf_{n} \| x_{n} \|$ und falls $x_{k}' \xrightarrow[]{w^{*}} x' \Rightarrow \| x' \| \leq \liminf_{n} \| x_{n}' \|$ 
		\item Falls $X$ ein Banachraum ist und $(x_{n}) \subset X$ schwach konvergiert, so ist $(x_{n})$ beschränkt; das selbe gilt für $(x_{n}') \subset X'$
	\end{enumerate}	
\end{eig}


\begin{beweis}
	\begin{enumerate}[label=\alph*\upshape)]
		\item Gelte $x_{n} \xrightarrow[]{w} y_{1}, x_{n} \xrightarrow[]{w} y_{2} \Rightarrow x'(y_{1}) = \lim x'(x_{n}) = x'(y_{2})$ für alle $x' \in X'$
			\[ \Rightarrow x'(y_{1} - y_{2}) = 0 \text{ für alle } x' \in X' \Rightarrow y_{1} = y_{2} \]
		\item $| x'(x) | \leq \| x' \| \cdot \| x \|$
		\item $x_{n} \xrightarrow[]{w} x$. Nach \S 20 existiert $x' \in X'$ mit $\| x' \| = 1$ und $x'(x) = \| x \|$
			\[ \Rightarrow \| x \| = x'(x) = \lim x'(x_{n}) \leq \underbrace{\| x' \|}_{= 1} \| x_{n} \| \Rightarrow \| x \| \leq \liminf \| x_{n} \| \]
		\item Sei $(x_{n})$ schwach konvergent. Definiere $T_{n} \colon X' \rightarrow \MdK, T_{n}(x') = x'(x_{n})$. Für alle $x'$ ist $\{ T_{n}(x') \}$ beschränkt in $\MdK$. Mit \hyperref[satz:9.5-Banach-Steinhaus]{Banach-Steinhaus} folgt $\sup \| T_{n} \| < \infty$ und wegen $\| T_{n} \| = \| x_{n} \|$ folgt, dass $(x_{n})$ beschränkt ist. \\
			Für die zweite Behauptung definiere $T_{n} \colon X \rightarrow \MdK, T_{n} x \coloneqq x_{n}'(x)$ (erst hier wird die Vollständigkeit von $X$ benötigt)
	\end{enumerate}		
\end{beweis}


\begin{lemma}
	Sei $(x_{n}')$ beschränkt in $X'$ und $D \subset X$ mit $\overline{\ospan D} = X$. Falls $x_{n}'(x)$ eine Cauchy-Folge ist für alle $x \in D$, dann konvergiert $x_{n}'$ schwach* gegen ein $x' \in X'$
\end{lemma}

\begin{beweis}
	Sei $Y = \ospan(D)$, definiere $y' \colon Y \rightarrow \MdK, y'(\sum_{i = 1}^{n} \alpha_{i} x_{i}) = \lim_{n \rightarrow \infty}(\sum_{i = 1}^{n} \alpha_{i} x_{n}'(x_{i}))$. $y'$ ist beschränkt
	\[ | y'(x) | \leq \sup \| x_{n}' \| \cdot \| \underbrace{\sum \alpha_{i} x_{i}}_{= x} \| \]
	$\xRightarrow[]{\S 3}$ Es gibt eine stetige Fortsetzung $x'$ von $y'$ auf ganz $X$.
	\[ \Rightarrow x_{n}' \xrightarrow[]{w^{*}} x', ~ \| y' \| = \| x' \| \text{ denn } x'|_{Y} = y' \text{ und } \overline{Y} = X \] 
\end{beweis}


\begin{prop}[Kanonische Einbettung von X nach X''] \label{prop:21.5}
	Setze $J \colon X \rightarrow X'', J(x)(x') = x'(x)$, für alle $x' \in X$. \\
	Dann ist $J$ eine isometrische Einbettung von $X$ nach $X''$.	
\end{prop}

\begin{beweis}
	$J(x)$ linear, $|J(x)(x')| = |x'(x)| \leq \| x' \| \cdot \| x \|$ $\forall x' \in X'$ 
	\[ \Rightarrow J(x) \in X'', ~ \| J(x) \|_{X''} \leq \| x \| \]
	Zu $x \in X$ gibt es ein $x_{0}'$ mit $\| x_{0}' \| = 1$, $x_{0}'(x) = \| x \|$ 
	\[ \Rightarrow \| J(x) \|_{X''} \geq |J(x)(x_{0}') | = | x_{0}'(x) | = \| x \| \]
\end{beweis}


\begin{definition}
	Ein Banachraum hei{\ss}t \begriff{reflexiv}, falls $J \colon X \rightarrow X''$ (in \hyperref[prop:21.5]{21.5}) surjektiv ist ($X \equalhat X''$).
\end{definition}


\begin{beispiel}
	\begin{enumerate}[label=\alph*\upshape)]
		\item Sei $X = H$ HR, $X' = X$ (\hyperref[lemma:6.3]{Riesz}) $\Rightarrow X'' = X' = X$ reflexiv. Schwache Konvergenz im obigen Sinn stimmt überein mit der schwache Konvergenz im Hilbertraum-Sinne.
		\item $X = \ell^{p}, 1 < p < \infty$, dann ist $\left( \ell^{p} \right)' = \ell^{q}$ mit $\frac{1}{p} + \frac{1}{q} = 1 \Rightarrow \left(\ell^{p}\right)'' = \ell^{p}$ reflexiv
			\[ x_{n} = (\alpha_{n_{k}})_{k} \xrightarrow[]{w} x = (a_{k})_{k} \gdw \alpha_{n_{k}} \rightarrow \alpha_{k} \quad \forall k \in \MdN \]
		\item $(c_{0})' = \ell^{1}$, $(\ell^{1})' = \ell^{\infty} \Rightarrow J_{c_{0}} \colon c_{0} \hookrightarrow \ell^{\infty}$ nicht reflexiv
		\item $X = L^{p}(\Omega), L^{1}(\Omega)' = L^{\infty}(\Omega), L^{p}(\Omega)' = L^{q}(\Omega), \frac{1}{p} + \frac{1}{q} = 1,  p, q \in (1, \infty)$
			\[ \Rightarrow L^{p}(\Omega) \text{ reflexiv für } p \in (1, \infty) \]
			Für $x, x_{n} \in L^{p}(\Omega)$ gilt $x_{n} \xrightarrow[]{w} x \gdw \int_{A} x_{n} d\mu \rightarrow \int_{A} x d\mu$ $\forall A \in \mathcal{A}$ (falls $(x_{n})$ beschränkt)
		\item $X = C(K)$, $K$ kompakt. Sei $(x_{k}) \subseteq C(K)$ beschränkt und $x \in C(K)$. Dann gilt
			\[ x_{k} \xrightarrow[]{w} x \gdw x_{k}(u) \rightarrow x(u) \text{ für alle } u \in K \]
	\end{enumerate}
\end{beispiel}


\begin{definition}
	\begin{enumerate}[label=\alph*\upshape)]
		\item Sei $X$ ein normierter Vektorraum, $U \subseteq X$ hei{\ss}t \begriff{schwach kompakt}, falls zu jeder Folge $(x_{n}) \subseteq U$ eine Teilfolge $x_{n_{k}}$ und ein $x \in U$ existiert mit $x_{n_{k}} \xrightarrow[]{w} x$
		\item $V \subseteq X'$ hei{\ss}t \begriff{schwach*-kompakt}, falls jede Folge $(x_{n}') \subseteq V$ eine Teilfolge $(x_{n_{k}}')$ besitzt, sodass ein $x' \in V$ existiert mit $x_{n_{k}}' \xrightarrow[]{w^{*}} x'$
	\end{enumerate}
\end{definition}


\begin{satz}[Alaoglu-Bourbaki] \index{Alaoglu-Bourbaki}
	Sei $X$ separabel, dann ist $U_{X'}$ w*-kompakt.	
\end{satz}

\begin{beweis}
	\textit{todo} % todo todo
\end{beweis}


\begin{prop}
	\begin{enumerate}
		\item Ein abgeschlossener Untervektorraum $U$ eines reflexiven Raumes $X$ ist reflexiv.
		\item Ein Banachraum $X$ ist genau dann reflexiv, falls sein Dualraum $X'$ reflexiv ist.
	\end{enumerate}
\end{prop}

\begin{beweis}
	\textit{todo} % todo todo
\end{beweis}


\begin{lemma}
	Sei $X$ reflexiv, dann gilt:
	\[ X \text{ ist separabel } \gdw X' \text{ ist separabel} \]
\end{lemma}

\begin{beweis}
	\textit{todo} % todo todo
\end{beweis}


\begin{satz}
	Sei $X$ reflexiv, dann ist $U_{X}$ schwach folgenkompakt (und umgekehrt).
\end{satz}

\begin{beweis}
	\textit{todo} % todo todo
\end{beweis}


\begin{satz}
	Sei $X$ ein normierter Vektorraum und $V \subseteq X$ konvex und abgeschlossen. Sei $(x_{n}) \subseteq V$ und $x_{n} \xrightarrow[]{w} x$, dann ist $x \in V$.
\end{satz}

\begin{beweis}
	\textit{todo} % todo todo
\end{beweis}


\begin{kor}[Satz von Mazur] \index{Mazur}
	Ist $X$ ein normierter Vektorraum, so gibt es für eine Folge $(x_{n})_{n \geq 1} \subseteq X$ mit $x_{n} \xrightarrow[]{w} x$ eine Folge von Linearkombinationen
	\[ y_{n} = \sum_{j = 1}^{k_{n}} \lambda_{j} x_{j} \text{ mit } \sum_{j = 1}^{k_{n}} \lambda_{j} = 1 \text{ und } \lambda_{j} \geq 0 \]
	so dass $\| y_{n} - x \| \rightarrow 0$.
\end{kor}

\begin{beweis}
	\textit{todo} % todo todo
\end{beweis}


\newpage