%!TEX root = Funktionalanalysis - Vorlesung.tex



\section{Orthogonalität und Orthonormalbasen}


\begin{definition}
	Sei $X$ ein Prähilbertraum
	\begin{enumerate}[label=\alph*\upshape)]
		\item $x, y \in X$ hei{\ss}en \begriff{orthogonal}, falls $\< x, y \> = 0$. Schreibweise: $x \bot y$
		\item $A, B \subseteq X$ sind orthogonal, falls $\< x, y \> = 0$ für alle $x \in A, y \in B$. Schreibweise $A \bot B$
		\item Sei $A \subset X$. $A^{\bot} = \{ y \in X: \< y, x \> = 0$ $\forall x \in A \}$ ist das \begriff{orthogonale Komplement} von $A$ in $X$.
	\end{enumerate}
\end{definition}


\begin{bemerkung} \label{bem:16.2}
	\begin{enumerate}[label=\alph*\upshape)]
		\item $x \bot y \Rightarrow \| x + y \|^{2} = \| x \|^{2} + \| y \|^{2} $ (Pythagoras) \index{Pythagoras}
		\item $A \subseteq B \Rightarrow B^{\bot} \subseteq A^{\bot}$
		\item $A^{\bot}$ ist stets ein abgeschlossener Unterraum von $X$
		\item $A \subseteq \left( A^{\bot} \right)^{\bot}$, $A^{\bot} = \overline{ \ospan(A) }^{\bot}$
	\end{enumerate}	
\end{bemerkung}


\begin{satz}[Orthogonalzerlegung] \index{Orthogonalzerlegung} \label{satz:13.6-Orthogonalzerlegung}
	Sei $X$ ein Hilbertraum und $U$ ein abgeschlossener Teilraum von $X$.
		\[ \text{Dann gilt: } X = U \oplus U^{\bot} \]
\end{satz}

\begin{beweis}
	Falls $x \in U \cap U^{\colon}$, dann $x \bot x \Rightarrow x = 0$. Sei $x \in X$, dazu gibt es eine Element bester Approximation $x_{1} \in U$. Setze $x_{2} \coloneqq x - x_{1}$. \\
	Zu zeigen ist $x_{2} \in U^{\bot}$ \\
	Nach \hyperref[eq:15.7.5-SkalarproductbestApproximation]{$(w)$} gilt für alle $y \in U: \Re \< x_{2}, y - x_{1} \> \leq 0 \Rightarrow \Re \< x_{2}, y \> \leq 0$ $\forall y \in U$, denn mit $y$ durchläuft auch $y - x_{1}$ ganz $U$ \\
	$\Rightarrow \< x_{2}, y \> = 0$ $\forall y \in U$, denn mit $y$ durchläuft auch $-y, iy$ und $-iy$ ganz $U$ $\Rightarrow x_{2} \in U^{\bot}$.
\end{beweis}


\begin{definition*}
	Sei $X$ ein Hilbertraum, $U \subseteq X$ abgeschlossen und $X = U \oplus U^{\bot}$. Für $X \ni x = x_{1} + x_{2}$, mit $x_{1} \in U, x_{2} \in U^{\bot}$ definiere
		\[ P_{U} \colon X \rightarrow U, ~ P x = x_{1} \]
		$P_{U}$ hei{\ss}t \begriff{Orthogonalprojektion} von $X$ auf $U$.	
\end{definition*}


\begin{folgerung} \label{folg-16.4}
	Die Orthogonalprojektion hat folgende Eigenschaften:
	\begin{enumerate}[label=\alph*\upshape)]
		\item $\bild P_{U} = U$, $\kernn P_{U} = U^{\bot}$
		\item $\| P_{U} \| = 1$, denn $\| x \|^{2} = \| P_{U} x \|^{2} + \| x_{2} \|^{2} \geq \| P_{U} x \|^{2}$
		\item $P_{U} + P_{U^{\bot}} = Id_{X}$
	\end{enumerate}	
\end{folgerung}

\begin{definition}
	Sei $X$ ein Hilbertraum.
	\begin{itemize}
		\item Eine Folge $(h_{n}) \subseteq X$ hei{\ss}t \begriff{Orthogonalsystem}, falls $h_{n} \bot h_{m}$ für $m \neq n$.
		\item $(h_{n})$ hei{\ss}t \begriff{Orthonormalsystem}, falls zusätzlich $\| h_{n} \| = 1$ für alle $n$ gilt.
		\item Ein Orthonormalsystem $(h_{n}) \subseteq X$ hei{\ss}t \begriff{Orthonormalbasis} von $X$ falls $\overline{\ospan(h_{n})} = X$.
	\end{itemize}
\end{definition}


\begin{beispiel}
	Sei $X = \ell^{p}$, $h_{n} = e_{n} =$ Einheitsvektor, also $(h_{n}) = ( \delta_{n, m} )_{m \in \MdN}$. \\
	Weiter sei $x = (x_{n}) \in \ell^{p}$, mit $\< x , e_{n} \> = x_{n}$, also 
	\[ x = \sum_{n \in \MdN} x_{n} e_{n} = \sum_{n \in \MdN} \< x , e_{n} \> e_{n} \text{ und damit } \| x \|^{2} = \sum_{n \in \MdN} |x_{n}|^{2} = \sum_{n \in \MdN} | \< x , e_{n} \> |^{2}. \]
	Für $y = (\beta_{n}) \in \ell^{2}$ gilt: $ $ $\< x , y \> = \sum_{n} x_{n} \overline{\beta_{n}} = \sum \< x , e_{n} \> \overline{ \< y , e_{n} \> }$.
\end{beispiel}


\begin{satz} \label{satz:16.7}
	Sei $(h_{n})$ eine Orthonormalbasis. Für $U = \overline{\ospan(h_{n})}$ gilt dann
		\[ P_{U} x = \sum_{n} \< x, h_{n} \> h_{n} \quad \forall x \in X \]
	Weiter ist $\| P_{U} x \|^{2} = \sum_{n} | \< x , h_{n} \> |^{2} \leq \| x \|^{2}$ $\forall x \in X$ $ $ (Besselsche Ungleichung) \index{Besselsche Ungleichung}
\end{satz}


\begin{kor} \label{kor:16.8}
	Sei $(h_{n})$ eine Orthonormalbasis von X. Dann ist $x = \sum_{n} \< x, h_{n} \> h_{n}$, $\| x \|^{1} = \sum_{n} | \< x, h_{n} \> |^{2}$ (Parseval) \index{Parseval} und
		\[ \< x , y \> = \sum_{n} \< x, h_{n} \> \overline{\< y , h_{n} \>} \]
\end{kor}

\begin{beweis}
	Beweis des \hyperref[kor:16.8]{Korollars 16.8}: \\
	$U = \overline{\ospan(h_{n})} = X$, also $P_{U} = Id_{X}$. Für $x, y \in X: \< x , y \> = \< \sum_{n} \< x , h_{n} \> h_{n} , \sum_{m} \< y , h_{m} \> h_{m} \>$. \\ \\
	Beweis von \hyperref[satz:16.7]{Satz 16.7}:
	\begin{enumerate}[label=\alph*\upshape)]
		\item Sei $J$ eine endliche Indexmenge und $(h_{n})_{n \in J}$ gegeben. Für $U \coloneqq \overline{\ospan(h_{n})}$ nehme ein $x \in X$ und setze $y_{0} = \sum_{n \in J} \< x , h_{n} \> h_{n} \quad y_{1} \coloneqq x - y_{0}$. \\
		Für $j \in J$ gilt:
			\[ \< y_{1} , h_{j} \> = \< x , h_{j} \> - \sum_{n \in J} \< x , h_{n} \> \< h_{n} , h_{j} \> = 0 \]
			$\Rightarrow y_{1} \in U^{\bot} \Rightarrow x = y_{0} + y_{1}$ mit $y_{0} \in U, y_{1} \in U^{\bot}$
			\[ \Rightarrow P_{U} x = y_{0} = \sum_{j \in J} \< x , h_{j} \> h_{j} \]
			$\sum_{j \in J} | \< x , h_{j} \> |^{2} = \sum_{j \in J} \| \< x , h_{j} \> h_{j} \|^{2} \overset{Pyth.}{=} \| \sum_{j \in J} \< x , h_{j} \> h_{j} \|^{2} = \| P_{U} x \|^{2} \leq \| x \|^{2}$
		\item Sei $J = \MdN$. Definiere $J_{m} = \{ 1, \dotsc, m \}$, daraus folgt:
			\[ \sum_{j = 1}^{\infty} | \< x , h_{j} \> |^{2} = \sup_{m} \sum_{j = 1}^{m} | \< x , h_{j} \> |^{2} \leq \| x \|^{2} \quad (+) \label{eq:16.7.5.b-+} \]
			Sei $U_{m} = \overline{\ospan\{h_{1}, \dotsc, h_{m}\}}$, dann
			\[ \| P_{U_{n}} x - P_{U_{m}} x \|^{2} = \| \sum_{j = m + 1}^{n} \< x , h_{j} \> h_{j} \|^{2} = \sum_{j = m + 1}^{n} | \< x , h_{j} \> | ^{2} \rightarrow 0 \text{ für } n, m \rightarrow \infty \text{ wegen } \hyperref[eq:16.7.5.b-+]{(+)} \]
			Da $X$ vollständig ist: $\sum_{j = 1}^{\infty} \< x , h_{j} \> h_{j} = \lim_{n \rightarrow \infty} \sum_{j = 1 }^{n} \< x , h_{j} \> h_{j}$ existiert. \\
			Setze $y_{0} = \sum_{j = 1}^{\infty} \< x , h_{j} \> h_{j}$ und $y_{1} = x - y_{0}$. Dann ist wie oben $\< y_{1} , h_{j} \> = 0$ $\forall j \in \MdN$, also $y_{1} \in U^{\bot}$ $\Rightarrow x = y_{0} + y_{1}, P_{U} x = y_{0} = \sum_{j \in \MdN} \< x , h_{j} \> h_{j}$.
	\end{enumerate}
\end{beweis}


\begin{kor}
	Ein Orthonormalsystem $(h_{n})_{n \in J}$ ist genau dann eine Orthonormalbasis, wenn $\< x, h_{n} \> = 0$ für alle $n \in J$ bedeutet $x = 0$. 	
\end{kor}

\begin{beweis}
	Ist $ \< x , h_{n} \> = 0$ $\forall n \in J$ gilt aber:
		\[ x \in U^{\bot} = \{ 0 \} \text{ mit } U = \overline{\ospan(h_{n})}. \] 
	Und somit ist $U = \left( \{ h_{n} \}^{\bot} \right)^{\bot} = \{ 0 \}^{\bot} = X$.
\end{beweis}


\begin{beispiel}
	Sei $X = L^{2}(\Omega, \Sigma, P)$, wobei $(\Omega, \Sigma, P)$ ein W-Raum sei. Sei au{\ss}erdem $\Omega = \bigcup_{j = 1}^{n} U_{j}$ eine Partition von $\Omega$ mit $P(U_{j}) > 0$ und $U_{j} \cap U_{k} = \emptyset$. \\
	Definiere $h_{j} \coloneqq \frac{1}{P(U_{j})^{\frac{1}{2}}} \mathds{1}_{U_{j}}$, damit gilt:
	\[ \| h_{j} \|_{L^{2}} = 1 \quad \text{und} \quad h_{j} \bot h_{i}, \text{ da } U_{i} \cap U_{j} = \emptyset  \]
	$\Rightarrow \{ h_{j} \}$ ist ein Orthonormalsystem in $X$. Nach \hyperref[satz:16.7]{16.7}: $U = \overline{\ospan\{\mathds{1}_{U_{j}}: j = 1, \dotsc, n\}}$ \\
	\begin{align*}
		\Rightarrow \text{ } P_{U} x & = \sum_{j = 1}^{n} \left( \frac{1}{P(U_{j})^{\frac{1}{2}}} \int_{U_{j}} x(u) dP(u) \right) \left( \frac{1}{P(U_{j})	^{\frac{1}{2}}} \mathds{1}_{U_{j}} \right) \\
		& = \sum_{j = 1}^{n} \left( \frac{1}{P(U_{j})} \int_{U_{j}} x dP \right) \mathds{1}_{U_{j}} \\
		& = \mathds{E}[x | U_{1}, \dotsc, U_{n}] \text{ bedingte Erwartung.}
	\end{align*}
\end{beispiel}


\begin{beispiel}
	Sei $X = L^{2}[-\pi, \pi], \MdK = \MdC$. Die Fourierreihe $h_{n}(t) = \frac{1}{\sqrt{2 \pi}} e^{int}, t \in [-\pi , \pi]$ bildet eine Orthonormalbasis von $L^{2}[-\pi, \pi]$, denn 
	\[ \| h_{n} \|^{2} = \frac{1}{2\pi} \int_{\pi}^{\pi} e^{int} e^{\overline{int}} dt = \frac{1}{2\pi} 2 \pi = 1. \]
	Für $n \neq m$ gilt weiter
	\[ \< h_{n} , h_{m} \> = \frac{1}{2 \pi} \int_{- \pi}^{\pi} e^{int} e^{\overline{imt}} dt = \frac{1}{2 \pi} \int_{- \pi}^{\pi} e^{i(n - m)t} dt = \frac{1}{2 \pi} \frac{1}{i(n - m)} [ e^{i(n - m)t} ]_{-\pi}^{\pi} = 0 \]
	Dabei ist $\mathds{1}_{[0, a]}(t) = \sum_{k \in \MdZ} c_{k} e^{ikt}$ für $t \notin \{ 0, a \}$ ($\Rightarrow$ konv in $L^{2}$) wobei $c_{k} = \frac{1}{2 \pi} \int_{0}^{a} e^{-ikr} dt = \frac{1}{2 \pi k} \left( e^{-ika} - 1 \right)$ für $k \neq 0$ und $c_{0} = \frac{a}{2 \pi}$. \\
	Da Linearkombinationen von $\mathds{1}_{[0, a]}, a \in (0, \pi]$ dicht in $L^{2}[-\pi, \pi]$ liegen, folgt $\overline{\ospan\left(e^{in \cdot} \right)} = L^{2}[-\pi, \pi]$.
\end{beispiel}


\begin{beispiel}
	Sei $S = \{ \frac{1}{\sqrt{2 \pi}}, \frac{cos(nt)}{\sqrt{\pi}}, \frac{sin(nt)}{\sqrt{\pi}}; n \in \MdN \}$ ist eine Orthonormalbasis von $L^{2}[-\pi, \pi]$.	
\end{beispiel}

\begin{beweis}
	Zuerst benötigen wir folgende Identitäten
	\begin{align*}
		2 \cos(nx) \cos(mx) & = \cos((n + m) x) + \cos( (n - m) x) \\
		2 \sin(nx) \sin(mx) & = \cos((n - m) x) - \cos( (n + m) x) \\
		2 \cos(nx) \sin(mx) & = \sin((n + m) x) - \sin( (n - m) x) 
	\end{align*}
	$\Rightarrow$ Orthogonalität folgt durch Integration und Periodizität vom $\sin$.
	\[ \int_{-\pi}^{\pi} \cos^{2}(nx) dx + \int_{-\pi}^{\pi} \sin^{2}(nx) dx = \int_{-\pi}^{\pi} 1 dx = 2 \pi, \quad \int_{-\pi}^{\pi} \sin^{2}(nx) dx = \int_{-\pi}^{\pi} \cos^{2}(nx) dx \]
	$\Rightarrow S$ ist ein Orthonormalsystem. \\
	Die Vollständigkeit folgt wegen $e^{0} = 1, e^{inx} = \cos(nx) + i \sin(nx)$.
\end{beweis}


\[ 16.13 \text{ missing in my notes } \] \setcounter{satz}{13} % todo at 16.13: 16.13 missing in my notes


\begin{satz}
	Jeder separable, unendlich dimensionale Hilbertraum $X$ hat eine Orthonormalbasis $(h_{n})_{n \in \MdN}	$. \\
	Diese Orthonormalbasis definiert eine Isometrie
		\[ \phi \colon \ell^{2} \rightarrow X, \phi( (\alpha_{n}) ) = \sum_{n \in \MdN} \alpha_{n} h_{n}, \quad (\alpha_{n}) \in \ell^{2} \]
	mit 
	\begin{itemize}
		\item $\phi( e_{j} ) = h_{j}$
		\item $\| \phi( (\alpha_{j}) ) \|_{X} = \< (\alpha_{j}),(\beta_{j}) \>_{\ell^{2}} = \sum_{j \in \MdN} \alpha_{j} \overline{\beta_{j}}$
		\item $\phi^{-1} \colon X \rightarrow \ell^{2}, \phi^{-1}(x) = \left( \< x, h_{j} \>_{X} \right)_{j \in \MdN} \in \ell^{2}$
	\end{itemize}
\end{satz}

\begin{beweis}
	Es gibt eine Folge $(y_{j})_{j \in \MdN}$ in $X$ mit $\overline{\ospan\{ y_{j} \}} = X$ wobei $(y_{j})$ linear unabhängig sind. \\
	Auf $(y_{j})$ wende \hyperref[satz:13.6-Orthogonalzerlegung]{16.13} an und erhalte eine Orthonormalbasis $(h_{j})$ mit 
		\[ \overline{\ospan}(h_{j}) = \overline{\ospan(y_{j})} = X. \]
	Setze $\phi \left( (\alpha_{j}) \right) = \sum_{j \in \MdN} \alpha_{j} h_{j}$, damit gilt $\< \phi \left( (\alpha_{j}) \right) , h_{n} \> = \< \sum_{j} \alpha_{j} h_{j}, h_{n} \> = \alpha_{n}$ und weiter
	 \[ \| \phi \left( (\alpha_{j}) \right)  \|^{2}_{X} = \sum_{k \in \MdN} | \< \phi \left( (\alpha_{j}) \right), h_{k} \> |^{2} = \sum_{k \in \MdN} | \alpha_{k} |^{2} = \| (\alpha_{j}) \|_{\ell^{2}} \]
	$\phi$ ist somit eine Isometrie und ist surjektiv, da jedes $x$ die Form
	\[ x = \sum_{j} \< x, h_{j} \> h_{j} \text{ besitzt, also mit }\alpha_{j} = \< x, h_{j} \> \in \ell^{2}: \text{ } \phi(\alpha_{j} ) = \sum \alpha_{j} h_{j} = x. \]
\end{beweis}



\newpage