%!TEX root = Funktionalanalysis - Vorlesung.tex

\chapter*{Kompakte Operatoren} \addcontentsline{toc}{chapter}{Kompakte Operatoren} \setcounter{section}{1}



\section{Normierte R{\"a}ume}



\begin{definition}
	Sei $X$ ein Vektorraum über $\MdK \in\{ \MdR, \MdC \}$. \\
	Eine Abbildung  $\| \cdot \|: X \rightarrow \MdR_{+}$ hei{\ss}t eine \begriff{Norm}, wenn
	\begin{enumerate}
		\item $ \| x \| \geq 0, \| x \| = 0 \gdw x = 0 $
		\item $\ | \lambda x \| = | \lambda \| x \| $
		\item $ \| x + y \| \leq \| x \| + \| y \| $
	\end{enumerate}	
\end{definition}


\begin{bemerkung}
	Falls $ \| \cdot \| $ all die oben genannten Eigenschaften erfüllt au{\ss}er $ \| x \| = 0 \Rightarrow x = 0 $, dann hei{\ss}t $ \| \cdot \| $ Halbnorm.
\end{bemerkung}


\begin{vereinbarung}
Die Menge $ U_{X} = \{ x \in X:  \|x \| \leq 1 \}$ hei{\ss}t \begriff{Einheitskugel}. \\
Eine Folge $(x_{n})$ des normierten Raums $X$ \begriff{konvergiert} gegen ein $ x \in X $, falls  $\| x_{n} - x \| \xrightarrow[n \rightarrow \infty]{} 0. $	
\end{vereinbarung}


\begin{bemerkung}
Für zwei Elemente $x, y \in (X, \| \cdot \|)$ in normierten Räumen gilt auch die umgekehrte Dreiecksungleichung $( | \| x \| + \| y \| | \leq \| x + y \|)$
\end{bemerkung}


\begin{beispiel}
Sei $ X = \MdK^{n}, \hspace{0.25cm} x = ( x_{1}, \dotsc, x_{n}), \hspace{0.25cm} x_{i} \in \MdK$ 
\begin{align*}
	\| x \|_{p} & = \left( \sum_{j = 1}^{n} |x_{j}^{p}| \right)^{\frac{1}{p}}, \quad 1 \leq p < \infty \quad (p = 2: \text{ Euklidische Norm}) \\
	\| x \|_{\infty} & = \hspace{0.3cm} \sup_{j = 1}^{n} |x_{j}|	
\end{align*}

Beh.: $\| \cdot \| $ ist Norm auf $\MdK^n$ für $1 \leq p \leq \infty$

$\| x + y \|_{\infty} = sup_{j = 1}^{k} |x_{j} + y_{j}| \leq \|x\|_{\infty} + \|y\|_{\infty} $
Für $p \in (1, \infty), p \neq 2:$ siehe Übungsaufgabe (Fall $p = 2$ läuft über Cauchy-Schwarz)

Beachte: $\|x\|_{\infty} \leq  \|x\|_{p} \leq n^{\frac{1}{p}} \|x\|_{\infty} \leq n \| x \|_{\infty}$
\end{beispiel}


\begin{definition}
	Zwei Normen $\| \cdot \|_{1}, \| \cdot \|_{2}$ hei{\ss}en \begriff{äquivalent} auf $X$, falls es $0 < m, M < \infty$ gibt, so dass für alle $ x \in X$ gilt:
	\[ m \| x \|_{2} \leq \| x \|_{1} \leq M \| x \|_{2} \]
\end{definition}
 
 
\begin{satz} 
	Auf einem endlich dimensionalen Vektorraum sind alle Normen äquivalent.
\end{satz}

\begin{beweis}
	Wähle eine algebraische Basis $(e_{1}, \dotsc, e_{n})$ von X, wobei $ n = dim X < \infty$. \\
	Definiere $ \vertiii{x} = \left(\sum_{i = 1}^{n} |x_{i}|^2\right)^{\frac{1}{2}}$, wobei $x = \sum_{i = 1}^{n} x_{i} e_{i}$ \\
	
	z. z. die gegeben Norm $\vertiii{ \cdot }$ ist äquivalent zu $\| \cdot \|$. \\ \\
	Beweis: \\
	In der einen Richtung betrachte: 
	\begin{align*}
		\| x \| = \left\| \sum_{i = 1}^{n} x_{i} e_{i} \right\| & \leq \sum_{i = 1}^{n} |x_{i}| \|  e_{i} \| \\ 
													& \leq \underbrace{ \left( \sum_{i = 1}^{n} |x_{i}|^{2} \right)^\frac{1}{2} }_{=: \vertiii{ x }} \underbrace{ \left( \sum_{i = 1}^{n} \| e_{i} \|^{2} \right)^\frac{1}{2} }_{=: \nu}
	\end{align*}
	
	Für die Umkehrung benutze die Funktion $J: \MdK^{n} \rightarrow X, \hspace{0.1cm} J(x_{1}, \dotsc, x_{n}) = \sum_{i = 1}^{n} x_{i} e_{i} $ \\
	\begin{align*}
 	 \text{Die Abbildung } y & \in \MdK^{n}  \rightarrow \| Jy \| \text{ ist stetig, denn} \\
 	 	 \vertiii{ Jy } = \| y \|_{\MdK^{n}} & = \left( \sum_{i = 1}^{n} |y_{i}|^{2} \right)^{\frac{1}{2}}, y = (y_{1}, \dotsc, y_{n}) \\
 	 	 \text{und } \left| \| Jy \| - \| Jz \| \right| & \leq | \| Jy - Jz \| = | \| J(y - z) \| \\
 	 	 & \leq \nu \vertiii{ J (y - z) } \\
 	 	 & = M \| y - z \|_{\MdK^{n}} \\
 	 \end{align*}
 Daraus folgt die Stetigkeit von $ y \rightarrow \| Jy \| \in \MdR $ \\ \\
	Sei $S = \{y \in \MdK^{n}: \| y \|_{\MdK^n} = 1 \}$. Dann ist $S$ abgeschlossen und beschränkt.
	Die Abbildung $N : y \in S \rightarrow \| Jy \| > 0$ ist wie in (*) gezeigt stetig. Nach Analysis II nimmt $N$ sein Minimum in einem Punkt $y_{0} \in S$ an. Setze
		\begin{align*}
			m = \inf\{\| x \| : \vertiii{ x } = 1\} & = \inf \{ \| Jy \|: y \in S \} \\
												& = \| J y_{0} \| > 0 \\ \\
			\text{Also } m \leq \| \frac{x}{ \vertiii{ x } } \| =  \frac{ \| x \| }{ \vertiii{ x } }  &\Rightarrow \vertiii{ x } \leq m \| x \|.
		\end{align*}
\end{beweis}


\begin{prop}
	Für zwei Normen $\| \cdot \|_{1}, \| \cdot \|_{2}$ auf $X$ sind äquivalent:
	\begin{enumerate}[label=\alph*\upshape)]
		\item $\| \cdot \|_{1}, \| \cdot \|_{2}$ sind äquivalent
		\item Für alle $(x_{n}) \subset X$, $x \in X$ gilt $\| x_{n} - x \|_{1} \rightarrow 0 \gdw \| x_{n} - x \|_{2} \rightarrow 0 $
		\item Für alle $(x_{n}) \subset X$ gilt $\| x_{n} |_{1} \rightarrow 0 \gdw \| x_{n} \|_{2} \rightarrow 0 $
		\item Es gibt Konstanten $0 < m$, $M < \infty$, so dass $m U_{(X, \| \cdot \|_{1})} \leq U_{(X, \| \cdot \|_{2})} \leq M U_{(X, \| \cdot \|_{1})}$
	\end{enumerate}
	\begin{beweis}
		\begin{description}
			\item[] $a) \Rightarrow b) \Rightarrow c)$ folgt direkt durch die Definition von äquivalenten Normen. 
  			\item[] $c) \Rightarrow d)$ Annahme: Es existiert kein $M$ mit $U_{(X, \| \cdot \|_{2})} \subset M U_{(X, \| \cdot \|_{1})}$. \\
  				Dann gibt es eine Folge $x_{n} \in U_{(X, \| \cdot \|_{2})}$ mit $\| x_{n} \|_{1} \geq n^{2}$ \\
  				Setze $y_{n} =  \frac{1}{n} x_{n}$. Dann gilt $\| y_{n} \|_{1} \rightarrow 0$ und $\| y_{n} \|_{2} \rightarrow \infty$. \\
  				Widerspruch zu $c)$.
  			 \item[] $d) \Rightarrow a)$ Gegeben ist $U_{(X, \| \cdot \|_{2})} \subset M U_{(X, \| \cdot \|_{1})}$ \\
  			 Das ist äquivalent zu $\| x \|_{2} \leq M \| x \|_{1}$ \\
  			 Analog folgt aus $m U_{(X, \| \cdot \|_{1})} \subset U_{(X, \| \cdot \|_{2})}$ dann $m \| x \|_{1} \leq \| x \|_{2}$. \\
  			 Also $m \| x \|_{1} \leq \| x \|_{2} \leq M \| x \|_{1}$ 
		\end{description}
	\end{beweis}
\end{prop}


\begin{vereinbarung}
Sei $\MdF = \{ (x_{n}) \in \MdK^{\MdN}: x_{i} = 0 \text{ bis auf endlich viele } n \in \MdN \} $ der \begriff{Folgenraum} \\
und $e_{j} = (0, \dotsc, 0, 1, 0, \dotsc, 0) $ der Einheitsvektor, wobei die $1$ an j-ter Stelle steht.
\end{vereinbarung}


\begin{beispiel} \index{l$^{p}$-Raum} \index{l$^{\infty}$-Raum} \index{c$_{0}$-Raum}
	\begin{itemize}
		\item $\ell^{p} = \{ x = (x_{n}) \in \MdK^{\MdN}: \|x\|_{p} = \left( \sum_{n = 1}^{\infty} | x_{n} |^{p}\right)^{\frac{1}{p}} < \infty \}$
		\item $\ell^{\infty} = \{ x = (x_{n}) \in \MdK^{\MdN  }: \|x\|_{\infty} = \sup_{n \in \MdN} |x_{n}| < \infty \}$
		\item $c_{0} = \{ x = (x_{n}) \in \ell^{\infty}: \lim_{n \rightarrow \infty} |x_{n}| = 0 \}$
	\end{itemize}
Gültigkeit der Dreiecksungleichung beweist man ähnlich wie bei $(\MdK^{n}, \| \cdot \|_{p})$.
\end{beispiel}


\begin{lemma}
\begriff{Minkowskii-Ungleichung}: $\left( \sum_{i=1}^{\infty} |x_{i} + y_{i}|^p\right)^{\frac{1}{p}} \leq\left( \sum_{i=1}^{\infty} |x_{i}|^p\right)^{\frac{1}{p}} \left( \sum_{i=1}^{\infty} |y_{i}|^p\right)^{\frac{1}{p}} $	\\
\begriff{Hölder-Ungleichung}: mit $\frac{1}{p} + \frac{1}{p'} = 1 \text{ gilt } \sum_{i=1}^{\infty} |x_{i}| |y_{i}| \leq \left( \sum_{i=1}^{\infty} |x_{i}|^{p} \right)^{\frac{1}{p}} \left( \sum_{i=1}^{\infty} |y_{i}|^{p'} \right)^{\frac{1}{p'}} $	\\	
\end{lemma}


\begin{bemerkung}
	Im unendlich dimensionalen Fall sind die Normen $\| \cdot \|_{p}$ auf $\MdF$ nicht äquivalent.
	\begin{beweis}
		Sei $p > q$, setze 
		\[ X_{n} := \sum_{j = 2^{n} + 1}^{2^{n + 1}} j^{-\frac{1}{p}}e_{j}, \hspace{0.5cm} e_{j} = ( \delta_{ij} )_{u \in \MdN} \]
		Damit gilt $x_{n} \in \MdF$ und weiter
		\[ \| x_{n} \|_{p} = \left( \sum_{j = 2^{n}}^{2^{n + 1}} \frac{1}{j} \right)^{\frac{1}{p}} \simeq \left( ln(2) \right)^{\frac{1}{p}} \]
		aber $\| x_{n} \|_{q} \rightarrow \infty$, also sind $\| \cdot \|_{p}, \| \cdot \|_{q}$ keine äquivalente Normen.
	\end{beweis}	
\end{bemerkung}


\begin{beispiel}
	\begin{enumerate}[label=\alph*\upshape)]
		\item Raum der stetigen Funktionen \index{Raum der stetigen Funktionen} \\
		$\Omega \subset \MdR^{n}, \hspace{0.25cm} C(\Omega) = \{ f: \Omega \rightarrow \MdR \text{ | } f \text{ stetig} \}$, \hspace{0.25cm} $\| f \|_{\infty} = \sup_{u \in \Omega} |f(u)|$\\
		$\Rightarrow \| f - f_{n} \|_{\infty} \rightarrow 0$ bedeutet gleichmä{\ss}ige Konvergenz von $f_{n}$ gegen $f$ auf $\Omega$.
		\item Raum der differenzierbaren Funktionen \index{Raum der differenzierbaren Funktionen} \\
		$\Omega \subset \MdR^{n}$ offen, $f: \Omega \rightarrow \MdR, \hspace{0.25cm} \alpha = (\alpha_{1}, \dotsc, \alpha_{m}) \in \MdN_{0}^{m}$ \\
		$D^{\alpha}f(x) = \frac{ \delta^{ | \alpha | } }{ \delta x_{1}^{ \alpha_{1} } \dotsc \delta x_{n}^{ \alpha_{n} } } f(x), \text{ wobei } | \alpha | = \alpha_{1} + \dotsc + \alpha_{n} $ \\ 
	\end{enumerate}
\end{beispiel}


\begin{definition}
Wir nennen $C_{b}^{m}(\Omega) = \{ f: \Omega \rightarrow \MdR | D^{\alpha}f \text{ sind stetig in } \Omega, \text{ beschränkt auf } \Omega \text{ für alle } \alpha \in \MdN^{n}, |\alpha| \leq m \}$ den \begriff{Raum der beschränkten, m-fach stetig differenzierbaren Funktionen}.  \\
Auf $C_{b}^{m}$ definieren wir die Norm: $\| f \|_{C_{b}^{m}} = \sum_{|\alpha| \leq m} \| D^{\alpha}f \|_{\infty}$
\end{definition}


\begin{bemerkung}
	Auf $C_{b}^{m} [0, 1]$ ist eine äquivalente Norm zu  $\| f \|_{C_{b}^{m}}$ gegeben durch
	\begin{align*}
		\| f \|_{0} & = \sum_{i = 0}^{m - 1} |f^{(i)}(0)| + \| f^{(m)} \|_{\infty} \\
		\text{Denn } f^{(i)}(t) = f^{(i)}(0) + \int_{0}^{t} & f^{(i + 1)}(s) ds \text{ und damit } \| f^{(i)}\|_{\infty} \leq | f^{(i)}(0) | + \| f^{(i + 1)}\|_{\infty}	
	\end{align*}
\end{bemerkung}


\begin{beispiel}
$X = C(\bar\Omega), \Omega \subset \MdR^{n}$ offen, beschränkt. \\
Definiere $\| f \|_{L^{p}} = \left( \int_{\Omega} |f(u)|^{p} du \right)^{\frac{1}{p}}$ und betrachte $f_{k}(t) = t^k, t \in [0, 1]$, dann gilt:
 \[ \| f \|_{L^{p}} = \left( \frac{1}{kp + 1} \right)^{\frac{1}{p}} \xrightarrow[k \rightarrow \infty]{} 0, \hspace{0.25cm} p < \infty \]
\end{beispiel}


\begin{definition}[\begriff{Quotientenräume}] \label{def:2.15-Quotientenraeume}
Sei $(X, \| \cdot \|)$ ein normierter Raum. $M \subset X$ sei abgeschlossener, linearer Unterraum. \[ (\text{abgeschlossen: d.h. für alle } (x_{n}) \in M, \| x_{n} - x \| \rightarrow 0 \Rightarrow x \in M) \]

Definiere $\hat X = X / M, \hspace{0.25cm} \hat x \in X/M: \hspace{0.25cm} \hat x = \{ y \in X: y - x \in M \} = x + M$ \\
Dabei gilt unter anderem $\hat x_{1} + \hat x_{2} = \widehat{x_{1} + x_{2}}$ und $\lambda \hat x_{1} = \widehat{\lambda x_{1}}$; $\hat X$ bildet somit einen Vektorraum. \\
Definieren wir eine Norm für die Äquivalenzklassen mittels $\| \hat x \|_{\hat X} : = \inf \{ \| x - y \|_{X}: y \in M \} =: d(x, Y)$

Behauptung: $(\hat X, \| \cdot \|_{\hat X})$ ein normierter Raum. \\
Beweis: Sei $\hat x \in \hat X$ beliebig mit $\| \hat x \|_{\hat X} = 0$ \\
 dann existiert ein $y_{n} \in \hat X \text{ mit } \| y_{n} \| \rightarrow 0$ und $x - y_{n} \in M$
	\[ \Rightarrow x \in M, \hat x = 0 \]
Zu $\epsilon > 0$ wähle für $\hat x_{1}, \hat x_{2} \in \hat X, y_{1}, y_{2} \in M$ mit
	\[ \| \hat x_{i} \| \geq \| x_{i} - y_{i} \| - \epsilon \] 
	Damit folgt:
	\begin{align*}
		\| \widehat{x + y} \| & \leq \| x_{1} + x_{2} - y_{1} - y_{2} \| \\
						  & \leq \| x_{1} - y_{1} \| + \| x_{2} - y_{2} \| \\
						  & \leq \| \hat x_{1} \| + \| \hat x_{2} \| + 2 \epsilon
	\end{align*}
\end{definition}


\begin{bemerkung}
	Ist $\| \cdot \|$ nur eine Halbnorm auf $X$, so ist $M = \{ x: \| x \| = 0 \}$ ein abgeschlossener, linearer Teilraum von $X$ und der Quotientenraum $(\hat X, \| \cdot \|_{\hat X})$ ist ein normierter Raum.
\end{bemerkung}


\begin{beispiel}
	\begin{itemize}
	\item Hölderstetige Funktionen \index{Hölderstetige Funktionen} \\
	Wenn $h_{\alpha}(f) = \sup_{u,v \in \MdR, u \neq v} \frac{\| f(u) - f(v)\|}{|u - v|^{\alpha}} < \infty \hspace{0.25cm} \left( \alpha \in (0, 1] \right)$, dann nennt man $f$ hölderstetig.
	\[ C^{\alpha}(\Omega) := \{ f: \Omega \rightarrow \MdR: h_{\alpha}(f) < \infty \} \hspace{0.5cm} \Omega \subset \MdR^{n}, \hspace{0.25cm} \]
	Im Moment ist $h_{\alpha}( \cdot )$ eine Halbnorm. Unter der Voraussetzung $\Omega$ zusammenhängend gilt aber weiter:
	\[ h_{\alpha}(f) = 0 \gdw f \equiv c \text{ konstant} \]
	Wenn z.B. $M = \{ \mathds{1} \Omega \}$ und $V = C^{\alpha}/M$ ist oben genanntes sogar ein normierter Raum.
	\item Lebesgues-Integrierbare Funktionen \index{Lebesgue-Integrierbare Funktionen} \\
	Sei $\Omega \subset \MdR^{n}$ offen, $\mathcal{L}^{p}(\Omega) = \{ f: \Omega \rightarrow \MdR : |f|^{p}$ ist Lesbesgue-integrierbar auf $\Omega$  \}.
	Wir definieren $\| f \|_{p} := \left( \int_{\Omega} |f(x)|^{p} d\mu \right)^{\frac{1}{p}}$, wobei $\| \cdot \|_{p}$ hier eine Halbnorm bildet.
	\[ \| f \|_{p} = 0 \gdw  f(x) = 0 \text{ fast überall auf } \Omega \]
	Wähle $M = \{ f: \Omega \rightarrow \MdR: f = 0 \text{ fast überall auf } \Omega \}$. \\ \\
	Dann ist 
	\[ L^{p}(\Omega) := \QR{ \mathcal{L}^{p}(\Omega) }{ M } \text{ ein normierter Raum.} \]
	\end{itemize}
\end{beispiel}



\newpage
































