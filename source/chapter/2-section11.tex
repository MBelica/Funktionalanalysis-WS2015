%!TEX root = Funktionalanalysis - Vorlesung.tex

\section{Projektionen}



\begin{definition} \label{def:11.1-Projektion}
	Sei $X$ ein Banachraum. $P \colon X \rightarrow X$ hei{\ss}t \begriff{Projektion}, wenn $P$ linear und $P^{2} = P$ ist.
\end{definition}

Die Frage die wir uns dabei stellen sollten, lautet: wann ist $P$ beschränkt?

\begin{beispiel}
	\begin{enumerate}[label=\alph*\upshape)]
		\item $X = L^{p}(\MdR)$, $\Omega \subset \MdR: $ $\mu(\Omega) > 0$, $\mu(\MdR \setminus \Omega) > 0$
			\[ P f \coloneqq \1_{\Omega} f ~ \Rightarrow ~ P^{2} f = \1_{\Omega}^{2} f = \1_{\Omega} f = P f \]
			$\| P f \|_{L^{p}} = \left( \int | \1_{\Omega} f |^{p} d\mu \right)^{\frac{1}{p}} = \left( \int_{\Omega} | f |^{p} d\mu \right)^{\frac{1}{p}} \leq \| f \|_{L^{p}}$ $\Rightarrow P$ ist beschränkt.
		\item $X = L^{p}[0, 1]^{2},$ mit $1 \leq p < \infty$
			\[ P f(x, y) \coloneqq \int_{0}^{1} f(s, y) ds ~ \Rightarrow ~ P^{2} f (x, y) = \int_{0}^{1} \int_{0}^{1} f(s, y) ds dt = \int_{0}^{1} f(s, y) ds = P f(x, y) \]
			\begin{align*}
				\| P f \|_{L^{p}} & = \left( \int_{0}^{1} \int_{0}^{1} \left| \int_{0}^{1} f(s, y) ds \right|^{p} dy dx \right)^{\frac{1}{p}} \\
				& \overset{\triangle-\text{Ungl.}}{\leq} \left( \int_{0}^{1} \left( \int_{0}^{1} |f(s, y)| ds \right)^{p} dy \right)^{\frac{1}{p}} \\
				& \underset{()^{p} \text{ konvex!}}{\overset{\text{Jensen}}{\leq}}\left( \int_{0}^{1} \int_{0}^{1} |f(s, y)|^{p} ds dy \right)^{\frac{1}{p}} \\
				& = \| f \|_{L^{p}}
			\end{align*}
	\end{enumerate}
\end{beispiel}


\begin{bemerkung}
	Sei $X$ ein Vektorraum, $M \subset X$ ein Untervektorraum. Es gibt nach dem Basisergänzungssatz eine lineare Projektion $P \colon X \rightarrow X,$ $P(X) = M$	
\end{bemerkung}

\begin{beweis}
	Sei $(b_{i})_{i \in I}$ eine Basis von $M$ und $(b_{j})_{j \in J}$ von $X$, mit $I \subset J$. \\
	Nun existiert ein $(\alpha_{j}(x))_{j}$:
	\[ x = \sum_{j \in J} \alpha_{j}(x) b_{j} \text{ mit } \alpha_{j}(x) = 0 \text{ für fast alle } j \in J. \]
	$P(x) \coloneqq \sum_{i \in I} \alpha_{i}(x) b_{i}$.
\end{beweis}


\begin{erinnerung}
	Sind $X, Y$ Banachräume, dann ist auch $X \oplus Y$ ein Banachraum mit $\| (x, y) \|_{X \bigoplus Y} = \| x \|_{X} + \| y \|_{Y}$ $\forall x \in X, y \in Y$
\end{erinnerung}

\newpage % todo temporarily for optics

\begin{satz} \label{satz:11.4}
	Sei $X$ ein Banachraum, $M \subset X$ ein abgeschlossener Untervektorraum. Dann sind folgende Aussagen äquivalent:
	\begin{enumerate}[label=\alph*\upshape)]
		\item Es gibt eine stetige Projektion $P \colon X \rightarrow X$ mit $P(X) = M$
		\item Es gibt einen abgeschlossenen Untervektorraum $N \subset X: X = M \oplus N$.
		\item Es gibt einen abgeschlossenen Untervekottraum $N \subset X$ und $J: M \oplus N \rightarrow X,$ $J(x, y) = x + y$ ist ein Isomorphismus, insbesondere $\exists c > 0$ $\forall x \in M, y \in N:$ $c \left( \|x \| + \|y \| \right) \leq \|x + y \| \leq \|x \| + \| y \| $
	\end{enumerate}
\end{satz}

\begin{beweis}
	$a) \Rightarrow b)$ Definiere $N = (I - P)(X)$. Dann gilt
		\[ X = P(X) + (I - P)(X) \text{ und } P(X) \cap (I - P)(X) = \{ 0 \} \]
		$N$ ist abgeschlossen, denn $N = \kernn{P} = P^{-1}\{ 0 \}$ und $P$ ist stetig.	\\ \\
	$b) \Rightarrow c)$ Sei $N$ wie in $b)$. $J: M \oplus N \rightarrow X,$ $J(x, y) = x + y$. Dann gilt:
		\begin{align*}
			\| x + y \| & = \| J(x, y) \| \quad \forall x \in M, y \in N \\
						& \leq \| x \| + \| y \| \\
						& = \| (x, y) \|_{M \oplus N}
		\end{align*}
		Au{\ss}erdem ist $J$ bijektiv, da $X = M \oplus N$
		$\xRightarrow[]{\ref{kor:10.5}} J^{-1} stetig,$ $\exists \hat{c} > 0: \forall x \in M, y \in N$:
		\[ \| x + y \| - \| (x, y) \| = \| J^{-1}(x + y) \| \leq \hat{c} \| x + y \| \]

	$c) \Rightarrow a)$ Definiere $\hat{P} \colon M \oplus N \rightarrow M \oplus N,$ $\hat{P}(x, y) = (x, 0)$
		\[ \Rightarrow \hat{P}(M \oplus N) = M \oplus \{ 0 \}. \text{ Setze } P = J \hat{P} J^{-1} \quad (P \text{ ist Projektion!}) \]
		$P(X) = M$, denn $P(X) = J \hat{P}(M \oplus N) = J (M \oplus \{ 0 \}) = M$
\end{beweis}


\begin{vereinbarung}
	$M$ hei{\ss}t komplementierter Raum, $N = \kernn(P)$ Komplementärraum.
\end{vereinbarung}


\begin{beispiel*}
	\begin{enumerate}[label=\alph*\upshape)]
		\item $X = L^{p}(\MdR)$ $\Omega \subseteq \MdR,$ $P f = \1_{\Omega} f$
			\begin{align*}
				M & = P(X) = \{ f \in L^{p}(\MdR): f = 0 \text{ fast überall auf } \Omega^{c} \} \\
				N & = \kernn(f) = \{ f \in L^{p}(\MdR): f = 0 \text{ fast überall auf } \Omega \}
			\end{align*} 
		\item $X = L^{p}([0, 1]^{2})$ $P f(x, y) = \int_{0}^{1} f(s, y) ds$
			\begin{align*}
				M & = \{ f \in L^{p}([0, 1]^{2}), f \text{ konstant in 1. Komponente} \} \\
				N & = \{ f \in L^{p}([0, 1]^{2}): \int_{0}^{1} f(s, y) ds = 0 \text{ fast überall, } y \in [0, 1] \}
			\end{align*}
	\end{enumerate}
	Sowohl in $a)$ als auch in $b)$ gilt $X = M \oplus N$, da $P$ in beiden Fällen stetig ist.
\end{beispiel*}



\newpage