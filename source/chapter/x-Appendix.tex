%!TEX root = Funktionalanalysis - Vorlesung.tex


\chapter*{Anhang} \addcontentsline{toc}{chapter}{Anhang} 

So lange dieses Skript in Arbeit ist, werde ich hier Definitionen und Sätze sammeln, die über das Skript hinaus genutzt werden, aber (noch) nicht eingeführt wurden.


\begin{satznbfr}[Hahn-Banach] \index{Hahn-Banach} label{satz:x-1-hahn-banach}
	Es sei $X$ ein Vektorraum über $\MdK \in \{ \MdR, \MdC \}$. \\
	Seien nun
	\begin{itemize}
		\item $Y \subseteq X$ ein linearer Unterraum;
		\item $p: X \rightarrow \MdR$ eine sublineare Abbildung;
		\item $f: Y \rightarrow \MdK$ ein lineares Funktional, für das $\Re(f(y)) \leq p(y)$ für alle $y \in Y$ gilt.
	\end{itemize}
	Dann gibt es ein lineares Funktional $F: X \rightarrow \MdK$, so dass
	\[ F|_{Y} = f \quad \text{und} \quad \Re(F(x)) \leq p(x) \quad \text{für alle } x \in X \text{ gilt.} \]
\end{satznbfr}



\newpage