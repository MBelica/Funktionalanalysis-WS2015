%!TEX root = Funktionalanalysis - Vorlesung.tex



\section{Duale Operatoren auf Hilberträumen}



Für $X = \MdC^{n}$ mit euklidischem Skalarprodukt: \\
$A \in B(X) \colon \< A x , y \> = \< x , A^{*} y \>, \quad x, y \in X$, wobei
\[ A = (a_{ij})_{i, j = 1, \dotsc, n}, \quad A^{*} = (\overline{a_{ji}})_{i, j = 1, \dotsc, n} \]


\begin{satz}
	Seien $X, Y$ Hilberträume und $T \in B(X, Y)$. Dann gibt es genau ein $T^{*} \in B(Y, X)$ mit
	\begin{itemize}
		\item $\< T x , y \> = \< x , T^{*} y \> \quad \forall x \in X, y \in Y$,
		\item $\| T^{*} \| = \| T \|$,
		\item $\left(T	^{*}\right)^{*} = T$.
	\end{itemize}	
\end{satz}

\begin{beweis}
	Für $y \in Y$ fest, setze $l_{y}(x) \coloneqq \< T x , y \>$ für alle $x \in X \Rightarrow l_{y} \in X'$, denn
	\[ | l_{y}(x) | \leq \| T x \| \cdot \| y \| \leq \|T \| \cdot \| x \| \cdot \| y \|, \text{ also } \| l_{y} \| \leq \| T \| \cdot \| y \| \]
	Nach dem \hyperref[lemma:6.3-Riesz]{Lemma von Riesz} gibt es ein $z \in X$ mit
	\[ l_{y} (x) = \< x , z \> \quad \text{und} \quad \| z \|_{X} = \| l_{y} \|_{X'} \]
	Definiere $T^{*} y \coloneqq z$, dann gilt: $T^{*} \colon Y \rightarrow X$ ist linear und $\< x , T^{*} y \> = l_{y}(x) = \< T x , y \>$. \\
	Weiter ist 
	\[ \| T^{*} y \| = \| z \| = \| l_{y} \| \leq \| T \| \cdot \| y \| \quad \Rightarrow \quad \| T^{*} \| \leq \| T \|. \]
	Nach Definition ist 
	\[ \left( T^{*} \right)^{*} = T \quad \Rightarrow \quad \| T \| = \| \left( T^{*} \right)^{*} \| \leq \| T^{*} \| \]
	$\Rightarrow \| T^{*} \| = \| T \|$. 
\end{beweis}


\begin{bemerkung}[Eigenschaften der Adjungierten] \label{bem:19.2}
	Sei $S, T \in B(X), \lambda \in \MdK$
	\begin{enumerate}[label=\alph*\upshape)]
		\item $\left( S + T \right)^{*} = T^{*} + S^{*}$
		\item $\left( \lambda S \right)^{*} = \overline{\lambda} S^{*}$
		\item $\left( T \cdot S \right)^{*} = S^{*} T^{*}$
		\item $\| S \cdot S^{*} \| = \| S \|^{2} = \| S^{*} \cdot S \|$
	\end{enumerate}
\end{bemerkung}

\begin{beweis}
	\begin{enumerate}[label=\alph*\upshape)]
		\setcounter{enumi}{3}
		\item $\| S^{*} \|  \leq \| S^{*} \| \cdot \| S \| = \| S \|^{2}$
			\begin{align*}
				\| S \|^{2} & = \sup_{\| x \| \leq 1} \| S x \|^{2} = \sup_{\| x \| \leq 1}  \< S x , S x \> \\
					& = \sup_{\| x \| \leq 1} \< x , S^{*} S x \> \leq \sup_{\| x \| \leq 1}  \underbrace{\| x \|}_{\leq 1} \cdot \| S^{*} S x \| = \| S^{*} S \|
			\end{align*}
			$T = S^{*} \colon \| S S^{*} \| = \| T^{*} T \| = \| T \|^{2} = \| S^{*} \|^{2} = \| S \|^{2}$
	\end{enumerate}
\end{beweis}


\begin{kor}
	Für $S \in B(X)$ gilt:
	\[ \kernn(S) = \left( \bild(S^{*}) \right)^{\bot}, \quad \kernn(S^{*}) = \left( \bild(S) \right)^{\bot} \]	
\end{kor}

\begin{beweis}
	Zuerst zeigen wir $\kernn(S) = \left( \bild(S^{*}) \right)^{\bot}$:
	\begin{align*}
		x \in \kernn S & \gdw S x = 0 \gdw \< S x , y \> = 0  ~ \forall y \in X \\
			& \gdw \< x , S^{*} y \> = 0 ~\forall y \in X \gdw x \in \left( \bild S^{*} \right)
	\end{align*}
	Die zweite Aussage zeigt man analog.
\end{beweis}


\begin{beispiel}
	Integraloperator: Sei $k \in L^{2}( \Omega \times \Omega), (T_{k}x)(u) = \int_{\Omega} k(u, v) x(v) dv$, $T_{k} \in B(L^{2}(\Omega))$
	\begin{align*}
		\Rightarrow \< T_{k} x , y \> & = \int_{\Omega} \left( \int_{\Omega} k(u, v) x(v) dv \right) \overline{y(u)} du \\
			& \overset{Fubini}{=} \int_{\Omega} x(v) \left( \int_{\Omega} k(u, v)\overline{y(u)} du \right) dv \\
			& \int_{\Omega} x(v) \left( \overline{ \int_{\Omega} \overline{k(u, v)} y(u) du} \right) dv \\
			& = \< x , T_{k^{*}} y \>, \text{ wobei } T_{k^{*}} y(u) = \int_{\Omega} \overline{k(v, u)} y(v) dv
	\end{align*}
	Es gilt also: $\left( T_{k} \right)^{*} = T_{k^{*}}$, $k^{*}(u, v) = \overline{k(v, u)}$ $ $ (vgl. $A = (a_{ij}), A^{*} = \overline{(A^{T})}$)
\end{beispiel}


\begin{definition}
	Sei $T \in B(X, Y)$, $X, Y$ Hilberträume
	\begin{enumerate}[label=\alph*\upshape)]
		\item $T$ hei{\ss}t \begriff{unitär}, falls $T$ invertierbar ist und $T^{-1} = T^{*}$
			\[ \text{d.h. } T \text{ ist surjektiv und } \< Tx , Ty \> = \< x , T^{*} T y \> = \< x , y \> \quad \forall x, y \in X \]
		\item Sei $X = Y$. $T$ ist \begriff{selbstadjungiert}, falls $T^{*} = T$, d.h. $\< Tx , y \> = \< x , T y \> \quad \forall x, y \in X$
		\item Sei $X = Y$. $T \in B(X)$ hei{\ss}t \begriff{normal}, falls $T^{*} T = T T^{*}$.
			\[ \text{d.h. } \< Tx , Ty \> = \< T^{*} x , T^{*} y \> \quad \forall x, y \in X \]
	\end{enumerate}
\end{definition}


\begin{bemerkung*}
	Unitäre und selbstadjungierte Operatoren sind normal.	
\end{bemerkung*}


\begin{beispiel}
	Seien $X, Y$ Hilberträume und $T \in B(X, Y)$.
	\begin{enumerate}[label=\alph*\upshape)]
		\item Integraloperator \\
			$T_{k}$	ist selbstadjungiert $\gdw T_{k^{*}} = T_{k} \gdw \overline{k(v, u)} = k(u, v) \gdw k^{*} = k$. \\
			Falls $k \in L^{2}\left([0, 1]^{2}\right)$, so ist $T_{k}$ kompakt und damit nicht unitär, da $T_{k}$ keine Isometrie sein kann ($\dim = \infty$).
		\item Verschiebungsoperator auf $X = \ell^{2}$ \\
			$T(\alpha_{1}, \alpha_{2}, \dotsc ) = (\alpha_{2}, \alpha_{3}, \dotsc )$, $(\alpha_{j}) \in \ell^{2} \Rightarrow T^{*}(\alpha_{1}, \alpha_{2}, \dotsc ) = (0, \alpha_{1}, \alpha_{2}, \dotsc )$ \\
			\[ \text{somit ist zwar } T T^{*} = Id, \text { aber } T^{*} T (\alpha_{1}, \alpha_{2}, \dotsc ) = (0, \alpha_{2}, \alpha_{3}, \dotsc ) \]
			$\Rightarrow T* T = P_{U}$ mit $U = \{ (\alpha_{j}) \in \ell^{2} \colon \alpha_{1} = 0 \}$.
		\item Sei $U \subset X$ ein Teilraum und $P_{U} \colon X \rightarrow U$ die Orthogonalprojektion auf $U$. \\
			$P_{U}$ ist selbstadjungiert.
			\[ \text{Für } x, y \in X ~ ~ \exists ~ x_{1}, y_{1} \in U, ~ x_{2}, y_{2} \in U^{\bot}: x = x_{1} + x_{2} \text{ und } y = y_{1} + y_{2} \]
			$\Rightarrow \< P x , y \> = \< x , P y \>,$ $\forall x,y \in X$.
	\end{enumerate}
\end{beispiel}



\textbf{Ziel in diesem Abschnitt:} \\
	Ist $T \in B(X)$ normal und kompakt, so existiert eine Folge $\lambda_{n} \in \MdC$ und ein Orthonormalsystem $(h_{n})$ so, dass
	\[ T x = \sum_{n \in \MdN} \lambda_{n} \< x , h_{n} \> h_{n} \quad \forall x \in X \]	
	(vgl. Diagonalisierung von selbstadjungierten Matrizen)



\begin{satz}
	Sei $X$ ein Hilbertraum über $\MdC$. \\
		\[ T \in B(X) \text{ ist selbstadjungiert } \gdw \< T x , x \> \in \MdR \text{ für alle } x \in X \]
\end{satz}

\begin{beweis}
	$" \Rightarrow "$: $\< T x , x \> = \< x , T x \> = \overline{\< T x , x \>} \in \MdR$ \\ \\
	$" \Leftarrow "$: $\lambda \in \MdC$, $x, y \in X$ \\
	\[ \< T ( x + \lambda y ) , x + \lambda y \> =  \< T x , x \> + \overline{\lambda} \< T y , x \> + \lambda \< y, T x \> + | \lambda |^{2} \< T y , T y \> \quad (1) \label{eq:19.7.5-1} \]
	Konjugieren dieser Gleichung liefert:
	\[ \< T ( x + \lambda y ) , x + \lambda y \> =  \< T x , x \> + \lambda \< x , T y \> + \overline{\lambda} \< T x , y \> + | \lambda |^{2} \< T y , T y \> \quad (2) \label{eq:19.7.5-2} \]
	Bilde $\hyperref[eq:19.7.5-1]{(1)} - \hyperref[eq:19.7.5-2]{(2)}$ für $\lambda = 1$:
	\[  \< T x , y \> + \< T y , x \> = \< y , T x \> + \< x , T y \> \quad (3) \label{eq:19.7.5-3} \]
	Bilde $\hyperref[eq:19.7.5-1]{(1)} - \hyperref[eq:19.7.5-2]{(2)}$ für $\lambda = i$:
	\[ \< T x , y \> - \< T y , x \> = - \< y , T x \> + \< x , T y \> \quad (4) \label{eq:19.7.5-4} \]
	und $\hyperref[eq:19.7.5-3]{(3)} + \hyperref[eq:19.7.5-4]{(4)}$ liefert dann:
	\[ \< T x , y \> = \< x , T y \> \]
\end{beweis}


\begin{prop} \label{prop:19.8}
	Für $T \in B(X)$ selbstadjungiert, gilt:
	\[ \| T \| = \sup_{\| x \| \leq 1} |\< T x , x \>| \]	
\end{prop}

\begin{beweis}
	$" \geq "$: $| \< T x , x \> | \leq \| T x \| \cdot \|x \| \leq \| T x \| \leq \| T \|$, falls $\| x \| \leq 1$. \\ \\
	$" \leq "$: Definiere $M \coloneqq \sup_{\| x \| \leq 1} \left| \< T x , x \> \right|$. Zu zeigen ist damit $\| T \| \leq M$: \\ \\
	Nach $T^{*} = T$ und der Definition von $M$ gilt:
	\begin{align*}
		\< T ( x + y ) , ( x + y ) \> - \< T ( x - y ) , ( x - y) \> & = 2 \< T x , y \> + 2 \underbrace{\< T y , x \>}_{= \< y , T x \>} \\
			& = 2 \left( \< T x , y \> + \overline{\< x , T y \>} \right) \\
			& = 4 \Re \left( \< T x , y \> \right)
	\end{align*}
	Also mittels der Parallelogrammgleichung 
	\begin{align*}
		| 4 \cdot \Re \< T x , y \> | & \leq | \< T ( x + y ) ,  x + y \> | + | \< T ( x - y ) , x - y \> | \\
			& \leq M \| x + y \|^{2} + M \| x - y \|^{2} \\
			& = M \left( 2 \| x \|^{2} + 2 \| y \|^{2} \right)
	\end{align*}
	Für $\| x \|, \| y \| \leq 1$ ist 
		\[ | 4 \Re \< T x , y \> | \leq 4 M \]
	Wähle $\theta \in (-\pi , \pi]$, für $\| x \| \leq 1, \| e^{i \theta} y \| \leq 1$ gilt:
	\begin{align*}
		| \< T x, y \> | & = e^{i \theta} \Re \< T x , y \> \\
						 & = \Re\< T x , e^{i \theta} y \> \\
						 & \leq M
	\end{align*}
	$\sup_{\| x \|, \| y \| \leq 1} \< T x , y \> = \| T \| \leq M$.
\end{beweis}


\begin{prop} \label{prop:19.9}
	Sei $T \in B(X)$ normal
	\begin{enumerate}[label=\alph*\upshape)]
		\item $r(T) = \sup \{ | \lambda | : \lambda \in \sigma(T) \} = \| T \|$
		\item $\kernn T = \kernn T^{*}$
	\end{enumerate}	
\end{prop}

\begin{beweis}
	\begin{enumerate}[label=\alph*\upshape)]
		\item Es gilt $\| T \|^{2} = \| T^{2} \|$ $(*)$ \label{prop:19.9.5-*}, denn:
			\begin{align*}
				\| T^{2} \|^{2} & \overset{\hyperref[bem:19.2]{19.2}}{=} \| T^{2} \left( T^{2} \right)^{*} \| \qquad \left((T^{2})^{*} = T^{*} T^{*} = (T^{*})^{2}\right) \\
					& ~= \| (T T^{*}) (T T^{*})^{*} \| \overset{\hyperref[bem:19.2]{19.2}}{=} \| T T^{*} \|^{2} \overset{\hyperref[bem:19.2]{19.2}}{=} \left( \| T \|^{2} \right)^{2}
			\end{align*}
			$T$ normal $\Rightarrow T^{2}$ normal. Also $\| T^{2^{k}} \| = \| T \|^{2^{k}}$, wende \hyperref[prop:19.9.5-*]{$(*)$} $k$-mal an. 
			\[ r(T) = \lim_{n \rightarrow \infty} \| T^{n} \|^{\frac{1}{n}} = \lim_{k \rightarrow \infty} \| T^{2^{k}} \|^{\frac{1}{2^{k}}} = \lim_{k \rightarrow \infty} \left( \| T \|^{2^{k}} \right)^{\frac{1}{2^{k}}} = \| T \| \]
		\item $\| T x \|^{2} = \< T x , T x \> = \< x , T^{*} T x \> = \< x , T T^{*} x \> = \< T	^{*} x , T^{*} x \> = \| T^{*} x \|^{2}$ \\
			\[ \Rightarrow T x = 0 \quad \gdw \quad T^{*} x = 0 \]
	\end{enumerate}				
\end{beweis}


\begin{lemma} \label{lemma:19.10}
	Sei $X$ ein Hilbertraum und $T \in B(X)$ kompakt und normal, d.h.  $T T^{*} = T^{*} T$. Dann gilt
	\begin{enumerate}[label=\alph*\upshape)]
		\item $T x = \lambda x \gdw T^{*} x = \overline{\lambda} x$
		\item $T x = \lambda x, T y = \mu y$ mit $\mu \neq \lambda$  dann ist $x \bot y$
		\item Falls $\MdK = \MdC$, dann gibt es ein $\lambda \in \sigma(T)$ mit $| \lambda | = \| T \|$
		\item Falls $\MdK = \MdR$, dann ist $\sigma(T) \subset \MdR$ und $\| T \| \in \sigma(T)$ oder $- \| T \| \in \sigma(T)$
	\end{enumerate}
\end{lemma}

\begin{beweis}
	\begin{enumerate}[label=\alph*\upshape)]
		\item $T x = \lambda x \gdw x \in \kernn(T - \lambda) \gdw x \in \kernn(T - \lambda)^{*} \overset{\hyperref[prop:19.9]{19.9}}{\gdw} x \in \kernn(T^{*} - \overline{\lambda}) \gdw T^{*} x = \overline{\lambda} x$
		\item $\lambda \< x , y \> = \< \lambda x , y \> = \< T x , y \> = \< x , T^{*} y \> = \< x , \overline{\mu} y \> = \mu \< x, y \>$ $\Rightarrow \< x , y \> = 0$.
		\item Nach \hyperref[prop:19.9]{19.9a} ist $\ospan \{ | \lambda | : \lambda \in \sigma(T) \} = \| T \|$ \\
			Wähle $\lambda_{j} \in \sigma(T)$ mit $| \lambda_{j}| \rightarrow \| T \|$. Für eine Teilfolge $(\lambda_{n})$ von $(\lambda_{j})$ gilt: 
			\[ \lambda_{n} \rightarrow \lambda, | \lambda_{n} | \rightarrow | \lambda | = \| T \| \]
			Da $\lambda_{n} \in \sigma(T)$ und $\sigma(T)$ abgeschlossen folgt $\lambda \in \sigma(T), | \lambda | = \| T \|$.
		\item $T x = \lambda x, x \neq 0: \lambda \< x, x \> = \< T x , x \> = \< x , T x \> = \overline{\lambda} \< x , x \>$ \\
			\[ \Rightarrow \lambda = \overline{\lambda} \in \MdR \]
			$\MdK = \MdR$: Nach \hyperref[prop:19.8]{Prop. 19.8} ist $\| T \| = \sup \{ \< x , Tx \> : \| x \| \leq 1\}$. \\
			\[ \exists x_{j}, \| x_{j} \| = 1, | \< T x_{j} , x_{j} \> | \rightarrow \| T \|. \]
			Wähle Teilfolge $(x_{n})$ von $(x_{j})$ so, dass 
			\[ x \coloneqq \lim_{j} \< T x_{j}, x_{j} \>, \quad y \coloneqq \lim_{n} T x_{n} \]
			Da $|x| = \lim_{j} | \< T x_{j} , x_{j} \> | = \| T \|$ ist:
			\begin{align*}
				\| T x_{n} - \lambda_{n} x_{n} \|^{2} & = \| T x_{n} \|^{2} - 2 \< T x_{n} , x_{n} \> + \lambda_{n}^{2} \underbrace{\| x_{n} \|^{2}}_{= 1} \\
				& \rightarrow \lambda^{2} - 2 \lambda^{2} + \lambda^{2} = 0
			\end{align*} 
			Da $y = \lim_{n} \left( \lambda_{n} x_{n} \right)$: $T y = \lambda \lim T x_{n} = \lambda y$. $\Rightarrow \lambda \in \sigma(T)$, $\lambda \in \MdR$, $| \lambda | = \| T \|$.
	\end{enumerate}	
\end{beweis}


\begin{satz}[Spektralsatz für kompakte, normale Operatoren] \index{Spektralsatz für kompakte, normale Operatoren} \label{satz:19.11.b}
	Sei $X$ ein Hilbertraum, $T \in B(X)$ kompakt und normal. \\
	Dann gibt es eine Folge $(\lambda_{n}) \in \MdC \setminus \{ 0 \}$, die entweder endlich oder eine Nullfolge und es gibt ein Orthonormalsystem $(h_{n})$ in $X$, das endlich ist, falls $(\lambda_{n})$ endlich ist, so dass
	\[ Tx = \sum_{n} \lambda_{n} \< x , h_{n} \> h_{n} \quad \forall x \in X \]
	Insbesondere:
	\begin{itemize}
		\item $\sigma(T) \setminus \{ 0 \} = \{ \lambda_{n} \}$, $T h_{n} = \lambda_{n} h_{n}$
		\item $X = \left( \kernn T \right) \oplus \overline{\ospan} \{ h_{n} \}$, orthogonale Komplement
		\item $\| T \| = \sup_{n} | \lambda_{n} |$ 
	\end{itemize}
\end{satz}

\begin{bemerkung*}
	Falls $X$ separabel ist, so gibt es eine Orthonormalbasis $(e_{n})$ von $X$, die diese $(h_{n})$ und eine orthonormalbasis von $\kernn(T)$ so, dass 
	\[ T x = \sum_{n = 0}^{\infty} \mu_{n} \< x , e_{n} \> e_{n} \]
	wobei $\mu_{n} = 0$, falls $e_{n} \in \kernn(T), \mu_{n} = \lambda_{m}$ falls $e_{n} = h_{m}$. \\ \\  
	Die Abbildung $\phi \colon \ell^{2} \rightarrow X, \phi \left( (\alpha_{n}) \right) = \sum_{n} \alpha_{n} e_{n}$ ist eine Isometrie. \\
	Setze $D \coloneqq \phi^{-1} T \phi ( T e_{n} = \mu_{n} e_{n} )$, dann ergibt sich das Folgende und damit ein kommutative Diagramm:
	\[ x = (\< x , e_{n} \>) \rightarrow T x = ( \mu_{n} \< x , e_{n} \> ), \quad D( (\alpha_{n}) ) = ( (\mu_{n} \alpha_{n}) )\]
	\[ \begin{xy} \xymatrix{
			\ell^{2} \ar[r]^{D} \ar[d]^{\phi}  &  \ell^{2} \ar[d]_{\pi} \\
			X 		 \ar[r]_{T} 		  		   &  X		
	} \end{xy} \]
	Wenn $X$ endlich dimensional und $D$ kompakt wäre, könnte man $D$ als folgende Diagonalmatrix auffassen:
		\[ A = \begin{pmatrix}
					\diagentry{\mu_{1}}\\
					&\diagentry{\xddots}\\
					&&\diagentry{\mu_{n}}\\
				\end{pmatrix} 
		\] 
\end{bemerkung*}
 

\begin{beweis}
	Aus dem Kapitel über Spektraltheorie von kompakten Operatoren wissen wir $\sigma(T) \setminus \{ 0 \}$ besteht aus endlich vielen oder einer Nullfolge von Eigenwerten $(\nu_{n})$ mit $X_{k} = \kernn( T - \nu_{k} Id )$, $\dim X_{k} = d_{k} < \infty$. \\
	Sei $(\lambda_{n})$ eine Folge, die jeden Eigenwert $\nu_{k}$ entsprechend seiner Vielfachheit $d_{k}$ enthält:
\begin{align*}
	(\lambda_{n}) & = (\underbrace{v_{1}, \dotsc, v_{1}}_{d_{1}\text{-mal}}, \underbrace{v_{2}, \dotsc, v_{2}}_{d_{2}\text{-mal}}, \underbrace{v_{3}, \dotsc, v_{3}}_{d_{3}\text{-mal}}, \dotsc) \\
	(h_{n}) & = (e_{1}^{1}, \dotsc, e_{d_{1}}^{1}, e_{1}^{2}, \dotsc, e_{d_{2}}^{2}, e_{1}^{3}, \dotsc, e_{d_{3}}^{3}, \dotsc)
\end{align*}
Da für jedes $k$ $(e_{1}^{k}, \dotsc, e_{d_{k}}^{k})$ ein Orthonormalsystem von $X_{k}$ ist und $X_{k} \bot X_{l}$ für $k \neq l$ nach \hyperref[lemma:19.10]{19.10b}, ist $(h_{n})$ ein Orthonormalsystem in X. \\ 
Also: $T h_{n} = \lambda_{n} h_{n}$ und $h_{n} \bot \kernn(T)$ nach \hyperref[lemma:19.10]{19.10b} \\ 
Beh: $X = \kernn(T) \oplus X_{0}$, $X_{0} = \overline{\ospan}(h_{n})$, denn daraus würde schon die Behauptung folgen: \\
Für alle $x \in X$ ist $x = y + \sum_{n} \< x , h_{n} \> h_{n}$, da $(h_{n})$ Orthonormalbasis von  $X_{0}$
	\[ \Rightarrow T x = \underbrace{T y}_{= 0} + \sum_{n} \< x , h_{n} \> \underbrace{( T h_{n} )}_{\lambda_{n} h_{n}} ) \sum \lambda_{n} \< x, h_{n} \> h_{n}. \]
Zu zeigen ist also $X_{0}^{\bot} = \kernn(T)$ \\
Beweis: für $x \in X_{0}^{\bot}$ ist $Tx \in X_{0}^{\bot}$, denn für alle $n$ ist $\< T x , h_{n} \> = \< x , T h_{n} \> = \lambda \underbrace{\< x , h_{n} \>}_{= 0}$. \\
Also $T(X_{0}^{\bot}) \subset X_{0}^{\bot}$, setze $S = T|_{X_{0}^{\bot}} \in B(X_{0}^{\bot})$ kompakt und normal. Nach \hyperref[lemma:19.10]{19.10c, d)} gibt es  einen Eigenwert $\mu$ von $S$ mit $| \mu | = \| S \|$. \\ 
	\[ \Rightarrow \operatorname{Eig}(S - \mu) \subset X_{0} \cap X_{0}^{\bot} = \{  0 \} \Rightarrow S = 0 \Rightarrow X_{0}^{\bot} \subset \kernn S. \]
Andererseits: $\kernn T \bot X_{k} \Rightarrow \kernn T \subset X_{0}^{\bot} \Rightarrow X_{0}^{\bot} = \kernn(T)$.
\end{beweis}

	
	
\newpage