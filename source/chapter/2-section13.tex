%!TEX root = Funktionalanalysis - Vorlesung.tex

\section{Spektrum und Resolvente}



Sei $Y \supset D(A) \rightarrow X$ linear, $\lambda \in \MdC$.
	\[ (\lambda I - A) x = y \quad (*) \label{eq:13.0-lineareGleichung} \]
Problem: Gegeben $y \in X$ finde $x \in D(A)$ so, dass \hyperref[eq:13.0-lineareGleichung]{$(*)$} erfüllt ist.
Formel: $x = (\lambda I - A)^{-1} y$ ist Lösung, falls $(\lambda I - A)^{-1}$ existiert.


\begin{definition} \label{def:13.1}
	Sei $X$ ein Banachraum über $\MdC$, $A \colon X \supset D(A) \rightarrow X$ linear und abgeschlossen.
	\begin{enumerate}[label=\alph*\upshape)]
		\item $\lambda \in \MdC$ gehört zur \begriff{Resolventenmenge} von $A$, $\lambda \in \rho(A)$, falls
			\[ \lambda I - A \colon D(A) \rightarrow X \text{ bijektiv, d.h. } (\lambda I - A)^{-1} \colon X \rightarrow D(A) \text{ linear} \]
		\item $\sigma(A) = \MdC \setminus \rho(A)$ hei{\ss}t \begriff{Spektrum} von $A$
		\item $\lambda \in \rho(A) \rightarrow R(\lambda, A) = (\lambda - A)^{-1}$ hei{\ss}t \begriff{Resolventenfunktion} von A
	\end{enumerate}	
\end{definition}


\begin{erinnerung}
	$A$ abgeschlossen $\gdw$ $(x_{n})_{n} \subset D(A): \begin{cases}
			x_{n} \xrightarrow[]{n \rightarrow \infty} x & \text{ in } X \\ A x_{n} \xrightarrow[]{n \rightarrow \infty} y & \text{ in } X \end{cases}$, $ $ so ist $x \in D(A), A x = y$\end{erinnerung}


\begin{bemerkung}
	$A$ ist abgeschlossen, falls $\lambda \in \rho(A)$, so ist $R(\lambda, A) \in B(X)$ und $R(\lambda, A) \colon X \rightarrow (D(A), \| \cdot \|_{A})$ ein Isomorphismus.
\end{bemerkung}

\begin{beweis}
	$(\lambda - A): (D(A), \| \cdot \|_{A}) \rightarrow A$ ist bijektiv und stetig, denn $\|(\lambda - A)x \| \leq \max(1, |\lambda|) (\|x\| + \|Ax\|) = c \| x \|_{A}$ \\
	Nach dem \hyperref[satz:10.3-offeneAbbildung]{Satz der offenen Abbildung} ist
		\[ R(\lambda, A): X \rightarrow (D(A), \| \cdot \|_{A}) \]
	ein Isomorphismus, $X \xrightarrow[]{R(\lambda, A)} (D(A), \| \cdot \|_{A}) \subset X$, also $R(\lambda, A) \in B(X)$.
\end{beweis}


\begin{beispiel}
	\begin{enumerate}[label=\alph*\upshape)]
		\item $X = \MdC^{d}, A \in B(X) \equalhat M(d, d)$
			\[ \sigma(A) = \{ \lambda \text{ Eigenwerte von A} \} \]
		\item Sei $\alpha_{n} \in \MdC,$ $X = \ell^{p}, 1 \leq p < \infty, A(x_{n}) = (\alpha_{n} x_{n}),$ $(x_{n}) \in D(A) = \{ (x_{n}): \sum_{n \geq 1} | \alpha_{n} x_{n} |^{p} < \infty \}$. \\
			Falls $(\alpha_{n})$ beschränkt ist, dann ist $D(A) = \ell^{p}, A \in B(\ell^{p})$. \\
			Für allgemeine $(\alpha_{n}) \subset \MdC$ ist $A$ nur abgeschlossen (Übung). \\
			Dann ist $\sigma(A) = \overline{\{ \alpha_{n}, n \in \MdN \}}^{c}$, da $A(e_{n}) - \alpha_{n}(e_{n}) = 0$ ($e_{n}$ n-ter Einheitsvektor) \\ \\
			Beweis: $(\lambda I - A)(x_{n}) = ((\lambda - \alpha) x_{n})$ \\
			Formal: $(\lambda I - A)^{-1}(x_{n}) = ((\lambda - \alpha_{n})^{-1}x_{n})$
			\begin{align*}
				\| (\lambda - A)^{-1} \| \underset{\text{Übung}}{=} \sup_{n} | \lambda - \alpha_{n} |^{n} = \frac{1}{d(\lambda, \overline{(\alpha_{n})})} < \infty & \gdw d(\lambda, \overline{(\alpha_{n})}) > 0 \\
				& \gdw \lambda \notin \overline{(\alpha_{n})}
			\end{align*}
			$\Rightarrow \lambda \in \rho(A) \gdw \lambda \in \overline{\{\alpha_{n}\}}, $ $ $ $ \lambda \in \sigma(A) \gdw \lambda \in \overline{(\alpha_{n})}$ \\ \\
			\textbf{Folgerung:} Jede abgeschlossene Menge $S \subseteq \MdC$ kann das	 Spektrum eines abgeschlossenen Operators sein. Insbesondere: $\sigma(A)$ kann überabzählbar sein. Das Spektrum $\sigma(A)$ besteht im Allgemeinen nicht nur aus Eigenwerten.
			\begin{beweis}
				Gegeben $M \subset \MdC$ abgeschlossen, wähle dichte Folge $\alpha_{n} \in M$, d.h. $\{ \alpha_{n} \} = M$ \\
				Wähle $X = \ell^{p},$ $A(x_{n}) = (\alpha_{n} x_{n}),$ $\sigma(A) = \overline{\{ \alpha_{n} \}} = M$.
			\end{beweis}
			Falls $\lambda \in \overline{\{ \lambda_{n} \}} \setminus \{ \lambda_{n} \}$ dann ist $\lambda$ kein Eigenwert von $A$.
		\item Sei $X = \ell^{p}$, $e_{n}$ Einheitsvektoren.
			\[ A (e_{1}) = 0, A (e_{n}) = e_{n - 1}, n > 1 \Rightarrow A(x_{1}, x_{2}, x_{3}, \dotsc) = (x_{2}, x_{2}, x_{3}, \dotsc) \]
			\[ B (e_{n}) = e_{n + 1}, n \geq 1 \Rightarrow B(x_{1}, x_{2}, x_{3}, \dotsc) = (0, x_{1}, x_{2}, x_{3}, \dotsc) \]
			Übung: $\sigma(A) = \sigma(B) = \{ \lambda : | \lambda | \leq 1 \}$
	\end{enumerate}
\end{beispiel}


\begin{satz}[Resolventendarstellung] \label{satz:13.4-Resolventendarstellung} \index{Resolventendarstellung}
	Sei $X \supset D(A) \xrightarrow[]{A} X$ abgeschlossen, $X$ ein Banachraum. \\
	Für $\lambda_{0} \in \rho(A)$ und $\lambda \in \MdC$ mit $|\lambda - \lambda_{0}| < \frac{1}{\| R(\lambda_{0}, A) \|}$ ist auch \\
		\[ \lambda \in \rho(A) \text{ und } R(\lambda, A) = \sum_{n \geq 0} (\lambda_{0} - \lambda)^{n} R(\lambda_{0}, A)^{n + 1}. \] \\
	Insbesondere ist $\rho(A)$ offen und $\sigma(A)$ abgeschlossen.
\end{satz}

\begin{beweis}
	\begin{align*}
		 (\lambda - A) = (\lambda_{0} + \lambda) + (\lambda_{0} - A) & = (\lambda_{0} - A)\left[I - (\lambda_{0} - \lambda) R(\lambda_{0}, A)\right] \\
		 & = (\lambda_{0} - A)(I - S) \quad \text{ mit } S = (\lambda_{0} - \lambda)R(\lambda_{0}, A)
	\end{align*}	
	\[ \Rightarrow \| S \| \leq | \lambda_{0} - \lambda| \| R(\lambda_{0}, A) \| \overset{Vor.}{<} 1. \text{ Nach dem \hyperref[prop:5.8-NeumannscheReihe]{Satz über die Neumannsche Reihe}: } (I - S)^{-1} = \sum_{n \geq 0} S^{n} \]
	Dann ist $(\lambda - A)$ ein Produkt von invertierbaren Operatoren $(\lambda_{0} - A)$ und $(I - S)$, d.h. 
	\begin{align*}
		(\lambda - A)^{-1} & = (I - S)^{-1}(\lambda_{0} - A)^{-1} \\
			& = \sum_{n \geq 0} \underbrace{(\lambda_{0} - \lambda)^{n}R(\lambda_{0}, A)^{n}}_{= S^{n}} R(\lambda_{0}, A) \\
			& = \sum_{n \geq 0} (\lambda_{0} - \lambda)^{n} R(\lambda_{0}, A)^{n + 1}
	\end{align*} 
\end{beweis}


\begin{satz}[Resolventengleichung] \index{Resolventengleichung}
	Sei $A$ ein abgeschlossener Operator auf $X$. Für $\lambda, \mu \in \rho(A)$ gilt:
		\[ R(\lambda, A) - R(\mu, A) = (\mu - \lambda) R(\lambda, A) R(\mu, A) \]
	Insbesondere ist $\lambda \in \rho(A) \rightarrow R(\lambda, A) \in B(X)$ eine komplex differenzierbare Abbildung und 
		\[ \frac{d}{d \lambda} R(\lambda, A) = - R(\lambda, A)^{2} \]
\end{satz}

\begin{beweis}
	\begin{align*}
		R( \lambda, A) - R( \mu, A) & = R( \lambda, A) \left[ I - (\lambda - A) R(\mu, A) \right]	\\
			& = R( \lambda, A) \left[ \mu - A - \lambda + A \right] R(\mu, A)
	\end{align*}
	$\Rightarrow$ Behauptung. (Idee: $\frac{1}{\lambda - a} - \frac{1}{\mu - a} = \frac{\mu - \lambda}{(\lambda - a)(\mu - a)}$)
	\begin{align*}
		\frac{d}{dx} R(\lambda, A) & = \lim_{\mu \rightarrow \lambda} \frac{R(\mu, A) - R(\lambda, A)}{\mu - \lambda} \\
				& \overset{s.o.}{=} \lim_{\mu \rightarrow \lambda} \left[ - R( \lambda, A) R(\mu, A) \right] \\
				& = - R(\lambda, A)^{2}, \text{ denn } \lambda \in \rho(A) \rightarrow R(\lambda, A) \in B(X) \text{ ist stetig als Potenzreihe.} 		
	\end{align*}
\end{beweis}


\begin{satz} \label{satz:13.6}
	Falls $A \in B(X)$, dann ist $\sigma(A)$ nichtleer und kompakt mit $\sigma(A) \subset \{ \lambda : |\lambda| \leq \| A \| \}$ \\
	\[ \text{Für } \lambda > \| A \| \text{ gilt: } R(\lambda, A) = \sum_{n \geq 0} \lambda^{-n-1} A^{n} \]
\end{satz}

\begin{beweis}
	Für $| \lambda | > \| A \|$ gilt: $\lambda - A = \lambda [I - S]$ mit $S = \frac{A}{\lambda}, \| S \| \leq \frac{\| A \|}{| \lambda |} < 1$ nach Voraussetzung. \\
	Nach Neumann: 
		\[ (I - S)^{-1}  = \sum_{n = 0}^{\infty} S^{n} \]
	\begin{align*}
		\Rightarrow (\lambda - A)^{-1} & = \lambda^{-1} [I - S]^{-1} \\
			& = \lambda^{-1} \sum_{n = 0}^{\infty} \left( \frac{A}{\lambda} \right)^{n} \\
			& = \sum_{n = 0}^{\infty} \lambda^{-n-1} A^{n}
	\end{align*}
	Also: $\sigma(A) \subset \{ \lambda: |\lambda| \leq \|A\| \}, \sigma(A)$ beschränkt, abgeschlossen $\Rightarrow \sigma(A)$ kompakt. \\
	Wir müssen noch zeigen, dass $\sigma(A) \neq \emptyset$. \\
	Nach \hyperref[bem:8.7]{Bemerkung 8.7} gibt es in jedem Banachraum $X$ $x \in X, x' \in X'$ mit $x'(x) \neq 0$ $(*)$ \label{eq:13.6.5-DualAbbildungAuf0} \\
	Indirekter Beweis für $\sigma(A) \neq \emptyset$. \\
	Annahme $\sigma(A) = \emptyset$ bzw. $\rho(A) = \MdC$ \\
	$\Rightarrow \lambda \in \MdC \rightarrow r(\lambda) = x'[R(\lambda, A)x] \in \MdC$ mit $x, x'$ wie in \hyperref[eq:13.6.5-DualAbbildungAuf0]{$(*)$}. \\
	$r(\lambda)$ ist holomorph auf $\MdC$, denn lokal gilt:
		\[ r(\lambda) = \sum_{n = 0}^{\infty} (\lambda_{0} - \lambda)^{n} \underbrace{x'[R(\lambda, A) x]}_{\in \MdC}, \quad |\lambda - \lambda_{0}| \overset{\ref{satz:13.4-Resolventendarstellung}}{<} \frac{1}{\| R(\lambda_{0}, A)\|} \]
	Wir definieren $\Gamma = \{ \lambda: |\lambda| = 2 \| A \| \}$. \\
	Nach dem Cauchischen Integralsatz: $0 = \int_{\Gamma} r(x) d\lambda$
	\[ \lambda \in \Gamma, |\lambda| > \| A \|, R(\lambda, A) = \sum_{n = 0}^{\infty} \lambda^{-n-1} A^{n} \]
	\[ \Rightarrow r(\lambda) = x'(R(\lambda, A)x) = \sum_{n = 0}^{\infty} \lambda^{-n-1} x'(A^{n}x) \]
	Dann ist $0 = \int_{\Gamma} r(\lambda) d\lambda = \int_{\Gamma} \left[ \sum_{n = 0}^{\infty} \lambda^{-n-1} x'(A^{n}x) \right] dx$
	\[ 0 = \sum_{n = 0}^{\infty} \underbrace{\left[ \int_{\Gamma} \lambda^{-n-1} d\lambda \right]}_{= 0, \text{ für } n > 0, = 2 \pi i,  \text{ für } n = 0} x'(A^{n} x) \]
	$\Rightarrow \sigma(A) \neq \emptyset$.
\end{beweis}


\begin{bemerkung} \label{bem:13.7}
	Für $X = L^{p}(\Omega)$ ist \hyperref[eq:13.6.5-DualAbbildungAuf0]{$(*)$} erfüllt. $x \in L^{p}(\Omega), x \neq 0, x'(w) = |x(w)|^{p-1} \sign(w), w \in \Omega$ \\
	Dann ist 
	\[ x'(x) = \int | x(w) |^{p} dw \neq 0, x' \in L^{p'}, \int |x'|^{p'} dw = \int |x(w)^{(p-1)p'} dw \]
	Allgemein folgt \hyperref[eq:13.6.5-DualAbbildungAuf0]{$(*)$} aus dem \hyperref[satz:20.2-HahnBanach]{Satz von Hahn-Banach}. 
\end{bemerkung}


\begin{definition} \label{def:13.8-Spektralradius}
	Für $A \in B(X)$ hei{\ss}t $r(A) = \sup \{ | \lambda |: \lambda \in \sigma(A) \}Q$ der \begriff{Spektralradius} von $A$.
\end{definition}


\begin{satz} \label{satz:13.9}
	Für $A \in B(X)$ ist $r(A) = \lim_{n \rightarrow \infty} \| A^{n} \|^{\frac{1}{n}} = \inf_{n \in \MdN} \| A^{n} \|^{\frac{1}{n}}$ \\ \\
	Hilfssatz: Ist $a_{n} \in \MdR$ mit $0 \leq a_{n + m} \leq a_{n} \cdot a_{m}$, $n,m \in \MdN$, dann gilt $\lim_{n \rightarrow \infty} a_{n}^{\frac{1}{n}} = \inf_{n \in \MdN} a_{n}^{\frac{1}{n}}$
\end{satz}

\begin{beweis}
	Beweis des Hilfssatzes: \\
	Zu $\epsilon > 0$ wähle $N \in \MdN$ mit $(a_{N})^{\frac{1}{N}} \leq a + \epsilon$. Setze $b = b(\epsilon) = \max \{ a_{1}, \dotsc, a_{N} \}$. \\
	Jedes $n \in \MdN$ lässt sich schreiben als $n = k \cdot N + r, k \in \MdN, r \in \{ 1, \dotsc, N - 1 \}$. \\
	Damit folgt:
		\[ (a_{n})^{\frac{1}{n}} = a_{k N + r}^{\frac{1}{n}} \leq (a_{N}^{k} a_{r})^{\frac{1}{n}} \leq (a + \epsilon)^{\frac{k N}{n}} \cdot b^{\frac{1}{n}} = (a + \epsilon)(a + \epsilon)^{- \frac{r}{n}} b^{\frac{1}{n}} < a + 2 \epsilon \text{ für } n \text{ gro{\ss} genug.} \]
	Beweis von \hyperref[satz:13.9]{13.9}: $a_{n} \coloneqq \| A^{n} \|$ erfüllt
	 \[ a_{n + m} = \| A^{n + m} \| = \| A^{n} A^{m} \| \leq \|A^{n}\| \cdot \|A^{m}\| = a_{n} \cdot a_{m} \Rightarrow \lim_{n \rightarrow \infty} \| A^{n} \|^{\frac{1}{n}} = \inf_{n} \| A^{n} \|^{\frac{1}{n}} \]
	Nach \hyperref[satz:13.6]{13.6} gilt für $\left| \frac{1}{\lambda} \right| > \| A \|$ die Darstellung: $R(\frac{1}{\lambda}, A) = \sum_{n \geq 0} \lambda^{n +1} A^{n} (*) \label{eq:13.9.5-*}$ \\ 
	Nach dem Wurzelkriterium konvergiert diese Reihe falls $\overline{\lim_{n \rightarrow \infty}} \| \lambda^{n + 1} A^{n} \|^{\frac{1}{n}} = |\lambda| \lim_{n \rightarrow \infty} \| A^{n} \|^{\frac{1}{n}} < 1$, d.h. $| \lambda | < \left[ \lim_{n \rightarrow \infty} \| A^{n} \|^{\frac{1}{n}} \right]^{-1}$
	\[ \Rightarrow R( \frac{1}{\lambda}, A) \in B(X) \text{ für } |\frac{1}{\lambda} > \lim_{n \rightarrow \infty} \| A^{n} \|^{\frac{1}{n}}. \]
	Also $\sigma(A) \subset \{ \lambda: |\lambda| < \lim \| A^{n} \|^{\frac{1}{n}} \}, \quad r(A) \leq \lim_{n \rightarrow \infty} \| A^{n} \|^{\frac{1}{n}} \leq \|A\|$. \\ \\
	Wie in der Funktionentheorie zeigt man, dass wegen \hyperref[eq:13.9.5-*]{$(*)$} $R(\lambda, A)$ im grö{\ss}ten Kreis, der zum Holomorphie-Gebiet von $\frac{1}{\lambda} \in \rho(A) \rightarrow R(\frac{1}{\lambda}, A)$ gehört, dort als die Potenzreihe in \hyperref[eq:13.9.5-*]{$(*)$} dargestellt werden kann. \\
	Also $\lim_{n \rightarrow \infty} \|A^{n} \|^{\frac{1}{n}} = r(A)$
\end{beweis}

Im Allgemeinen gilt $r(A) < \| A \|$.


\begin{beispiel} \label{bsp:13.10}
	Sei $X = C[0, 1]$ und  $Q = \{ (s, t) \in [0, 1]^{2}: s \leq t \}$, betrachte $k: Q \rightarrow \MdR$ stetig.  \\
	Wir definieren einen Volterraoperator $V: C[0, 1] \rightarrow C[0, 1]$ durch 
		\[ (V x)(t) = \int_{0}^{t} k(t, s) x(s) ds, t \in [0, 1], x \in C[0, 1] \]
	Behauptung: $V \in B(C[0, 1]), \| V \| = \sup_{t \in [0, 1]} \int_{0}^{t} k(t, s) ds > 0,$ $r(v) = 0,$ $\sigma(v) = \{ 0 \}$, d.h. \\
	\[ (\lambda - v)x = y \text{ ist für alle } y \in C[0, 1], \lambda \neq 0, \text{ eine eindeutige Lösung } x = (\lambda - A)^{-1} y \in C[0, 1] \]
\end{beispiel}

\begin{beweis}
	$\| V x \|_{\infty} = \sup_{7 \in [0, 1]} \int_{0}^{t} k(t, s) x(s) ds \leq \underbrace{\sup_{t \in [0, 1]} \int_{0}^{t} k(t, s) ds}_{=: K} \| x \|_{\infty} \Rightarrow \| V \| \leq K$ \\
	\[ \| \1 \|_{\infty} = 1, \quad \| V \1 \|_{\infty} = \sup_{t \in [0, 1]} \int_{0}^{t} k(t, s) ds \quad \Rightarrow \quad \| V \| = K \]
	Wir müssen zeigen, dass $r(V) = \lim_{n \rightarrow \infty} \| V^{n} \|^{\frac{1}{n}} = 0$. \\
	$V^{n}$ hat die Form $V^{n} x(t) = \int_{0}^{t} k^{n}(t, s) x(s) ds$, mit
	\begin{align*}
		\left( V^{n + 1} x \right)(t) & = a
	\end{align*}
\end{beweis}



\newpage