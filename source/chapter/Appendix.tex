%!TEX root = Funktionalanalysis - Vorlesung.tex


\section*{Ergänzungen}\addcontentsline{toc}{section}{Ergänzungen} 


Zur Vollständigkeit und solange dieses Skript in Arbeit ist, werden hier Definitionen und Sätze gelistet, auf die das Skript verweist, aber (noch) nicht eingeführt wurden.

~

\begin{definition*}
	Ist $V$ ein Vektorraum über einem beliebigen Körper und ist $U$ ein Untervektorraum von $V$, dann wird die \begriff{Kodimension} von $U$ in $V$ durch
	\[ \mathrm{codim}(U,V) = \dim ( \QR{V}{U} ), \]
	also als die Dimension des Faktorraums $\QR{V}{U}$, definiert.
\end{definition*}

\begin{bemerkung*}
	Es gilt stets
		\[ \dim U+\mathrm{codim}(U,V)=\dim V. \]
	Ist $W$ ein Komplementärraum von $U$ in $V$, d. h. $U \oplus W = V$, so ist
		\[ \mathrm{codim}(U,V)=\dim W. \]
	Sind $U_1,U_2\subseteq V$ zwei Unterräume, so gilt stets
		\[ \mathrm{codim}(U_1\cap U_2,V)\leq\mathrm{codim}(U_1,V)+\mathrm{codim}(U_2,V). \]
	Sind $U,W\subseteq V$ Unterräume, so gilt
		\[ \mathrm{codim}(U\cap W,W)=\mathrm{codim}(U,U+W)\leq\mathrm{codim}(U,V). \]
\end{bemerkung*}


\begin{satz*}[Fubini] \index{Fubini} \label{satz:x-SatzvonFubini}
	Seien $(\Omega_1, \mathcal{A}_1, \mu_1)$ und $(\Omega_2, \mathcal{A}_2, \mu_2)$ zwei $\sigma$-endliche Maßräume und $f \colon \Omega_1 \times \Omega_2 \rightarrow \MdR$ eine messbare Funktion, die bezüglich des Produktmaßes $\mathrm \mu_1 \otimes \mu_2$ integrierbar ist, d.h. es gelte
		\[ \int\limits_{\Omega_1 \times \Omega_2} |f| \, \mathrm d(\mu_1 \otimes \mu_2) < \infty \ \]
	oder es gelte $f \geq 0$ fast überall. \\ \\
	Dann ist für fast alle $y$ die Funktion $ x \mapsto f(x,y)$ und für fast alle $x$ die Funktion $y \mapsto f(x,y)$ integrierbar bzw. nichtnegativ. \\
	Man kann deshalb die durch Integration nach $y$ beziehungsweise $x$ definierten Funktionen
		\[  x \mapsto  \int\limits_{\Omega_2} f(x,y) ~ \mathrm d\mu_2(y)  \]
		\[ y \mapsto  \int\limits_{\Omega_1} f(x,y) ~ \mathrm d\mu_1(x) \]
	betrachten. Diese sind auch integrierbar bzw. nichtnegativ und es gilt
		\[ \int\limits_{\Omega_1 \times \Omega_2} f ~ \mathrm d(\mu_1 \otimes \mu_2) = \int\limits_{\Omega_2}^{}\int\limits_{\Omega_1}^{}f(x,y)~ \mathrm d\mu_1(x)~ \mathrm d\mu_2(y) = \int\limits_{\Omega_1}^{}\int\limits_{\Omega_2}^{}f(x,y)~\mathrm d\mu_2(y)~ \mathrm d\mu_1(x). \]
\end{satz*}

~

\begin{satznbfr}[Lebesgue] \index{Lebesgue} \label{satz:x-SatzvonLebesgue}
Sei $(\Omega,\mathcal{A},\mu)$ ein Maßraum und sei $\left(f_n\right)$ eine Folge von $\mathcal{A}$-messbaren Funktionen $f_n \colon \Omega\rightarrow\MdR\cup\{\infty\}$. \\ \\
Die Folge $\left( f_n \right)$ konvergiere $\mu$-fast überall gegen eine $\mathcal{A}$-messbare Funktion $f$. Ferner werde die Folge $\left(f_n\right)$ von einer $\mu$-integrierbaren Funktion $g$ auf $\Omega$ majorisiert, sprich für alle $n \in \mathbb{N}$ gelte $|f_{n}| \leq g$ $\mu$-fast überall. \\
Beachte, dass bei der hiesigen Definition von Integrierbarkeit der Wert $\infty$ ausgeschlossen ist, das heißt $\int_\Omega|g|\,d\mu < \infty$. \\ \\
Dann ist $f$ und alle $f_n$ $\mu$-integrierbar und es gilt:
	\[ \lim_{n \rightarrow \infty}\int_\Omega{|f_n - f|}\,d\mu = 0 \]
Dies impliziert auch, dass
	\[ \lim_{n \rightarrow \infty}\int_\Omega{f_n\,}d\mu = \int_\Omega{f\,}d\mu \]
gilt.
\end{satznbfr}


Dem Skript vorgreifend, aber da bereits darauf verwiesen wurde, hier der Satz von Hahn-Banach:


\begin{satznbfr}[Hahn-Banach] \index{Hahn-Banach} \label{satz:x-hahn-banach}
	Es sei $X$ ein Vektorraum über $\MdK \in \{ \MdR, \MdC \}$. \\
	Seien nun
	\begin{itemize}
		\item $Y \subseteq X$ ein linearer Unterraum;
		\item $p \colon X \rightarrow \MdR$ eine sublineare Abbildung;
		\item $f \colon Y \rightarrow \MdK$ ein lineares Funktional, für das $\Re(f(y)) \leq p(y)$ für alle $y \in Y$ gilt.
	\end{itemize}
	Dann gibt es ein lineares Funktional $F \colon X \rightarrow \MdK$, so dass
	\[ F|_{Y} = f \quad \text{und} \quad \Re(F(x)) \leq p(x) \quad \text{für alle } x \in X \text{ gilt.} \]
\end{satznbfr}



\newpage