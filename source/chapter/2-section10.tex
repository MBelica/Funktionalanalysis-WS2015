%!TEX root = Funktionalanalysis - Vorlesung.tex

\section{Satz von der offenen Abbildung}



\begin{definition} \label{def:10.1-offAbbildung}
	Eine Abbildung zwischen metrischen Räumen heißt \begriff{offen}, wenn offene Mengen auf offene Mengen abgebildet werden.
\end{definition}


\begin{lemma} \label{lemma:10.2}
	Seien $X, Y$ normierte Räume und $T: X \rightarrow Y$ ein linearer Operator, dann sind äquivalent:
	\begin{enumerate}[label=\alph*\upshape)]
		\item $T$ ist offen
		\item $\exists \epsilon > 0: K_{Y}(0, \epsilon) \subset T(K_{X}(0, 1))$
	\end{enumerate}
\end{lemma}

\begin{beweis}
	$"\Rightarrow":$ $T(K_{X}(0, 1))$ ist offen und $T(0) = 0$. \\ \\
	$"\Leftarrow":$ Sei $U \subset X$ offen und $x \in U$ beliebig. Nehme ein $\epsilon > 0$, sodass $K_{X}(x, \epsilon) \subset U$. \\
	Dann ist $T( K_{X}(x, \epsilon)) \subset T(U)$ und damit auch
	\[ Tx + \epsilon T(K_{X}(0, 1)) = Tx + T(K_{X}(0, \epsilon)) \subset T(U) \]
	Nach $b)$ $\exists \delta > 0: K_{Y}(0, \delta) \subset T(K_{X}(0, 1))$
	\[ \Rightarrow ~ K_{Y}(Tx, \epsilon \delta) = T x + \epsilon K_{Y}(0, \delta) \subset T(U) \]	
	Somit ist $T(U)$ offen.
\end{beweis}


\begin{satz}[von der offenen Abbildung] \index{Satz von der offenen Abbildung}  \label{satz:10.3-offeneAbbildung}
	Seien $X, Y$ Banachräume und $T \in B(X, Y)$, dann gilt:
	\[ T \text{ surjektiv} \gdw T \text{ offen} \]
\end{satz}

\begin{beweis}
	$"\Leftarrow":$ Nach \hyperref[lemma:10.2]{Lemma 10.2} $\exists \epsilon > 0: K_{Y}(0, \epsilon) \subset T(K_{X}(0, 1)) $
	\begin{align*}
		& \Rightarrow K_{Y}(0, R) \subset T(K_{X}(0, \frac{R}{\epsilon})), ~ \forall R > 0 \\
		& \Rightarrow Y \subset T(X)
	\end{align*}
	$"\Rightarrow":$ Behauptung: $\exists \epsilon > 0: K_{y}(0, \epsilon) \subseteq \overline{T(K_{X}(0,1))}$. \\
	Beweis: Definiere $E_{n} \coloneqq \overline{T(K_{X}(0, n))}$, dann ist $E_{n}$ abgeschlossen, symmetrisch und konvex, da Kugeln um $0$ symmetrisch und konvex sind und $T$ linear ist. \\
	Nach Voraussetzung ist $Y = \bigcup_{n = 1}^{\infty} E_{n}$, mit \hyperref[satz:9.1-baire]{Baire} folgt $\exists N \in \MdN, y \in E_{N}, \hat \epsilon > 0: K(y, \hat \epsilon) \subseteq E_{N}$. Sei $z \in Y$ mit $\| z \| < \hat \epsilon$ und $z = \frac{1}{2} (z + y) + \frac{1}{2} (z - y) \in E_{N}$
		\[ \Rightarrow K_{Y}(0, \hat \epsilon) \subseteq \overline{T(K_{X}(0, N))} \Rightarrow K_{Y}(0, \frac{\hat \epsilon}{N}) \subseteq \overline{T(K_{X}(0, 1))} \quad \Rightarrow \quad \text{Behauptung mit } \epsilon = \frac{\hat \epsilon}{N}  \]
		Behauptung: $\exists \delta > 0: K_{Y}(0, \delta) \subseteq T(K_{X}(0, 1))$ \\
		Beweis: Wenn $\hat y \in K_{Y}(0, \epsilon)$ gibt es ein $\hat x \in K_{X}(0, 1)$ mit $\| \hat y - T \hat x \| \leq \frac{\epsilon}{2}$. \\
		Au{\ss}erdem ist $2 (\hat y - T \hat x) \subset K_{Y}(0, \epsilon)$. Induktiv wählen wir auf diese Art zu $y \in K_{Y}(0, \epsilon): y_{0} = y, y_{k + 1} = 2 ( y_{k} - T x_{k})$ für $k \in \MdN$. \\
		Es gilt $(y_{k}) \subseteq K_{Y}(0, \epsilon)$ und $(x_{k}) \subseteq K_{X}(0, 1)$, damit folgt
		\[ T \left( \sum_{k = 0}^{N} 2^{-K} x_{k} \right) = \sum_{k = 0}^{N} 2^{-k} y_{k} - 2^{-(k+1)} y_{k+1} = y_{0} 2^{0} - 2^{-(N+1)} y_{N+1} \quad (*) \label{eq:10.3.5} \]
		Wegen $\| \sum_{k = 0}^{N} 2^{-k} x_{k} \| \leq \sum_{k = 0}^{N} 2^{-k} \leq 2$ ist $\left( \sum_{k = 0}^{N} x_{k} \right)_{N}$ eine Cauchy-Folge mit Grenzwert $x \in K_{X}(0, 2) \subseteq X \xRightarrow[]{\hyperref[eq:10.3.5]{(*)}} T(x) = \lim_{N \rightarrow \infty} y - 2^{-(N+1)} y_{N+1} = y$. 
		\[ K_{Y} \left( 0, \epsilon \right) \subseteq T \left( K_{X}(0, 2) \right) \Rightarrow K_{Y} \left( 0, \frac{\epsilon}{2} \right) \subseteq T \left( K_{X}(0, 1) \right) \quad \Rightarrow \quad \text{Behauptung für } \delta = \frac{\epsilon}{2} \]
\end{beweis}


\begin{kor} \label{kor:10.4}
	Seien $X, Y$ Banachräume und $T \in B(X, Y)$ bijektv, dann ist $T^{-1} \in B(Y, X)$	
\end{kor}

\begin{beweis}
	Nach \hyperref[satz:10.3-offeneAbbildung]{10.3} ist $T$ offen, d.h. ist $U \subset X$ offen, so ist auch $T(U)$ offen in $Y$. \\
	\[ \Rightarrow T(U) = \left( T^{-1} \right)^{-1}(U) \text{ ist offen.} \]
	Somit sind Urbilder offener Mengen offen unter $T^{-1}$ und damit ist $T^{-1}$ stetig. Und da die Inversen linearer Operatoren bekanntlich linear sind, 	ist $T^{-1} \in B(Y, X)$.
\end{beweis}


\begin{kor} \label{kor:10.5}
	Sei $X$	ein Vektorraum der sowohl mit $\| \cdot \|$ als auch mit $\vertiii{\cdot}$ ein Banachraum ist. Gilt 
		\[ \exists c > 0: \| x \| \leq c \cdot \vertiii{x}, ~ \forall x \in X, \]
		dann sind die Normen äquivalent, d.h. $\exists \hat{c} $ mit $\hat{c} \cdot \vertiii{x} \leq \| x \|$ $\forall x \in X \leq c \cdot \vertiii{x}$. 
\end{kor}

\begin{beweis}
	Per Voraussetzung ist $I \colon (X, \| \cdot \|) \rightarrow (X, \vertiii{\cdot})$ beschränkt, außerdem ist die Einbettung offensichtlich linear und bijektiv. Nach $\ref{kor:10.4}$ ist dann auch die Inverse $I \colon (X, \| \cdot \|) \rightarrow (X, \vertiii{\cdot})$ in beschränkt.
\end{beweis}



\newpage