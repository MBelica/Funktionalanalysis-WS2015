%!TEX root = Funktionalanalysis - Vorlesung.tex

\chapter*{Abkürzungsverzeichnis\markboth{Abkürzungsverzeichnis}{Abkürzungsverzeichnis}}
\addcontentsline{toc}{chapter}{Abkürzungsverzeichnis}
\begin{acronym}
    \acro{Beh.}{Behauptung}
    \acro{Bew.}{Beweis}
    \acro{bzgl.}{bezüglich}
    \acro{bzw.}{beziehungsweise}
    \acro{ca.}{circa}
    \acro{d. h.}{das hei{\ss}t}
    \acro{Def.}{Definition}
    \acro{etc.}{et cetera}
    \acro{ex.}{existieren}
    \acro{Hom.}{Homomorphismus}
    \acro{i. A.}{im Allgemeinen}
    \acro{o. B. d. A.}{ohne Beschränkung der Allgemeinheit}
    \acro{Prop.}{Proposition}
    \acro{sog.}{sogenannte}
    \acro{Vor.}{Voraussetzung}
    \acro{vgl.}{vergleiche}
    \acro{z. B.}{zum Beispiel}
    \acro{zhgd.}{zusammenhängend}
    \acro{z. z.}{zu zeigen}
\end{acronym}
