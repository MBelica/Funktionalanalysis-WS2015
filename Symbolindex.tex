%!TEX root = FunkAnalysis.tex

\markboth{Symbolverzeichnis}{Symbolverzeichnis}

\chapter*{Symbolverzeichnis}
\addcontentsline{toc}{chapter}{Symbolverzeichnis}


\section*{Zahlenmengen}

$	\MdN = \{ 1, 2, 3, \dots \}  \; \text{Natürliche Zahlen} 					$ \\
$	\mdz = \mdn \cup \{ 0, -1, -2, \dots \}  \; \text{Ganze Zahlen} 				$ \\
$	\mdq = \mdz \cup \{ \frac{1}{2}, \frac{1}{3}, \frac{2}{3} \} = \{ \frac{z}{n} \text{ mit } z \in \mdz \text{ und } n \in \mdz \setminus \{ 0 \} \}  \;\text{Rationale Zahlen} 		$ \\
$	\mdr = \mdq \cup \{ \sqrt{2}, -\sqrt[3]{3}, \dots \}  \; \text{Reele Zahlen}$ \\
$	\mdr_+\; \text{Echt positive reele Zahlen}					$ \\
$	\mdc = \{a+ib|a,b \in \mdr \} \; \text{Komplexe Zahlen}	$ \\
$	I = [0,1] \subsetneq \mdr \; \text{Einheitsintervall}	$	