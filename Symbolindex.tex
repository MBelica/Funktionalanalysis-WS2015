%!TEX root = FunkAnalysis.tex

\markboth{Symbolverzeichnis}{Symbolverzeichnis}

\chapter*{Symbolverzeichnis}
\addcontentsline{toc}{chapter}{Symbolverzeichnis}


%%%%%%%%%%%%%%%%%%%%%%%%%%%%%%%%%%%%%%%%%%%%%%%%%%%%%%%%%%%%%%%%%%%%%
% Zahlenmengen                                                      %
%%%%%%%%%%%%%%%%%%%%%%%%%%%%%%%%%%%%%%%%%%%%%%%%%%%%%%%%%%%%%%%%%%%%%
\section*{Zahlenmengen}
$\MdN = \Set{1, 2, 3, \dots} \;\;\;$ Natürliche Zahlen\\
$\mdz = \mdn \cup \Set{0, -1, -2, \dots} \;\;\;$ Ganze Zahlen\\
$\mdq = \mdz \cup \Set{\frac{1}{2}, \frac{1}{3}, \frac{2}{3}} = \Set{\frac{z}{n} \text{ mit } z \in \mdz \text{ und } n \in \mdz \setminus \Set{0}} \;\;\;$Rationale Zahlen\\
$\mdr = \mdq \cup \Set{\sqrt{2}, -\sqrt[3]{3}, \dots}\;\;\;$ Reele Zahlen\\
$\mdr_+\;$ Echt positive reele Zahlen\\
$\mdc = \Set{a+ib|a,b \in \mdr}\;\;\;$ Komplexe Zahlen\\
$I = [0,1] \subsetneq \mdr\;\;\;$ Einheitsintervall\\

\settowidth\mylengtha{$f:S^1 \hookrightarrow \mdr^2$}
\setlength\mylengthb{\dimexpr\columnwidth-\mylengtha-2\tabcolsep\relax}