\subsection*{14 - Spektr. komp. Op.}

\begin{karte}{$K \in K(X)$ vs. $I - K$}
	Sei $X$ ein Banchraum, $K \in K(X)$ (d.h. $K \in B(X)$ kompakt bzw. $K(U_{X})$ ist relativ kompakt in $X$), dann hat $I - K$ ein abgeschlossenen Bildraum und 
		\[ \dim \kernn(I  - K) = \codim(I - K)(X) \left[ = \dim \QR{ X }{ (I - K)(X) }  \right] < \infty \]
		Insbesondere: $I - K$ injektiv $\gdw I - K$ surjektiv
\end{karte}

\begin{karte}{Zerlegung von $X$ zu einem Operator $F$}
	Zu jedem endlich dimensionalen $F \in B(X)$ ($\dim F(X) < \infty$) gibt es eine Zerlegung
		\[ X = X_{0} \oplus X_{1}, \quad \dim X_{1} < \infty \quad \text{und} \quad F(X_{1}) \subset X_{1}, \quad F|_{X_{0}} = 0 \]
\end{karte}

\begin{karte}{Zusammenhang kompakter Operator und Spektrum}
	Sei $dim X = \infty$, $K \in B(X)$ kompakt, dann ist $0 \in \sigma(K)$ und $\sigma(K)$ ist endlich oder besteht aus einer Nullfolge. \\
	Jedes $\lambda \in \sigma(K), \lambda \neq 0$ ist ein Eigenwert mit endlich dimensionalem Eigenraum.
\end{karte}

\begin{karte}{Zusammenhang abgeschlossener Operator und Spektrum}
	Sei $X \supset D(A) \xrightarrow[]{A} X$ ein abgeschlossener, linearer Operator, $\rho(A) \neq \emptyset$, $(D(A), \| \cdot \|_{A}) \hookrightarrow X$ kompakt. \\ \\
	Dann besteht $\sigma(A)$ aus endlich vielen Eigenwerten oder einer Folge von Eigenwerten mit $|\lambda_{n}| \rightarrow \infty$ und die zugehörigen Eigenräume sind endlich dimensional.
\end{karte}