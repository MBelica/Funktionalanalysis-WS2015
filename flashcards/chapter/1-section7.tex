\subsection*{7 - Kompakte Op.}

	\begin{karte}{kompakter Operator}
		Sei $X$ ein normierter Raum, Y ein Banachraum. Ein linearer Operator $T \colon X \rightarrow Y$ hei{\ss}t kompakt, falls $T(U_{X})$ relativ kompakt ist in $Y$.
	\end{karte}

	\begin{karte}{$K(X, Y)$}	
		$K(X, Y) =$ Raum der linearen, kompakten Operatoren von $X$ nach $Y$.
		
		Bemerkung:
		\begin{enumerate}[label=\alph*\upshape)]
			\item $T \in K(X, Y) \gdw$ jede beschränkte Folge $(x_{n}) \subset X$ besitzt eine Teilfolge $(x_{n_{k}})$ mit $T(x_{n_{k}})$ ist Cauchy-Folge in $Y$.
			\item $K(X, Y) \subset B(X, Y)$, da die kompakte Menge $\overline{T(U_{X})}$ beschränkt in $Y$ ist.
		\end{enumerate}	
	\end{karte}

	\begin{karte}{2x Eigenschaften von $K(X, Y)$}		
		Seien $X, Y$ und $Z$ Banachräume.
		\begin{enumerate}[label=\alph*\upshape)]
			\item $K(X, Y)$ ist ist ein linearer, \textit{abgeschlossener} Teilraum von $B(X, Y)$.
			\item Seien $T \in B(X, Y), S \in B(Y, Z)$ und entweder $T$ oder $S$ kompakt. Dann ist $S \circ T \in K(X, Z)$. \\
			Insbesondere: $K(X) = K(X, X)$ ist ein Ideal in $B(X)$.
		\end{enumerate}
	\end{karte}

	\begin{karte}{Folge endlich dimensionaler beschränkter Operatoren}		
		Seien $X, Y$ Banachräume, $T \in B(X, Y)$. \\
		Falls es endlich dimensionale Operatoren $T_{n} \in B(X, Y)$ gibt, dann ist $T \in K(X, Y)$.
		
		Beweis:
			Bemerkung nach 7.1, 7.5 a)	
	\end{karte}

	\begin{karte}{Folge der Approximationseigenschaft}			
		Seien $X, Y$ Banachräume und $X$ habe die \begriff{Approximationseigenschaft} (d.h. es existieren endlich dimensionale Operatoren $S_{n} \in B(X): S_{n} x \rightarrow x, \quad \forall x \in X$). \\ \\
		Dann gilt: $K(X, Y) = \overline{F(X, Y)}$ in der Operatornorm, wobei $F(X, Y) = \{ T \in B(X, Y): \dim T(X) < \infty \}$.
	\end{karte}