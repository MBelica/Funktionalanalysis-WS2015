\subsection*{6 - Kompakte Mengen}

	\begin{karte}{kompakt, folgenkompakt und relativ kompakt}
		Sei $(M, d)$ ein metrischer Raum. Eine Menge $K \subseteq M$ hei{\ss}t (folgen-)\begriff{kompakt}, falls es in jeder Folge $(x_{n}) \subset M$ eine Teilfolge $(x_{n_{k}})$ und ein $x \in K$ gibt, so dass 
		\[ \lim_{k \rightarrow \infty} x_{n_{k}} = x \]
		$K \subseteq M$ hei{\ss}t \begriff{relativ kompakt}, falls $\overline{K}$ in $M$ kompakt ist.	
	\end{karte}
	
	\begin{karte}{Kompaktheit der Einheitskugel}
		Sei $X$ ein normierter Vektorraum. Dann ist
		\[ \overline{U_{x}} = \{ x \in X: \| x \| \leq 1 \} \]
		genau dann kompakt, wenn $dim X < \infty$.
	\end{karte}
	
	\begin{karte}{Satz von Riesz}
		Sei $Y$ ein abgeschlossener Teilraum von $X$ und $X \neq Y$. Zu $\delta \in (0, 1)$ existiert ein $x_{\delta} \in X \setminus Y$, sodass
		\[ \| x \| = 1, \quad \| x_{\delta} - y\| \geq 1 - \delta \quad \text{ für alle } y \in Y \]
	\end{karte}
	
	\begin{karte}{Äquivalenzen zur Kompaktheit}
		Sei $(M, d)$ ein metrischer Raum. Für $k \subset M$ sind folgende Aussagen äquivalent zu $K$ ist (folgen-)kompakt:
		\begin{enumerate}[label=\alph*\upshape)]
			\item $K$ ist vollständig und total beschränkt, d.h. für alle $\epsilon > 0$ gibt es endlich viele $x_{1}, \dotsc, x_{m} \in M$ so dass ~ $K \subset \bigcup_{j = 1}^{m} K(x_{j}, \epsilon)$
			\item Jede Überdeckung von $K$ durch offene Mengen $U_{j}, j \in J$ mit $K \subset \bigcup_{j \in J} U_{j}$ besitzt eine endliche Teilüberdeckung, d.h. $j_{1}, \dotsc, j_{m}$ mit ~ $K \subset \bigcup_{k = 1}^{m} U_{j_{k}}$
		\end{enumerate}	
	\end{karte}
	
	\begin{karte}{4x abgeschlossene bzw. kompakte Mengen}
		Sei $(M, d)$ ein metrischer Raum.
	\begin{enumerate}[label=\alph*\upshape)]
		\item Eine kompakte Teilmenge $K \subset M$ ist immer vollständig und abgeschlossen in $M$.
		\item Eine abgeschlossene Teilmenge eine kompakten Raums ist kompakt.
		\item Jede kompakte Menge in $M$ ist separabel.
		\item Eine kompakte Teilmenge eines normierten Raums ist beschränkt.
	\end{enumerate}
	\end{karte}
		
	\begin{karte}{Arzelà-Ascoli}
	Sei $(S, d)$ ein kompakter, metrischer Raum. Definiere $C(S) \coloneqq \{ d \colon S \rightarrow \MdK \text{ stetig} \}$, $\| f \|_{\infty} = \sup_{s \in S} | f(s) |$. Eine Teilmenge $M \subset C(S)$ ist kompakt, genau dann wenn gilt
		\begin{enumerate}[label=\alph*\upshape)]
			\item $M$ ist beschränkt in $C(S)$,
			\item $M$ ist abgeschlossen in $C(S)$ und
			\item $M$ ist gleichgradig stetig, d.h.
				\[ \forall \epsilon > 0 \text{ } \exists \delta > 0 \text{ } \forall x \in M: d(s, t) < \delta \Rightarrow | x(s) - x(t) | < \epsilon \]
		\end{enumerate}
	\end{karte}

	\begin{karte}{Äquivalenzen zur relativen Kompaktheit}	
		Sei $X$ ein Banachraum. Für $K \subseteq X$ sind äquivalent
		\begin{enumerate}[label=\alph*\upshape)]
			\item $K$ relativ kompakt (d.h. $\overline{K}$ ist kompakt)
			\item Jede Folge $(x_{k}) \subseteq K$ hat eine Cauchy-Teilfolge
			\item $\forall \epsilon > 0$ $\exists y_{1}, \dotsc, y_{m} \in K$ mit $K \subseteq K(y_{1}, \epsilon) \cup \dotsc \cup K(y_{m}, \epsilon)$
		\end{enumerate}
	\end{karte}