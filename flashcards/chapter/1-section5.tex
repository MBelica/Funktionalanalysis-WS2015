\subsection*{5 - Vollst{\"a}ndigkeit}

	\begin{karte}{Cauchy-Folge}
		Sei $(M, d)$ ein metrischer Raum. $x_{n} \in M$ hei{\ss}t \begriff{Cauchy-Folge}, falls es zu jedem $\epsilon > 0$ ein $n_{0} \in \MdN$ gibt, sodass $\forall m, n \geq n_{0}$ gilt:
			\[ d(x_{n}, x_{m}) \leq \epsilon \]	
	\end{karte}
	
	\begin{karte}{Vollständigkeit}
		Sei $(M, d)$ ein metrischer Raum, dann hei{\ss}t $(M, d)$ \begriff{vollständig}, falls jede Cauchy-Folge $(x_{n}) \subset M$ einen Grenzwert \uline{in M} hat:
			\[ \lim_{n \rightarrow \infty} x_{n} = x \quad x \in M \]
		Ein normierter Raum $(X, \| \cdot \|)$ der vollständig ist bezüglich $d(x, y) = \| x - y \|$ heißt \begriff{Banachraum}.
	\end{karte}
	
	\begin{karte}{Cauchy-Folge vs. konvergente Folge}
		Jede konvergente Folge in $(M, d)$ ist eine Cauchy-Folge:
			\[ \text{Sei } \lim_{n \rightarrow \infty} x_{n} = x: ~ d(x_{n}, x_{m}) \leq d(x_{n}, x) + d(x, x_{m}) \rightarrow 0 \]
		 	\uline{Aber:} nicht jede Cauchy-Folge eines normierten Raums $X$ konvergiert in $ = C[0, 2]$:
			\[ \| f \|_{1} = \int_{0}^{2} | f(t) | dt, ~ f_{n}(x) = \begin{cases}x^{n} & \text{ für } x \in [0, 1] \\ 1 & \text{ für } x \in [1, 2]\end{cases} \]
	\end{karte}

	\begin{karte}{Raum der Abbildungen zwischen metrischen und Banachraum}	
		Sei $X$ ein metrischer Raum, $Y$ ein Banachraum.
		\[ C(X, Y) = \{ f \colon X \rightarrow Y: f \text{ stetig} \}, ~ \| f \|_{\infty} = \sup_{x \in X} \| f(x) \|_{Y} \]
		Dann ist $C(X, Y)$ ein (linearer) Banachraum.
	\end{karte}
	
	\begin{karte}{Vollständigkeit vs. äquivalente Normen}		
		Sind $\| \cdot \|_{1}, \| \cdot \|_{2}$ äquivalente Normen auf $X$ und ist dann $X$ bezüglich $\| \cdot \|_{1}$ vollständig, so auch bezüglich $\| \cdot \|_{2}$; da äquivalente Normen haben gleiche Cauchy-Folgen.
	
		Bsp.: $C^{1}[0, 1]$
		\[ \vertiii{f} = |f(0)| + \sup_{t \in [0, 1]} | f'(t) | \]
		Früher: $\vertiii{\cdot} \sim \| \cdot \|_{\infty} \Rightarrow \left( C[0, 1], \vertiii{\cdot} \right)$ ist vollständig.
	\end{karte}

	\begin{karte}{Abg. Teilmengen von BR vs metrische Räume}	
		Abgeschlossene Teilmengen von Banachräumen sind vollständige metrische Räume bezüglich \[ d(x, y) = \| x - y\| \]
	\end{karte}

	\begin{karte}{Raum der beschränkten Operatoren vollständig}		
		Sei $X$ ein normiert Raum, $Y$ ein Banachraum.
	Dann ist $B(X, Y)$ mit der Operatornorm vollständig. \\ \\
	Insbesondere: $X' = B(X, \MdK)$ ist immer  vollständig.
	\end{karte}
	
	\begin{karte}{Neumann'sche Reihe}
		Sei $A \in B(X)$, $X$ ein Banachraum mit $\| A \| < 1$. \\
		Dann ist $Id - A$ invertierbar und 
		\[ \left( Id - A \right)^{-1} = \sum_{n = 0}^{\infty} A^{n} \]
	\end{karte}

	\begin{karte}{J (surjektiver) Isomorphismus, A beschränkt mit $\| A \| < \| J^{-1} \|^{-1}$:
	
		~
		
		 $J - A$}	
		Sei $X$ ein Banachraum und $J: X \rightarrow X$ ein (surjektiver) Isomorphismus. \\
		Für $A \in B(X)$ und $\| A \| < \| J^{-1} \|^{-1}$ ist auch $J - A$ ein Isomorphismus \\ \\
		Insbesondere: $G = \{ T \in B(X): T \text{ stetig und invertierbar} \}$ ist eine offene Menge in $B(X)$.
	\end{karte}	
	
	\begin{karte}{Fortsetzung von Operatoren}
		Sei $X$ ein normierter Raum, $Y$ ein Banachraum und $D \subset X$ ein dichter Teilraum. \\
		Jeder linearere Operator $T: X \rightarrow Y$ mit
		\[ \| T x \|_{Y} \leq M \| x \|_{X}, \quad \text{für alle } x \in D \]
		lässt sich zu einem eindeutig bestimmten Operator $ \tilde T \in B(X, Y)$ mit $\| \tilde T \| \leq M	$ fortsetzen.
	\end{karte}
	
	\begin{karte}{Operatorgrenzwert auf dichter Menge}
		Sei $X$ ein normierter Banachraum, $D \subset X$ dicht in $X$ und sei eine Folge $T_{n} \in B(X, Y)$, wobei $(T_{n} x)$ eine Cauchy-Folge für jedes $x \in D$ sei. \\
		Dann gibt es genau einen Operator $T \in B(X, Y)$ mit
		\[ \lim_{n \rightarrow \infty} T_{n} x = T x \]
	\end{karte}

	\begin{karte}{Äquivalenz zur Vollständigkeit eines normierten Raums}	
		Für einen normierten Raum $(X, \| \cdot \|)$ sind äquivalent:
		\begin{enumerate}[label=\alph*\upshape)]
			\item $X$ ist vollständig
			\item Jede absolut konvergente Reihe $\sum_{n \geq 1} x_{n}$ mit $x_{n} \in X$ hat einen Limes in $X$.
		\end{enumerate}	
	\end{karte}
	
	\begin{karte}{Vollständigkeit des Quotientenraums}
		Sei $X$ ein Banachraum und $M \subset X$ ein abgeschlossener, linearer Teilraum. \\
		Dann $\hat X = \QR{X}{M}$ ist vollständig.
	\end{karte}
	
	\begin{karte}{Lipschitz}	
		Sei $X$ ein normierter Vektorraum, $M \subset X$ beliebig, $d(x, y) \coloneqq \| x - y \|$, wobei $x, y \in M$ und damit $(M, d)$ ein metrischer Raum. \\
		Eine Abbildung $f \colon M \rightarrow \MdR$ hei{\ss}t \begriff{Lipschitz}, falls
		\[ \sup_{x, y \notin M, x \neq y} \frac{|f(x) - f(y)|}{d(x, y)} = \underbrace{\| f \|_{L}}_{\begin{matrix} Lipschitz- \\ Konstante \end{matrix}} < \infty \]
		
		Dann ist $X = \{ f \colon M \rightarrow \MdR : f \text{ Lipschitz und } f(x_{0}) = 0 \}$	
	bezüglich $\| \cdot \|_{L}$ ein normierter Raum und $X' = B(X, \MdR)$ ist vollständig.
	\end{karte}

	\begin{karte}{Isometrische Einbettung in den Raum der Lipschitz-Funktionen}		
		Sei $(M,d)$ ein metrische Raum, $x_{0} \in M$ fest, $X$ definiert wie in \hyperref[def:5.15-	Lipschitz]{5.15}: \\ \\
		Zu $x \in M$ definiere $F_{x} \in X'$ durch $F_{x}(f) = f(x)$ für $f \colon M \rightarrow \MdR$ in $X$. \\
		Dann ist $x \in M \rightarrow F_{x} \in X'$ eine Abbildung, die eine isometrische Einbettung von $M$ nach $X'$ gibt, d.h. 
		\[ d(x, y) = \| F_{x} - F_{y} \|_{X'} \]
	\end{karte}
	
	\begin{karte}{Vervollständigung}
			Sei $(M, d)$ ein metrischer Raum. Ein vollständiger metrischer Raum $(\hat M, \hat d)$ hei{\ss}t \begriff{Vervollständigung} von $(M, d)$, falls es eine Einbettung $J \colon M \rightarrow \hat M$ gibt mit:
		\begin{enumerate}[label=\roman*\upshape)]
			\item $\hat d (J(x), J(y)) = d(x, y)$ für alle $x, y \in M$ (Isometrie)
			\item J(M) ist dicht in $\hat{M}$
		\end{enumerate}
	\end{karte}
	
	\begin{karte}{Existenz einer Vervollständigung}
		Zu jedem metrischen Raum $(M, d)$ gibt es eine Vervollständigung, die bis auf Isometrie eindeutig bestimmt ist.	
	\end{karte}
