\subsection*{2 - Normierte Räume}

	\begin{karte}{Norm}
	Sei $X$ ein Vektorraum über $\mathbb{K} \in \{ \mathbb{R}, \mathbb{C} \}$. Eine Abbildung  $\| \cdot \| \colon X \rightarrow \mathbb{R}_{+}$ hei{\ss}t \begriff{Norm}, falls
		\begin{description}
			\item[$\hspace{0.5cm} (N1) \hspace{0.1cm} $] $\| x \| \geq 0, \quad \| x \| = 0 \gdw x = 0 $
			\item[$\hspace{0.5cm} (N2) \hspace{0.1cm} $] $\| \lambda x \| = | \lambda \| x \| $
			\item[$\hspace{0.5cm} (N3) \hspace{0.1cm} $] $\| x + y \| \leq \| x \| + \| y \| $
		\end{description}
	\end{karte}
	
	\begin{karte}{Halbnorm}
		Falls $ \| \cdot \| $ all die Eigenschaften einer Norm erfüllt au{\ss}er 
		\[ \| x \| = 0 \Rightarrow x = 0, \]
		dann hei{\ss}t $ \| \cdot \| $ \begriff{Halbnorm}.
	\end{karte}
	
	\begin{karte}{Einheitskugel}
		Die Menge $ U_{X} = \{ x \in X:  \|x \| \leq 1 \}$ hei{\ss}t \begriff{Einheitskugel}.
	\end{karte}
	
	\begin{karte}{Im normierten Vektorraum konvergente Folge}
		Eine Folge $(x_{n})$ des normierten Raums $X$ \begriff{konvergiert} gegen ein $ x \in X $, falls 
		\[ \| x_{n} - x \| \xrightarrow[n \rightarrow \infty]{} 0. \]
	\end{karte}
	
	\begin{karte}{umgekehrte Dreiecksungleichung}
		Für zwei Elemente $x, y \in (X, \| \cdot \|)$ in normierten Räumen gilt auch die \begriff{umgekehrte Dreiecksungleichung} 
		\[ ( \left| \| x \| - \| y \| \right| \leq \| x - y \|) \]
	\end{karte}
	
	\begin{karte}{äquivalente Normen}
		Zwei Normen $\| \cdot \|_{1}, \| \cdot \|_{2}$ hei{\ss}en \begriff{äquivalent} auf $X$, falls es $0 < m, M < \infty$ gibt, so dass für alle $ x \in X$ gilt:
		\[ m \| x \|_{2} \leq \| x \|_{1} \leq M \| x \|_{2} \]
	\end{karte}
	
	\begin{karte}{äquivalente Normen + endlich dimensionalen Vektorraum}
		Auf einem endlich dimensionalen Vektorraum sind alle Normen äquivalent.
	\end{karte}
	
	\begin{karte}{äquivalente Normen + unendlich dimensionalen Vektorraum}
		Im unendlich dimensionalen Fall sind die Normen $\| \cdot \|_{p}$ auf $\MdF$ nicht äquivalent.
	
		Sei z.B. o.B.d.A. $p > q$ und setze 
		\[ x_{n} \coloneqq \sum_{j = 2^{n} + 1}^{2^{n + 1}} j^{-\frac{1}{p}}e_{j}, ~ e_{j} = ( \delta_{ij} )_{i \in \MdN} \]
	\end{karte}
	
	\begin{karte}{Äquivalenzen zu äquivalente Norm}
		Für zwei Normen $\| \cdot \|_{1}, \| \cdot \|_{2}$ auf $X$ sind äquivalent:
		\begin{enumerate}[label=\alph*\upshape)]
			\item $\| \cdot \|_{1}, \| \cdot \|_{2}$ sind äquivalent
			\item Für alle $(x_{n})_{n} \subset X$, $x \in X$ gilt $ \| x_{n} - x \|_{1} \rightarrow 0 \gdw \| x_{n} - x \|_{2} \rightarrow 0 $
			\item Für alle $(x_{n})_{n} \subset X$ gilt $\| x_{n} \|_{1} \rightarrow 0 \gdw \| x_{n} \|_{2} \rightarrow 0 $
			\item Es gibt Konstanten $0 < m$, $M < \infty$, so dass 
				\[ m U_{(X, \| \cdot \|_{1})} \subseteq U_{(X, \| \cdot \|_{2})} \subseteq M U_{(X, \| \cdot \|_{1})} \]
		\end{enumerate}
	\end{karte}
	
	\begin{karte}{Folgenraum}
		Wir definieren den \begriff{Folgenraum} mittels	
		\[ \MdF = \{ (x_{n}) \in \mathbb{K}^{\mathbb{N}}: x_{i} = 0 \text{ bis auf endlich viele } n \in \mathbb{N} \} \]
		  und $e_{j} = (0, \dotsc, 0, 1, 0, \dotsc, 0) $ der j-te Einheitsvektor in $\MdF$, wobei die $1$ an j-ter Stelle steht.
	\end{karte}		

	\begin{karte}{Minkowski-Ungleichung}
			\begriff{Minkowski-Ungleichung}: 
			
			\[ \left( \sum_{i=1}^{\infty} |x_{i} + y_{i}|^p\right)^{\frac{1}{p}} \leq\left( \sum_{i=1}^{\infty} |x_{i}|^p\right)^{\frac{1}{p}} + \left( \sum_{i=1}^{\infty} |y_{i}|^p\right)^{\frac{1}{p}} \]
	\end{karte}
	
	\begin{karte}{Hölder-Ungleichung}
		\begriff{Hölder-Ungleichung} mit $\frac{1}{p} + \frac{1}{p'} = 1 \text{ gilt; }$
		 
		 \[ \sum_{i=1}^{\infty} |x_{i}| |y_{i}| \leq \left( \sum_{i=1}^{\infty} |x_{i}|^{p} \right)^{\frac{1}{p}} \left( \sum_{i=1}^{\infty} |y_{i}|^{p'} \right)^{\frac{1}{p'}} \]	
	\end{karte}
	
	\begin{karte}{äquivalente Normen + unendlich dimensionale Räume}
		Im unendlich dimensionalen Fall sind die Normen $\| \cdot \|_{p}$ auf $\MdF$ nicht äquivalent.
		
		Bsp.: sei o.B.d.A. $p > q$ und setze 
		\[ x_{n} \coloneqq \sum_{j = 2^{n} + 1}^{2^{n + 1}} j^{-\frac{1}{p}}e_{j}, e_{j} = ( \delta_{ij} )_{i \in \MdN} \]
	\end{karte}
	
	\begin{karte}{Raum der beschränkten, m-fach stetig differenzierbaren Funktionen}	
	Definiere
		\[ C_{b}^{m}(\Omega) \coloneqq \{ f \colon \Omega \rightarrow \MdR : D^{\alpha}f \text{ sind für alle } \alpha \in \MdN^{n} \text{ stetig} \] \[ \text{ und beschränkt auf } \Omega, |\alpha| \leq m \}. \]	
		und versehen ihn mit der Norm 
			\[ \| f \|_{C_{b}^{m}} \coloneqq \sum_{|\alpha| \leq m} \| D^{\alpha}f \|_{\infty} \]
		Äquivalent dazu ist die Norm
		\[ \| f \|_{0} = \sum_{i = 0}^{m - 1} |f^{(i)}(0)| + \| f^{(m)} \|_{\infty} \]
	\end{karte}
	
	\begin{karte}{Quotientenraum}		
	Sei $(X, \| \cdot \|)$ ein normierter Raum und $M \subset X$ sei abgeschlossener (d.h. für alle $(x_{n}) \in M, \| x_{n} - x \| \rightarrow 0 \Rightarrow x \in M$), linearer Unterraum.
	Definiere $\hat X \coloneqq \QR{X}{M}$, dann ist $\hat x \in \QR{X}{M}$:
		\[ \hat x = \{ y \in X: y - x \in M \} = x + M \]
	Dabei gilt unter anderem $\hat x_{1} + \hat x_{2} = \widehat{x_{1} + x_{2}}$ und $\lambda \hat x_{1} = \widehat{\lambda x_{1}}$; $\hat X$ bildet somit einen Vektorraum. \\
	Definieren wir eine Norm für die Äquivalenzklassen mittels
		\[ \| \hat x \|_{\hat X} : = \inf \{ \| x - y \|_{X}: y \in M \} =: d(x, Y) \]
	$(\hat X, \| \cdot \|_{\hat X})$ ein normierter Raum.
	\end{karte}