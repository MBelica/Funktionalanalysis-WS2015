\subsection*{17 - Riesz}

\begin{karte}{Einbettung von $X$ in den dazugehörigen Dualraum}
	Sei $(X, \< \cdot, \cdot \>)$ ein Hilbertraum. $X \hookrightarrow X'$ \\
	Für jedes $x \in X$ erhält man ein stetiges, lineares Funktional $x' \colon X \rightarrow \MdK$ durch
		\[ x'(y) = \< y , x \> \quad \text{für } y \in X \]
		mit $\| x' \| = \sup \{ x'(y) : \| y \| = 1 \} = \| x \|_{X}$
\end{karte}

\begin{karte}{Riesz}
	Zu jedem $x' \in X'$ gibt es genau ein $x \in X$ mit 
	\[ x'(y) = \< y , x \> \quad \text{für } y \in X. \]
	und $ \| x' \|_{X'} = \| x \|_{X}$. Kurz: $X' \cong X$.
\end{karte}


\begin{karte}{Fortsetzung eines linearen Funktionals von Untervektorraum}
	Sei $X$ ein Hilbertraum, $M \subseteq X$ ein Untervektorraum und $y' \in M'$. Dann existiert ein $x' \in X'$ mit $x'|_{M} = y'$ und $\| y' \| = \| x' \|$.	
\end{karte}