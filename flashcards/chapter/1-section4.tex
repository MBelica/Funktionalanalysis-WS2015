\subsection*{4 - Metrische R{\"a}ume}

	\begin{karte}{Metrik}
		Sei $M$ eine nichtleere Menge. Eine Abbildung $d \colon M \times M \rightarrow 	\MdR$ hei{\ss}t \begriff{Metrik} auf $M$, falls $\forall x, y, z \in M:$
			\begin{description}
				\item[$\hspace{0.5cm} (M1) \hspace{0.1cm} $] $d(x, y) \geq 0, \hspace{0.25cm} d(x, y) = 0 \gdw x = y $  (positive Definitheit)
				\item[$\hspace{0.5cm} (M2) \hspace{0.1cm} $] $d(x, y) = d(y, x)$  (Symmetrie)
				\item[$\hspace{0.5cm} (M3) \hspace{0.1cm} $] $d(x, z) \leq d(x, y) + d(y, z)$  (Dreiecksungleichung)
			\end{description}
	\end{karte}
	
	\begin{karte}{Konvergente Folge im metrischen Raum}
		Eine Folge $(x_{n})_{n \geq 1} \subset M$ konvergiert gegen $x \in M$, falls
			\[ d(x_{n}, x) \rightarrow 0 \hspace{0.5cm} \text{für } n \rightarrow \infty \]	 
			Notation: $x = \lim_{n \rightarrow \infty} x_{n}$ (in $M$)
	\end{karte}
	
	\begin{karte}{Durch Halbnorm induzierte Metrik}
		Sei $X$ ein Vektorraum und $p_{j}$ für $j \in \MdN$ Halbnormen auf $X$ mit der Eigenschaft, dass für jedes $x \in X \setminus \{ 0 \}$ ein $K \in \MdN$ existiert mit $p_{K} > 0$. Dann definiert
			\[ d(x, y) \coloneqq \sum_{j \geq 1} 2^{-j} \frac{p_{j}(x - y)}{1 + p_{j}(x -y)}, \hspace{0.5cm} x, y \in X \]
			eine Metrik auf $X$ mit
			\[ d(x_{n}, x) \rightarrow 0 \gdw p_{j}(x_{n} - x) \rightarrow 0 \hspace{0.25cm} (n \rightarrow  \infty) \hspace{0.25cm} \forall j \in \MdN \]
	\end{karte}

	\begin{karte}{Abgeschlossen Menge}	
		Sei $(M, d)$ ein metrischer Raum. Eine Teilmenge $A \subset M$ hei{\ss}t \begriff{abgeschlossen} (in $M$), falls für alle in $M$ konvergenten Folgen $(x_{n})_{n \geq 1} \subset A$ der Grenzwert von $(x_{n})$ in $A$ liegt
	\end{karte}

	\begin{karte}{Offene Menge}	
		Eine Teilmenge $U \subset M$ hei{\ss}t \begriff{offen} (in $M$), falls zu jedem $x \in U$ ein $\epsilon > 0$ existiert, sodass
			\[ \{ y \in M: d(x, y) < \epsilon \} \subset U \]
			$A \subset M$ ist offen in $M$ genau dann, wenn $U = M \setminus A$ abgeschlossen ist
	\end{karte}
	
	\begin{karte}{Offene bzw. abgeschlossene Kugel}		
		Wir benutzen die Bezeichnungen
			\begin{itemize}
				\item \textbf{offene Kugel}: $K(x, r)  \coloneqq \{ y \in M: d(x, y) < r \}$
				\item \textbf{abgeschlossene Kugel}: $\bar K(x, r) \coloneqq \{ y \in M: d(x, y) \leq r \}$
			\end{itemize}
			mit $x \in M, r > 0$. Man sieht leicht, dass $K(x, r)$ offen und $\bar K(x, r)$ abgeschlossen ist.
	\end{karte}
	
	\begin{karte}{Offene Menge bezüglich diskreter Metrik}
		Bezüglich der diskreten Metrik $d$ aus \hyperref[bsp:1-diskreteMetrik]{Beispiel 4.2 b)} ist $\{x\} \subset M$ offen für jedes $x \in M$, da
			\[ K(x, r) = \{ x \} \subset \{ x \} \text{ für } r \in (0, 1] \]	
	\end{karte}

	\begin{karte}{Vereinigungen/Schnitte offener/abgeschlossener Mengen}
		Für eine beliebige Familie von abgeschlossenen Mengen $(A_{i})_{i \in I}$ sind 
			\[ A \coloneqq \bigcap_{i \in I} A_{i} \hspace{0.5cm} \text{ und } \hspace{0.5cm} A_{i_{1}} \cup \dotsc \cup A_{i_{N}} \hspace{0.25cm} (i_{1}, \dotsc, i_{N} \in I) \]
			abgeschlossen in $M$.
			
		Für eine beliebige Familie offenere Mengen $(U_{i})_{i \in I}$ sind
			\[ U \coloneqq \bigcup_{i \in I} U_{i} \quad \text{und} \quad U_{i_{1}} \cap \dotsc \cap U_{i_{N}} \qquad (i_{1}, \dotsc, i_{N} \in I) \] 
			offen in $M$.
	\end{karte}
	
	\begin{karte}{Abschluss, Innere und Rand}
		Sei $(M, d)$ ein metrischer Raum und $V \subset M$. Dann hei{\ss}t 


		$\bar V \coloneqq \bigcap \{ A \subset M: A$ ist abgeschlossen mit $V \subset A \} $ der \begriff{Abschluss} von $V$.
		
		$\mathring V \coloneqq \bigcup \{ U \subset M: U$ ist offen mit $U \subset V \}$ das \begriff{Innere} von $V$. 
		
		$ \partial V \coloneqq \bar V \setminus \mathring V$ der \begriff{Rand} von $V$.	
	\end{karte}
	
	\begin{karte}{Dicht}	
		Sei $(M, d)$ ein metrischer Raum. Eine Menge $V \subset M$ hei{\ss}t \begriff{dicht} in M, falls $\bar V = M$, d.h. jeder Punkt in $M$ ist Grenzwert einer Folge aus $V$.
	\end{karte}	

	\begin{karte}{Separabel}		
		Sei $(M, d)$ ein metrischer Raum, $M$ hei{\ss}t \begriff{separabel}, falls es eine abzählbare Teilmenge $V \subset M$ gibt, die dicht in $M$ liegt.
	\end{karte}
	
	\begin{karte}{Stetige Abbildung}
			Seien $(M, d_{M}), (N, d_{N})$ metrische Räume. Eine Abbildung $f \colon M \rightarrow N$ hei{\ss}t \textbf{stetig in $x_{0} \in M$}, falls für alle $(x_{n}) \subset M$ gilt
		\[ x_{n} \rightarrow x_{0} \text{ in } M \Rightarrow f(x_{n}) \rightarrow f(x_{0}) \text{ in } N \]
		\[ d_{M}(x_{n}, x_{0}) \rightarrow 0 (n \rightarrow \infty) ~ \Rightarrow ~ d_{N}(f(x_{n}), f(x_{0})) \rightarrow 0 \]
		Die Abbildung $f$ hei{\ss}t \begriff{stetig} \textbf{auf $M$}, falls $f$ in jedem Punkt von $M$ stetig ist.	
	\end{karte}

	\begin{karte}{Ist $\ell^{p}$ separabel?}	
		Die Räume $\ell^{p}, p \in [1, \infty)$ und $c_{0}$ sind separabel, da
			\[ D = lin \{ e_{k}, k \in \MdK \} \text{ dicht in allen Räumen liegt.} \]
		Der Raum $\ell^{\infty}$ ist nicht separabel: 
			Die Menge $\Omega$ der $\{0, 1\}$-wertigen Folgen ist überabzählbar. Für $x, y \in \Omega$ mit $x \neq y$ gilt $\| x - y \|_{\infty} = 1$
	\end{karte}
	
	\begin{karte}{Äquivalenzen zur Stetigkeit einer Abbildungen}		
		Die folgenden Aussagen sind äquivalent:
			\begin{enumerate}[label=(\roman*\upshape)]
				\item $f$ ist stetig auf $M$
				\item Ist $U \subset N$ offen, so ist auch $f^{-1}(U)$ offen in $M$
				\item Ist $A \subset N$ abgeschlossen, so ist auch $f^{-1}(A)$ abgeschlossen in $M$.
			\end{enumerate}
	\end{karte}