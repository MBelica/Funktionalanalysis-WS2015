\subsection*{10 - offenen Abbildung}

	\begin{karte}{Offene Abbildung}
			Eine Abbildung zwischen metrischen Räumen heißt \begriff{offen}, wenn offene Mengen auf offene Mengen abgebildet werden.
	\end{karte}
	
	\begin{karte}{Äquivalenzen zu offenem Operator}
				Seien $X, Y$ normierte Räume und $T: X \rightarrow Y$ ein linearer Operator, dann sind äquivalent:
		\begin{enumerate}[label=\alph*\upshape)]
			\item $T$ ist offen
			\item $\exists \epsilon > 0: K_{Y}(0, \epsilon) \subset T(K_{X}(0, 1))$
		\end{enumerate}
	\end{karte}
	
	\begin{karte}{Satz von der offenen Abbildung}
		Seien $X, Y$ Banachräume und $T \in B(X, Y)$, dann gilt:
		\[ T \text{ surjektiv} \gdw T \text{ offen} \]
	\end{karte}

	\begin{karte}{Bijektiver beschränkter Operator}	
		Seien $X, Y$ Banachräume und $T \in B(X, Y)$ bijektv, dann ist $T^{-1} \in B(Y, X)$	
	\end{karte}

	\begin{karte}{Beschränkte Einbettung zwischen Banachräumen}	
		Sei $X$	ein Vektorraum der sowohl mit $\| \cdot \|$ als auch mit $\vertiii{\cdot}$ ein Banachraum ist. Gilt 
		\[ \exists c > 0: \| x \| \leq c \cdot \vertiii{x}, ~ \forall x \in X, \]
		dann sind die Normen äquivalent, d.h. $\exists \hat{c} $ mit 
		\[ \hat{c} \cdot \vertiii{x} \leq \| x \| ~ \forall x \in X \leq c \cdot \vertiii{x} \]
	\end{karte}